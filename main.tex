\documentclass{book}
\usepackage[utf8]{inputenc}
\usepackage[T1]{fontenc}
\usepackage[italian]{babel}
\usepackage{hyperref}
\usepackage{geometry}
\geometry{a4paper, top = 2.5cm, bottom = 3cm, left = 2.5cm, right = 2.5cm, heightrounded, bindingoffset = 5mm}
\usepackage{graphicx}
\usepackage{wrapfig}
\usepackage[labelformat=empty]{caption}
\usepackage{amsmath, amssymb, amsthm, amstext}
\usepackage{mathtools}
\usepackage{bbm}
\usepackage{cancel}
\pagestyle{plain}

\title{Appunti di\\\Huge Meccanica Razionale\\e dei Continui\thanks{Corso tenuto dal professor Michele Correggi presso il Politecnico di Milano}}
\author{Daniele Vozza}
\date{Anno Accademico 2022/23}

\theoremstyle{plain}
\newtheorem{teo}{Teorema}[chapter]
\newtheorem*{teo*}{Teorema}
\theoremstyle{plain}
\newtheorem*{lemma}{Lemma}
\theoremstyle{plain}
\newtheorem*{pos}{Postulato}
\theoremstyle{plain}
\newtheorem*{cor}{Corollario}
\theoremstyle{plain}
\newtheorem{prop}{Proposizione}[chapter]
\newtheorem*{prop*}{Proposizione}
\theoremstyle{definition}
\newtheorem{defi}{Definizione}[chapter]
\newtheorem*{defi*}{Definizione}
\theoremstyle{remark}
\newtheorem*{oss}{Osservazione}
\theoremstyle{definition}
\newtheorem*{ex}{Esempio}


\begin{document}

\maketitle
\thispagestyle{empty}
\frontmatter
\hypersetup{hidelinks}
\tableofcontents
\mainmatter

\part{Primo parziale}

\chapter{Cinematica}

\section{Cinematica del punto materiale}

Un \textbf{punto materiale} è un qualunque sistema meccanico le cui dimensioni spaziali siano trascurabili rispetto alla scala di lunghezza del moto.

\begin{ex}
    Una palla da tennis lanciata dal 6° piano di un palazzo può essere trattata in prima approssimazione (per tempi piccoli) come un punto materiale, quindi trascurando gli effetti di attrito dell'aria.
\end{ex}

\begin{ex}
    Ai fini dei moti astronomici del sistema solare (rotazione attorno al sole), i pianeti possono in prima approssimazione essere considerati come punti materiali, quindi tralasciando ad esempio gli effetti di marea.
\end{ex}

\noindent Possiamo identificare ogni sistema meccanico che studiamo con il suo \textbf{spazio delle configurazioni}, ovvero con i possibili "stati" del sistema.

\begin{defi}
    Lo spazio delle configurazioni di un punto materiale in $d \in \mathbb{N}$  dimensioni è $\mathcal{E}_d$ (spazio euclideo $d$ dimensionale).
\end{defi}

\begin{defi*}
    Il \textbf{simbolo di Levi-Civita} $\varepsilon_{ijk}$ è un tensore a 3 indici con le seguenti proprietà:
    \begin{displaymath}
        \varepsilon_{ijk}=
    \begin{cases}
        +1 \text{ se } i, j, k \text{ sono una permutazione pari di } 1, 2, 3 \\
        -1 \text{ se } i, j, k \text{ sono una permutazione dispari di } 1, 2, 3 \\
        0 \text{ altrimenti}
    \end{cases}
    \end{displaymath}
\end{defi*}

\begin{defi*}
    La \textbf{delta di Kronecker} vale $\delta_{ij}=
    \begin{cases}
        1 \text{ se } i=j \\
        0 \text{ altrimenti}
    \end{cases}$
\end{defi*}

\begin{prop*}
    $\forall\;\overline{u}, \overline{v}, \overline{w} \in \mathbb{R}^3$
    \begin{gather*}
        \overline{u}\wedge(\overline{v}\wedge\overline{w})=(\overline{u}\cdot\overline{w})\overline{v}-(\overline{u}\cdot\overline{v})\overline{w} \\ (\overline{u}\wedge\overline{v})\wedge\overline{w}=(\overline{u}\cdot\overline{w})\overline{v}-(\overline{v}\cdot\overline{w})\overline{u}
    \end{gather*}
\end{prop*}

\begin{defi*}
    Lo \textbf{spazio euclideo} $\mathcal{E}_d$ $d$ dimensionale è un insieme di punti di punti $\mathcal{E}_d$, uno spazio vettoriale $V$, $\dim{V}=d$ e un'applicazione: $\mathcal{E}_d \times \mathcal{E}_d \to V$ data da $P, Q \to \overline{PQ}$ tale che:
    \begin{itemize}
	\item $\overline{PQ} = - \overline{QP}$;
	\item $\overline{PQ} + \overline{QR} = \overline{PR}$;
	\item $\forall P \in \mathcal{E}_d, \forall \overline{u} \in V, \exists! \; Q \in \mathcal{E}_d \mid \overline{PQ} = \overline{u}$.
    \end{itemize}
\end{defi*}

\noindent Quindi la configurazione di un punto materiale in $d$ dimensioni è univocamente determinata quando è asseganto un punto $P \in \mathcal{E}_d$ al tempo $t \in \mathbb{R}$. Tuttavia per caratterizzare ("misurare") la posizione del punto dobbiamo introdurre un riferimento.

\begin{defi}
	Un \textbf{osservatore} (o \textbf{sistema di riferimento}) in $\mathcal{E}_d$ è un insieme $\mathcal{O}=\{O;\, \hat{e}_1, \ldots, \hat{e}_d\}$ dove $O$ è un punto di $\mathcal{E}_d$ (\textbf{origine}) e $\{\hat{e}_1, \ldots, \hat{e}_d\}$ è una base ortonormale di $V$ (\textbf{assi coordinati}).
\end{defi}

\noindent Fissato l'osservatore, la configurazione del punto materiale si può assegnare con le \textbf{coordinate} $\overline{x} \in \mathbb{R}^d$ del punto $P$ rispetto a $\mathcal{O}$.

\begin{defi}
	Le coordinate $\overline{x}(t)$ del punto materiale rispetto a un sistema di riferimento si possono vedere come:
	\begin{itemize}
	\item punto nello spazio-tempo $\mathbb{R}^d \times \mathbb{R}$ che individua la \textbf{linea di universo} di $P$;
	\item funzione $\overline{x}(t): \mathbb{R} \to \mathbb{R}^3$ detta \textbf{traiettoria} di $P$.
	\end{itemize}
\end{defi}

\begin{oss}
	Per la "funzione" $P(t)$: $\mathbb{R} \to \mathcal{E}_d$ si usa il nome di \textbf{moto} del punto materiale.
\end{oss}

\begin{oss}
    Assumeremo sempre che $\overline{x}(\cdot) \in C^2([t_0,t_1])$ cioè ammetta sempre almeno 2 derivate continue.
\end{oss}

\begin{defi}
    Dato un punto materiale $P$, un osservatore $\mathcal{O}$ e assumendo che $\overline{x}(\cdot) \in C^{2}\left([t_0, t_1]\right)$:
	\begin{itemize}
	\item la \textbf{velocità} è la funzione 	$\overline{v}(t)=\dfrac{d\overline{x}}{dt}(t)=\dot{\overline{x}}(t)$;
	\item la sua \textbf{accelerazione} è $\overline{a}(t)=\dfrac{d^2\overline{x}}{dt^2}(t)=\ddot{\overline{x}}(t)$.
	\end{itemize}
\end{defi}

\section{Moto in coordinate intrinseche}

Consideriamo un punto materiale in $\mathcal{E}_d$ (3D) in moto lungo la traiettoria $\overline{x}(t)$ (rispetto all'osservatore $\mathcal{O}$).

\begin{prop}
    Sia $\overline{x}(\cdot) \in C^{2} ([t_{0}, t_{1}])$ la traiettoria di P. Assumendo che $\dot{\overline{x}}(t)$ abbia solo zeri isolati in $[t_0, t_1]$ allora la lunghezza dell'arco di curva percorsa da $t_0$ a $t \in [t_0, t_1]$ è:
    \begin{displaymath}
    \boxed{
        s(t)=\int_{t_{0}}^{t} \dot{s}(\tau)\,d\tau = \int_{t_{0}}^{t}|\dot{\overline{x}}(\tau)|\,d\tau
        }
    \end{displaymath}
\end{prop}

\begin{oss}
    $s(t)$ è detta \textbf{ascissa curvilinea} o, come funzione: $\mathbb{R} \to \mathbb{R}^{+}$ \textbf{legge oraria} del moto del punto.
    
\end{oss}

\begin{oss}
     $s(t)$ è una funzione monotona crescente, cioè $s(t_1) < s(t_2) \iff t_1 < t_2$. Di conseguenza $s(\cdot)$ è invertibile, cioè  $\exists \; t(\cdot): \mathbb{R}^{+} \to \mathbb{R}$ t.c. $s(t(\sigma))=\sigma$ e
\begin{displaymath}
    \frac{ds}{dt}=\frac{1}{\frac{dt}{ds}} \iff \frac{dt}{ds}(s)=\frac{1}{|\overline{v}(t(s))|} \quad \forall t \mid \overline{v}(t) \neq \overline{0}
\end{displaymath}

\noindent Possiamo quindi scegliere di parametrizzare la curva equivalentemente con il tempo $t \in \mathbb{R}$ oppure con la distanza percorsa $s \in \mathbb{R}^{+}$.
\end{oss}

\begin{defi}
    Sotto le ipotesi fatte sulla traiettoria $\overline{x}(t)$, è definita una riparametrizzazione $\overline{\gamma}: [0, L] \to \mathbb{R}^{d}$ data da $\overline{\gamma}(s)=\overline{x}(t(s))$ e i versori $(\forall s \mid \overline{\gamma}^{\prime}(s)\neq \overline{0})$:
    
    \begin{itemize}
        \item \textbf{vettore tangente} $\hat{\tau}(s):=\overline{\gamma}^{\prime}(s)$;
        \item \textbf{vettore normale} $\hat{n}(s):=\dfrac{\hat{\tau}^{\prime}(s)}{|\hat{\tau}^{\prime}(s)|}$;
        \item \textbf{vettore binormale} $\hat{b}(s):=\hat{\tau}(s) \wedge \hat{n}(s)$.
    \end{itemize}

\end{defi}

\begin{defi}
    La quantità positiva $k: [0, L] \to \mathbb{R}^{+}$ data da $k(s):= |\hat{\tau}^{\prime}(s)|$ è detta \textbf{curvatura della traiettoria}.
\end{defi}

\begin{defi}
    La quantità positiva $h: [0, L] \to \mathbb{R}^{+}$ data da $h(s):= |\hat{b}^{\prime}(s)|$ è detta \textbf{torsione della curva}.
\end{defi}

\begin{oss}
    In 2D sono sufficienti i versori $\hat{\tau}$ e $\hat{n}$ (e quindi solo la curvatura della curva gioca un ruolo).
\end{oss}

\begin{teo}[formule di Frenet-Serret]
    Sotto le ipotesi fatte sulla traiettoria $\overline{x}(t)$ e $\forall s \in [0, L]$ t.c. $\overline{\gamma}^{\prime}(s) \neq \overline{0}$:
    \begin{itemize}
        \item $\hat{\tau}(s)$, $\hat{n}(s)$, $\hat{b}(s)$ formano una terna ortonormale in $\mathbb{R}^{3}$;
        \item
        $\begin{dcases}
        \hat{\tau}^{\prime}(s):=k(s)\hat{n}(s)\\\hat{n}^{\prime}(s):= - k(s)\hat{\tau}(s) + h(s)\hat{b}(s)\\\hat{b}(s) = - h(s)\hat{n}(s)
        \end{dcases}$.
    \end{itemize}
\end{teo}

\begin{proof}
    I vettori hanno ovviamente modulo unitario, quindi è sufficiente verificare l'ortogonalità:
    \begin{itemize}
        \item $\hat{\tau}\cdot\hat{n}=\frac{\hat{\tau}\cdot\hat{\tau}'}{|\hat{\tau}'|}=\frac{1}{2|\hat{\tau}'|}\cdot(|\hat{\tau}|^2)'=0$ perché il modulo di $\hat{\tau}$ è costantemente uguale a 1.
        \item $\hat{n}\cdot\hat{b}=\hat{\tau}\cdot\hat{b}=0$ per ortogonalità del prodotto vettore ai 2 fattori.
    \end{itemize}
    Inoltre per definizione
    \begin{itemize}
        \item $\hat{\tau}'=\frac{\hat{\tau}'}{|\hat{\tau}'|}|\hat{\tau}'|=k\hat{n}$;
        \item $\hat{n}'\perp\hat{n}$ cioè $\hat{n}'=\alpha\hat{\tau}+\beta\hat{b}$ ma $(\hat{\tau}\cdot\hat{n})'=0 \implies \alpha =\hat{\tau}\cdot\hat{n}'=-\hat{\tau}'\cdot\hat{n}=-k$ e $\beta=\hat{n}'\cdot\hat{b}=-\hat{n}\cdot\hat{b}'=h$ poiché $(\hat{b}\cdot\hat{n})'=0$ e $\hat{b}'=-h\hat{n}$;
        \item Infatti $\hat{b}'=\cancel{\hat{\tau}'\wedge\hat{n}}+\hat{\tau}\wedge\hat{n}'=\beta\,\hat{\tau}\wedge\hat{b}=-\beta\hat{n}$ e $\hat{b}'=h\frac{\hat{b}'}{|\hat{b}'|}$ da cui si può scegliere $\beta=h$ e $\frac{\hat{b}'}{|\hat{b}'|}=-\hat{n}$.
    \end{itemize}
\end{proof}

\begin{prop}
    Sotto le ipotesi fatte sulla traiettoria $\overline{x}(t)$ si ha:
    \begin{itemize}
        \item $\overline{v}(t)=\dot{s}(t)\hat{\tau}$;
        \item $\overline{a}(t)=\ddot{s}(t) \hat{\tau}+k \dot{s}^{2}(t) \hat{n}$.
    \end{itemize}
\end{prop}

\begin{proof}

    \noindent
    \begin{itemize}
        \item $\overline{v}(t)=\Dot{\overline{x}}(t)=\overline{\gamma}'(s(t))\Dot{s}(t)=\Dot{s}\hat{\tau}$
        \item $\overline{a}(t)=\Ddot{s}\hat{\tau}+\Dot{s}\Dot{\hat{\tau}}$ ma $\Dot{\hat{\tau}}=\hat{\tau}'\cdot\Dot{s}=\Dot{s}k\hat{n}$
    \end{itemize}
\end{proof}

\begin{oss}
    L'accelerazione ha sempre una componente normale alla curva, se il punto si muove, tranne che nel caso in cui $k = 0$, cioè il moto sia rettilineo.
\end{oss}

\section{Cinematica del Corpo Rigido}

\begin{defi}
    Un \textbf{corpo rigido} è un sistema di $N \in \mathbb{N} \cup\{+\infty\}$ punti materiali che soddisfano la propietà
    \begin{displaymath}
        \left|\overline{OP}_{i}(t)-\overline{OP}_{j}(t)\right| = cost \quad \forall t \in \mathbb{R}, \forall i, j \in\{1 \ldots N\}
    \end{displaymath}
    cioè la distanza tra i punti del sistema rimane costante nel tempo.
\end{defi}

\noindent Quante coordinate servono per identificare univocamente la configurazione di un corpo rigido? In 3D in assenza del vincolo di rigidità occorrerebbero $3N$ coordinate. Le condizioni di rigidità corrispondono a $\frac{1}{2}N(N-1)$ equazioni (\# coppie) ma queste equazioni non sono tutte indipendenti. In effetti se $N \geq 3$ quelle indipendenti sono $3N-6$.
\begin{defi}
    Un osservatore o un sistema di riferimento è \textbf{solidale} ad un corpo rigido se le posizioni $\mathcal{O}=\{O^{\prime}; \hat{e}_1, \hat{e}_2, \hat{e}_3\}$ dei punti del corpo rispetto a $\mathcal{O}$ non dipendono dal tempo.
\end{defi}

\begin{teo}
    Un sistema di punti materiali è un corpo rigido $\iff \exists \; \mathcal{O}$ solidale.
\end{teo}

\begin{proof}

    \noindent
    \begin{itemize}
        \item ($\impliedby$) Indichiamo con $\overline{x}_i'$ e $\overline{x}_j'$ le posizioni dei punti $P_i$ e $P_j$ nel sistema di riferimento solidale. Allora $\Dot{\left|\overline{x}_i'-\overline{x}_j'\right|^2}=\Dot{\left|\overline{O'P_i}-\overline{O'P_j}\right|^2}=2(\Dot{\overline{x}}_i'-\Dot{\overline{x}}_j')\cdot(\overline{x}_i'-\overline{x}_j')=0 \;\;\forall i, j \in \{1,\ldots,N\}$
        \item ($\implies$) Assumiamo che esistano 3 punti $P$, $Q$ e $R$ non collineari (altrimenti è sufficiente prendere uno degli assi $\hat{e}_j$ lungo il corpo e gli altri a caso).
        
        \noindent Allora gli angoli $\alpha, \beta, \gamma$ non dipendono dal tempo come segue dal teorema di Pitagora:
        \begin{displaymath}
            |\overline{QR}|^2=|\overline{PQ}|^2+|\overline{PR}|^2-2|\overline{PQ}||\overline{PR}|\cos{\alpha}
        \end{displaymath}
        
        \noindent Il sistema solidale si può quindi costruire come segue: $\hat{e}_1=\frac{\overline{PQ}}{|\overline{PQ}|}$, $\hat{e}_3=\frac{\overline{PQ}\wedge\overline{PR}}{|\overline{PQ}\wedge\overline{PR}|}$, $\hat{e}_2=\hat{e}_3\wedge\hat{e}_1$
    \end{itemize}
\end{proof}

\begin{prop}
    Per determinare univocamente la configurazione di un corpo rigido piano sono necessarie e sufficienti 3 coordinate.
\end{prop}

\begin{proof}
    Dobbiamo identificare le posizioni del sistema solidale $\{O'; \hat{e}_1, \hat{e}_2, \hat{e}_3\}$ rispetto a quello "fisso" $\{O; \hat{i}, \hat{j}, \hat{k}\}$.
    Poiché il corpo è piano possiamo identificare $\hat{e}_3\equiv\hat{k}$ con l'asse $\perp$ al piano del corpo. Resta quindi da determinare la posizione di $\hat{e}_1, \hat{e}_2$ rispetto $\hat{i}, \hat{j}$: per questo è sufficiente (e necessario) determinare la posizione di $O'$ (quindi $x_{O'}$ e $y_{O'}$) e l'angolo formato dagli assi: $(x_{O'}, y_{O'}, \theta)$.
\end{proof}

\begin{prop}
    Per determinare univocamente la configurazione di un corpo rigido tridiminesionale sono necessarie e sufficienti 6 coordinate.
\end{prop}

\begin{oss}
    La posizione di un punto nello spazio è assegnata univocamente $\iff$ sono assegnate le distanze da 3 punti fissi non collineari.
\end{oss}

\begin{prop*}
     Ogni rotazione attorno agli assi $\hat{i}, \hat{j}, \hat{k}$ è rappresentata da una matrice ortogonale della forma rispettivamente
     \begin{displaymath}
         R_1(\alpha_1)=
         \begin{pmatrix}
            1 & 0 & 0 \\
            0 & \cos{\gamma_1} & -\sin{\gamma_1} \\
            0 & \sin{\gamma_1} & \cos{\gamma_1}
         \end{pmatrix} \ \ \ \
         R_2(\alpha_2)=
         \begin{pmatrix}
            \cos{\gamma_2} & 0 & -\sin{\gamma_2} \\
            0 & 1 & 0 \\
            \sin{\gamma_2} & 0 & \cos{\gamma_2}
         \end{pmatrix} \ \ \ \
         R_3(\alpha_3)=
         \begin{pmatrix}
            \cos{\gamma_3} & -\sin{\gamma_3} & 0 \\
            \sin{\gamma_3} & \cos{\gamma_3} & 0 \\
            0 & 0 & 1
         \end{pmatrix}
     \end{displaymath}
     con $\gamma_1, \gamma_2, \gamma_3$ gli angoli di rotazione ($\cos{\gamma_i}=\alpha_i$).
\end{prop*}

\begin{defi}
    Gli \textbf{angoli di Eulero} $\theta, \varphi, \psi$ sono tre coordinate angolari che identificano univocamente la posizione di un sistema solidale $\hat{e}_1, \hat{e}_2, \hat{e}_3$ rispetto ad uno "fisso" $\hat{i}, \hat{j}, \hat{k}$ nell'ipotesi che $O=O^{\prime}$:
    \begin{itemize}
        \item $\theta$ angolo di \textbf{mutazione} $\in[0, \pi): \cos \theta:=\hat{e}_{3} \cdot \hat{k}$: è l'angolo formato da $\hat{e}_{3}$ e $\hat{k}$;
        \item $\varphi$ angolo di \textbf{precessione} $\in[0,2 \pi): \cos \varphi:=\hat{i} \cdot \frac{\hat{k} \wedge \hat{e}_3}{|\hat{k} \wedge \hat{e}_3|}$: è l'angolo formato da $\hat{i}$ con la \textbf{linea dei nodi} data dall'intersezione fra i piani $\hat{i}, \hat{j}$ e $\hat{e_1}, \hat{e_2}$;
        \item $\psi$ angolo di \textbf{rotazione propria} $\in[0,2 \pi): \cos \psi:=\hat{e}_{1} \cdot \frac{\hat{k} \wedge \hat{e}_{3}}{|\hat{k} \wedge \hat{e}_{3}|}$: è l'angolo formato da $\hat{e}_1$ e la linea dei nodi.
    \end{itemize}
\end{defi}

\begin{prop}
     Per ogni moto di un corpo rigido si ha
    \begin{displaymath}
    \boxed{
        (\overline{v}_{P}-\overline{v}_{Q}) \cdot \overline{PQ}=0 \quad \forall P, Q \in \mathcal{C} \quad \forall t \in \mathbb{R}
        }
    \end{displaymath}
\end{prop}

\begin{proof}
    $|\overline{x}_P-\overline{x}_Q|^2= cost \implies\Dot{|\overline{x}_P-\overline{x}_Q|^2}=0=2(\overline{v}_P-\overline{v}_Q)\cdot(\overline{x}_P-\overline{x}_Q)$
\end{proof}

\begin{teo}[Poisson]
    Dato un corpo rigido $\mathcal{C}$ e una terna ortonormale ad esso solidale $\hat{e}_1, \hat{e}_2, \hat{e}_3$, $\exists!$ vettore $\overline{\omega} \in \mathbb{R}^{3}$ detto \textbf{velocità angolare} t.c.
    \begin{displaymath}
    \boxed{
        \dot{\hat{e}}_{j}=\overline{\omega} \wedge \hat{e}_{j} \quad \forall j \in\{1,2,3\}
        }
    \end{displaymath}
\end{teo}

\begin{lemma}
    Dati $\overline{a}, \overline{b} \in \mathbb{R}^3$ con $\overline{a}, \overline{b} \neq \overline{0}$ e tali che $\overline{a}\cdot\overline{b}=0$, $\exists \; \overline{c} \in \mathbb{R}$ t.c. $\overline{b}=\overline{c}\wedge \overline{a}$ e $\overline{c}$ è unico se scelto $\perp \overline{a}, \overline{b}$.
\end{lemma}

\begin{proof} [Dimostrazione (Lemma)]
    Scegliamo un sistema di riferimento in modo che $\hat{e}=\frac{\hat{a}}{|\hat{a}|}$ e $\hat{j}=\frac{\hat{b}}{|\hat{b}|}$ poiché $\overline{a} \perp \overline{b}$. A questo punto ci chiediamo se $\exists \; \overline{c}=c\hat{k}$ t.c. $|\hat{b}|\hat{j}=c|\hat{a}|\hat{j}$ da cui $c=\frac{|\hat{b}|}{|\hat{a}|}$.
\end{proof}

\begin{proof}[Dimostrazione (Teorema)]
    $\forall j \in \{1, 2, 3\} \;|\hat{e}_j|^2=1\implies|\Dot{\hat{e}}_j|^2=2\,\hat{e}_j\cdot\Dot{\hat{e}}_j=0$. Per il lemma allora $\exists! \;\overline{\omega}_j(t)$ t.c. $\Dot{\hat{e}}_j=\overline{\omega}_j\wedge\hat{e}_j \; \forall j \in \{1, 2, 3\}$. Dobbiamo ora mostrare $\overline{\omega}_j=\overline{\omega}$ cioè il vettore è lo stesso. Poniamo dunque $\overline{\omega}=\frac{1}{2}\sum_{j=1}^3\hat{e}_j\wedge\Dot{\hat{e}}_j$ e mostriamo che per tale $\overline{\omega}$ le identità sono soddisfatte. Si ha
    \[
    \begin{split}
        \overline{\omega}(t)\wedge\hat{e}_j &=\frac{1}{2}\sum_{i=1}^3\left(\hat{e}_i\wedge\Dot{\hat{e}}_i\right)\wedge\hat{e}_j= \frac{1}{2}\sum_{i=1}^3\left[\left(\hat{e}_i\cdot\hat{e}_j\right)\Dot{\hat{e}}_i-\left(\Dot{\hat{e}}_i\cdot\hat{e}_j\right)\hat{e}_i\right] \\
        &=\frac{1}{2}\sum_{i=1}^3\delta_{ij}\Dot{\hat{e}}_i+\frac{1}{2}\sum_{i=1}^3\left(\hat{e}_i\cdot\Dot{\hat{e}}_j\right)\hat{e}_i=\frac{1}{2}\Dot{\hat{e}}_j+\frac{1}{2}\Dot{\hat{e}}_j=\Dot{\hat{e}}_j
    \end{split}
    \]
\end{proof}

\begin{teo}[Formula fondamentale della cinematica rigida]
    Per ogni corpo rigido $\mathcal{C}, \exists!$ vettore $\overline{\omega}(t) \in \mathbb{R}^{3}$ detto \textbf{velocità angolare} t.c.
    \begin{displaymath}
    \boxed{
        \overline{v}_{P}(t)-\overline{v}_{Q}(t)=\overline{\omega}(t) \wedge \overline{Q P}(t) \quad \forall P, Q \in \mathcal{C} \quad \forall t \in \mathbb{R}
        }
    \end{displaymath}
\end{teo}

\begin{proof}
\[
    \begin{split}
        \overline{v}_P-\overline{v}_Q&=\Dot{\overline{x}}_P-\Dot{\overline{x}}_Q=\Dot{\overline{OP}}-\Dot{\overline{OQ}}=\left(\Dot{\sum_{j=1}^3(x_{P,j}-x_{Q,j})\hat{e_j}}\right)=\sum_{j=1}^3(x_{P,j}-x_{Q,j})\Dot{\hat{e}}_j \\ &=\sum_{j=1}^3(x_{P,j}-x_{Q,j})\,\overline{\omega}\wedge\hat{e}_j=\overline{\omega}\wedge\sum_{j=1}^3(x_{P,j}-x_{Q,j})\hat{e}_j=\overline{\omega}\wedge\overline{OP}
    \end{split}
\]        
\end{proof}

\subsection{Moti e Atti di moto}

\begin{defi}
    Un moto di un corpo rigido $\mathcal{C}$ è:
    \begin{itemize}
        \item \textbf{traslatorio} se $\overline{v}_{P}=\overline{v}_{Q} \quad \forall P, Q \in \mathcal{C}$;
        \item \textbf{piano} se
        \begin{itemize}
            \item $\exists \, \hat{k} \in \mathbb{R}^{3}$ t.c. $\overline{v}_{P}(t) \perp \hat{k} \quad \forall P \in \mathcal{C} \quad \forall t \in \mathbb{R}$;
            \item $\overline{PQ}(t) \parallel \hat{k} \implies \overline{v}_{P}(t)=\overline{v}_{Q}(t)$.
        \end{itemize}
        \item \textbf{polare} se $\exists$ \textbf{polo} $O \in \mathcal{C}$ t.c. $\overline{v}_{O}(t)=\overline{0} \quad \forall t \in \mathbb{R}$;
        \item \textbf{rotatorio} se $\exists$ \textbf{asse di rotazione} $r$ t.c. $\forall P \in r$ $\overline{v}_{P}(t)=\overline{0} \quad \forall t \in \mathbb{R}$;
        \item \textbf{rototraslatorio} se è una combinazione di un moto rotatorio e uno traslatorio.
    \end{itemize}
\end{defi}

\begin{oss}
    Le prime due definizioni si applicano anche a sistemi di due punti.
\end{oss}

\begin{defi}
    Un sistema di punti materiali o un corpo rigido è \textbf{piano} se $\exists \, \hat{k} \in \mathbb{R}^{3}$ t.c. $\overline{x}_{j} \perp \hat{k} \quad \forall j \in \{1,\ldots,N\}$.
\end{defi}

\begin{oss}
    Un moto può essere piano anche se il sistema è 3D e in tal caso è sufficiente conoscere il moto del sistema su un piano opportuno per ricostruire il moto dell'intero sistema.
\end{oss}

\begin{prop}
     Il moto di un corpo rigido è traslatorio $\iff \overline{\omega}(t)=\overline{0}, \quad \forall t \in \mathbb{R}$.
\end{prop}

\begin{proof}

    \noindent
    \begin{itemize}
        \item ($\implies$) Se il corpo è in moto traslatorio la terna solidale deve rimanere costante (solo l'origine si sposta) e quindi $\Dot{\hat{e}}_j=\overline{0}$ ma per Poisson questo implica $\overline{\omega}(t)=\overline{0} \;\forall t \in \mathbb{R}$.
        \item ($\impliedby$) La formula fondamentale della cinematica rigida implica direttamente il risultato:
        
        \noindent $\overline{v}_P=\overline{v}_Q+\underbrace{\overline{\omega}(t)}_{=\overline{0}}\wedge\overline{QP}=\overline{v}_Q$
    \end{itemize}
\end{proof}

\begin{oss}
    un moto traslatorio non è necessariamente rettilineo e l'asse di rotazione ha equazione: $\overline{x}=\lambda \hat{k}+\overline{\omega}(t)$.
\end{oss}

\begin{prop}
     Il moto di un corpo rigido è rototraslatorio $\iff \overline{\omega}(t)=\overline{\omega}(t) \hat{k}$ con $\hat{k}$ costante.
\end{prop}

\begin{lemma}
    Un moto di un corpo rigido è rototraslatorio $\iff \exists \, \hat{e}_{j}$ solidale t.c. $\dot{\hat{e}}_{j}=\overline{0}$.
\end{lemma}

\begin{prop}
    Un moto rigido è piano $\iff
    \begin{cases}
        \exists \, \hat{k} \in \mathbb{R}^{3}$ t.c. $\overline{\omega}(t)=\omega(t) \hat{k}; \\ \exists \, Q \in C$ t.c. $\left(\overline{x}_{Q}(t)-\overline{x}_{Q}(0)\right) \cdot \hat{k}=0 \quad \forall t \in \mathbb{R}.
    \end{cases}$
\end{prop}

\begin{proof}

    \noindent
    \begin{itemize}
        \item ($\impliedby$) Dimostriamo per prima cosa che $\overline{v}_P \perp \hat{k}\; \forall P \in \mathcal{C}$: per ipotesi su $Q$ abbiamo che $\overline{v}_Q\cdot \hat{k}=\Dot{\overline{x}}_Q(t)\cdot\hat{k}=\Dot{\left(\overline{x}_Q(t)-\overline{x}_Q(0)\right)}\cdot\hat{k}=0$.
        
        D'altra parte, per ogni altro punto $P \in \mathcal{C}$, la formula fondamentale ci dà $\overline{v}_P=\overline{v}_Q+\omega(t)\hat{k}\wedge\overline{QP}$ e quindi perché $\hat{k}\wedge\overline{QP}\perp\hat{k}$ ne deduciamo che $\overline{v}_P\perp\hat{k} \;\forall P \in \mathcal{C}$. Inoltre se $\overline{QP}\parallel\hat{k}\implies\overline{v}_P=\overline{v}_Q$.
        \item ($\implies$) Presi due punti $P, Q$ t.c. $\overline{QP} \parallel \hat{k}$ si ha $\overline{0}=\overline{v}_Q-\overline{v}_P=|\overline{QP}|\,\overline{\omega}\wedge\hat{k}$ da cui $\overline{\omega}(t)=\omega(t)\hat{k}$. 
        
        \noindent Se invece non esistesse una tale coppia di punti vorrebbe dire che il corpo è piano (nel piano $\perp\hat{k}$) e $\omega\parallel\hat{k}$.
        
        \noindent Se $\overline{v}_Q\perp\hat{k}, \overline{v}_Q=(v_1, v_2, 0)\implies\overline{x}_Q(t)-\overline{x}_Q(0)\perp\hat{k}$.
    \end{itemize}
\end{proof}

\begin{cor}
    Ogni moto rigido piano è anche rototraslatorio.
\end{cor}

\begin{defi}
    Dato un sistema di punti materiali, il suo \textbf{atto di moto} $\mathcal{A}(t)$ al tempo $t$ è la collezione di tutte le velocità dei punti a quell'istante:
    \begin{displaymath}
        \mathcal{A}(t)=\left\{\left(\overline{x}_{j}(t); \overline{v}_{j}(t)\right), j \in\{1,\ldots, N\}\right\}
    \end{displaymath}
\end{defi}

\begin{oss}
    Per un corpo rigido, assegnare l'atto di moto al tempo $t$ equivale a fornire la distribuzione nello spazio delle velocità dei punti del corpo.
\end{oss}

\begin{defi}
    Un atto di moto si dice:
    \begin{itemize}
        \item \textbf{traslatorio} se $\overline{v}_{i}=\overline{v}_{j} \quad \forall i, j \in\{1, \ldots, N\}$;
        \item \textbf{piano} se
         \begin{itemize}
            \item $\exists \, \hat{k} \in \mathbb{R}^{3}$ t.c. $\overline{v}_{j} \cdot \hat{k} = 0 \quad \forall j \in \{1, \ldots, N\}$;
            \item $\overline{x}_i - \overline{x}_j \parallel \hat{k} \implies \overline{v}_{i} = \overline{v}_{j}$;
        \end{itemize}
        \item \textbf{rigido} se la distribuzione delle velocità è compatibile con la conservazione della distanza fra i punti.
        \item \textbf{rototraslatorio} se $\exists \, \hat{e}$ t.c. $\forall P, Q \in \mathcal{C}$, $\overline{PQ} \parallel \hat{e} \Longrightarrow \overline{v}_{P}=\overline{v}_{Q}$;
        \item \textbf{elicoidale} se è rototraslatorio ed $\exists \; r: \overline{x}=\lambda \hat{e} + \overline{\omega}$ con $\overline{v}_{P}=\overline{v}_{Q}=\mu \hat{e} \quad \forall P, Q \in r \cap \mathcal{C}$;
        \item \textbf{rotatorio} se è rototraslatorio e $\exists \, r_{ir}$ (\textbf{asse di istantanea rotazione}) $: \overline{x}=\lambda \hat{e}+\overline{\omega}$ t.c. $\overline{v}_{P}=\overline{0} \quad \forall P \in r_{ir} \cap \mathcal{C}$.
    \end{itemize}
\end{defi}

\begin{prop}
    Ogni atto di moto rigido è rototraslatorio.
\end{prop}

\begin{proof}
    Per la formula fondamentale della cinematica rigida $\overline{v}_P-\overline{v}_Q=\overline{\omega}\wedge\overline{QP}$ per cui $\hat{e}=\frac{\overline{\omega}}{|\overline{\omega}|}$
\end{proof}

\begin{oss}
    Dato un atto di moto rotatorio e un qualunque punto $Q \in r_{ir}$ si ha $\overline{v}_Q = \overline{\omega} \wedge \overline{QP}, \; \forall P \in \mathcal{C}$.
\end{oss}

\begin{defi}
    Dato un atto di moto rigido, l'\textbf{invariante scalare} è la quantità
    \begin{displaymath}
        \boxed{
        I=\overline{\omega} \cdot \overline{v}_{P}, P\in \mathcal{C}
        }
    \end{displaymath}
\end{defi}

\begin{oss}
    La definizione di invariante scalare è ben posto, nel senso che $I$ non dipende dalla scelta del punto $P \in \mathcal{C}$, grazie alla formula fondamentale della cinematica rigida.
\end{oss}

\begin{teo}[Mozzi]
    Un atto di moto rigido di velocità angolare $\overline{\omega}(t)$ è:
    \begin{itemize}
        \item traslatorio $\iff \overline{\omega}(t)=\overline{0}$;
        \item rotatorio $\iff \overline{\omega}(t) \neq \overline{0}$ e $I=0$, e in tal caso l'asse di istantanea rotazione ha equazione
        \begin{displaymath}
            r_{ir}: \overline{x}=\lambda \overline{\omega}+\frac{\overline{\omega} \wedge \overline{v}_P}{|\overline{\omega}|^{2}}+\overline{x}_{p}, P \in \mathcal{C}
        \end{displaymath}
        \item elicoidale $\iff \overline{\omega}(t) \neq \overline{0}$ e $I \neq \overline{0}$, e in tal caso l'asse di equazione
        \begin{displaymath}
            r_{Mozzi}: \overline{x}=\lambda \overline{\omega}+\frac{\overline{\omega} \wedge \overline{v}_{P}}{|\overline{\omega}|^{2}}+\overline{x}_{p}
        \end{displaymath}
        prende il nome di \textbf{asse di Mozzi} ed ha le seguenti propietà:
        \begin{itemize}
            \item $\forall P \in r_{Mozzi}, \; \overline{v}_P \parallel \overline{\omega}$ e $\overline{v}_P = \dfrac{I\overline{\omega}}{|\overline{\omega}|^2}$;
            \item $\forall P \in \mathcal{C}, |\overline{v}_P|$ è minima se $P \in r_{Mozzi}$;
            \item se $\overline{PQ} \parallel r_{Mozzi} \Longrightarrow \overline{v}_P = \overline{v}_Q$.
        \end{itemize}
    \end{itemize}
\end{teo}

\begin{proof}
    È sufficiente dimostrare le implicazioni ($\impliedby$) perché le altre si ottengono per esaurimento dei casi possibili.
    \begin{enumerate}
        \item Segue direttamente dalla formula fondamentale.
        \item Dobbiamo dimostrare che i punti di $r_{ir}$ hanno velocità nulla $\implies r_{ir}$ asse di istantanea rotazione $\implies$ moto rotatorio. Si ha $\forall \; Q \in r_{ir}$
        \[
        \begin{split}
            \overline{v}_Q&=\overline{v}_P+\overline{\omega}(t)\wedge\overline{PQ}=\overline{v}_P+\overline{\omega}(t)\wedge(\overline{x}_Q-\overline{x}_P)=\overline{v}_P+\overline{\omega}(t)\wedge\left(\cancel{\lambda\overline{\omega}}+\frac{\overline{\omega}\wedge\overline{v}_P}{|\overline{\omega}|^2}\right) \\
            &=\cancel{\overline{v}_P}+\frac{\overline{\omega}(t)\cdot\overline{v}_P}{|\overline{\omega}|^2}\overline{\omega}\cancel{-\frac{|\overline{\omega}|^2}{|\overline{\omega}|^2}\overline{v}_P}=\frac{I}{|\overline{\omega}|^2}\overline{\omega}
        \end{split}
        \]
        ma $I=0$ per ipotesi e quindi il risultato è dimostrato.
        \item Il conto precedente implica che $\forall\; Q \in r_{Mozzi}$, $\overline{v}_Q=\frac{I\overline{\omega}}{|\overline{\omega}|^2}$ mentre l'ultimo punto segue dalla formula fondamentale.

        \noindent Per calcolare il modulo della velocità di un generico punto $P$, chiamiamo $P_0$ la proiezione di $P$ sull'asse di Mozzi. Si ha $\overline{v}_P\parallel\overline{\omega}$ mentre $\overline{\omega}\wedge\overline{P_0P}\perp\overline{\omega}$ per cui
        \begin{displaymath}
            |\overline{v}_P|^2=|\overline{v}_{P_0}+\overline{\omega}\wedge\overline{P_0P}|=|\overline{v}_{P_0}|^2+|\overline{\omega}\wedge\overline{P_0P}|^2
        \end{displaymath}
        ma $\overline{P_0P}\perp\overline{\omega}$ perché $r_{Mozzi}\parallel\overline{\omega}$ per cui
        \begin{displaymath}
            |\overline{v}_P|^2=|\overline{v}_{P_0}|^2+|\overline{\omega}|^2|\overline{P_0P}|^2=\frac{I^2}{|\overline{\omega}|^2}+|\overline{\omega}|^2d^2
        \end{displaymath}
        dove $d$ è la distanza di $P$ dall'asse di Mozzi $\implies$ il modulo della velocità è minimo quando $d=0$.
    \end{enumerate}
\end{proof}

\begin{oss}
    Ogni atto di moto rotatorio è anche piano $\Longleftarrow \hat{k}=\frac{\overline{\omega}}{|\overline{\omega}|}$.
\end{oss}

\begin{cor}
    Ogni atto di moto rigido piano è:
    \begin{itemize}
        \item traslatorio $\iff \overline{\omega}(t)=\overline{0}$;
        \item rotatorio $\iff \overline{\omega}(t) \neq \overline{0}$, e in tal caso l'asse di istantanea rotazione ha equazione
        \begin{displaymath}
            r_{ir}: \overline{x}=\lambda \overline{\omega}+\frac{\overline{\omega} \wedge \overline{v}_{P}}{|\overline{\omega}|^{2}}+\overline{x}_{p}
        \end{displaymath}
        con $P$ generico punto di $\mathcal{C}$.
    \end{itemize}
\end{cor}
\begin{defi}
    Il punto $C$ t.c. $r_{ir} \cap \pi=\{C\}$ con $\pi$ piano del moto è detto \textbf{centro di istantanea rotazione} (\textbf{CIR)}.
\end{defi}

\begin{oss}
    In un atto di moto rigido piano la posizione del CIR permette di conoscere immediatamente l'intero atto di moto.
\end{oss}

\begin{oss}
    La posizione del CIR può anche essere fuori dal corpo rigido, ovvero il CIR non è necessariamente un punto di $\mathcal{C}$.
\end{oss}

\begin{teo}[Chasles]
    La posizione del CIR in un atto di moto rigido piano si trova:
    \begin{itemize}
        \item all'intersezione delle rette $\perp \overline{v}_{P}$ e $\overline{v}_{Q}$ passanti per $P$ e $Q$, se $\overline{v}_{P} \nparallel \overline{v}_{Q}$;
        \item sulla retta $\perp \overline{v}_{P} \parallel \overline{v}_{Q}$ in modo che sia soddisfatta la proporzione $\dfrac{\left|\overline{v}_{P}\right|}{|\overline{CP}|}=\dfrac{\left|\overline{v}_{Q}\right|}{|\overline{CQ}|}$.
    \end{itemize}
\end{teo}

\begin{proof}
    $\forall P \in \mathcal{C}, \overline{v}_P\perp\overline{CP}$ poiché $\overline{v}_P=\overline{\omega}\wedge\overline{CP}$ e quindi se le rette sono $\nparallel\exists!$ punto di intersezione.

    \noindent Se invece le rette sono $\parallel$ allora $|\overline{CP}|=|\overline{\omega}||\overline{v}_P|\sin{\alpha}$ e $|\overline{CQ}|=|\overline{\omega}||\overline{v}_Q|\sin{\alpha}$ da cui si ricava la proporzione.
\end{proof}

\begin{oss}
    Il teorema di Chasles non è sempre sufficiente a determinare la posizione del CIR, nel quel caso è necessario usare la definizione.
\end{oss}

\section{Cinematica Relativa}

Consideriamo 2 osservatori $\mathcal{O}=\{O;\, \hat{e}_1, \hat{e}_2, \hat{e}_3\}$ e $\mathcal{O}^{\prime}=\{O^{\prime};\, \hat{e}_1^{\prime}, \hat{e}_2^{\prime}, \hat{e}_3^{\prime}\}$.

\begin{prop}[Trasformate di Galileo]
    $\exists$ matrice ortogonale $R \in O_{3}$ t.c $\overline{x}_{P} = R\,\overline{x}_{P}^{\prime}+\overline{OO^{\prime}}$ e un vettore $\overline{\omega} \in \mathbb{R}^{3}$ t.c.
    \begin{displaymath}
        \overline{v}_{P}=\underbrace{\overline{v}_{P}^{\prime}}_{\text{vel. relativa}}+\underbrace{\overline{v}_{O^{\prime}}+\overline{\omega} \wedge \overline{x}_{P}^{\prime}}_{\text{vel. di trascinamento}}
    \end{displaymath}
\end{prop}

\begin{proof}
    Come sappiamo, la terna $\hat{e}_j'$ è ottenuta dalla terna $\hat{e}_j$ con una rotazione, cioè $\hat{e}_j'=\sum_{j=1}^3R_{ij}^T\hat{e}_j$ o equivalentemente $\hat{e}_j=\sum_{j=1}^3R_{ij}\hat{e}_j'$.

    \noindent Poiché $\overline{OP}=\overline{OO'}+\overline{O'P}$ si ha
    \begin{displaymath}
        \sum_{j=1}^3x_j\hat{e}_j=\overline{OO'}+\sum_{j=1}^3x_j'\hat{e}_j'=\overline{OO'}+\sum_{j,k=1}^3x_j'R_{jk}^T\hat{e}_k
    \end{displaymath}
    da cui ricaviamo la tesi:
    \begin{displaymath}
        x_i=\overline{OO'}\cdot\hat{e}_i+\sum_{j,k=1}^3R_{kj}\cdot x_k'\delta_{ki}=\overline{OO'}\cdot\hat{e}_i+\sum_{j=1}^3R_{ij}\cdot x_j'
    \end{displaymath}
    Per determinare la legge di trasformazione delle velocità calcoliamo
    \[
    \begin{split}
        \overline{v}_P&=\Dot{\overline{x}}_P=\Dot{\overline{OO'}}+ \Dot{\left(\sum_{j=1}^3x_j'{e}_j'\right)}=\overline{v}_{O'}+\sum_{j=1}^3\left(\Dot{x}_j'\hat{e}_j'+x_j'\Dot{\hat{e}}_j'\right) \\
        &=\overline{v}_{O'}+\overline{v}_P'+\sum_{j=1}^3x_j'\overline{\omega}\wedge\hat{e}_j'=\overline{v}_{O'}+\overline{v}_P'+\overline{\omega}\wedge\overline{x}_P'
    \end{split}
    \]
\end{proof}

\begin{oss}
    Sia $\overline{\omega}(t)$ la velocità angolare di $\mathcal{O}^{\prime}$ rispetto a $\mathcal{O}$ e sia $\overline{\omega}^{\prime}(t)$ quella di $\mathcal{O}$ rispetto a $\mathcal{O}^{\prime}$, allora $\dot{\overline{\omega}}=\dot{\overline{\omega}^{\prime}}$, $\overline{\omega}=\overline{\omega}^{\prime}$.
\end{oss}

\begin{teo}[Coriolis]
    Dati due osservatori $\mathcal{O}$ e $\mathcal{O}^{\prime}$ con accelerazione relativa $\overline{a}_{O^{\prime}}$ e velocità angolare $\overline{\omega}$, (di $\mathcal{O}^{\prime}$ rispetto ad $\mathcal{O}$) si ha:
    \begin{displaymath}
        \overline{a}_{P}=\overline{a}_{P}^{\prime}+\overbrace{\overline{a}_{O^{\prime}}+\dot{\overline{\omega}} \wedge \overline{x}_{P}^{\prime}+\underbrace{\overline{\omega} \wedge\left(\overline{\omega} \wedge \overline{x}_{P}^{\prime}\right)}_{\text{acc. centrifuga}}}^{\text{acc. di trascinamento}}+\underbrace{2\,\overline{\omega} \wedge \overline{v}_{P}^{\prime}}_{\text{acc. di Coriolis}}
    \end{displaymath}
\end{teo}

\begin{proof}
    Deriviamo rispetto al tempo la legge di trasformazione delle velocità:
    \begin{displaymath}
        \Dot{\overline{v}}_P=\overline{a}_P=\Dot{\overline{v}}_P'+\Dot{\overline{v}}_{O'}+\Dot{\overline{\omega}}\wedge\overline{x}_P'+\overline{\omega}\wedge\Dot{\overline{x}}_P'
    \end{displaymath}
    ma
    \begin{itemize}
        \item $\Dot{\overline{v}}_P'=\sum_{i=1}^3\Dot{(\Dot{x}_i'\hat{e}_i')}=\sum_{i=1}^3(\Ddot{x}_i'\hat{e}_i'+\Dot{x}_i'\,\overline{\omega}\wedge\hat{e}_i')=\overline{a}_P'+\overline{\omega}\wedge\overline{v}_P'$;
        \item $\overline{\omega}\wedge\Dot{\overline{x}'}_P=\overline{\omega}\wedge\sum_{i=1}^3(\Dot{x}_i'\hat{e}_i'+x_i'\Dot{\hat{e}}_i')=\overline{\omega}\wedge\overline{v}_P'+\overline{\omega}\wedge\sum_{i=1}^3x_i'\,\overline{\omega}\wedge\hat{e}_i'=\overline{\omega}\wedge\overline{v}_P'+\overline{\omega}\wedge(\overline{\omega}\wedge\overline{x}_p')$;
        \item $\Dot{\overline{v}}_{O'}=\overline{a}_{O'}$.
    \end{itemize}
\end{proof}

\begin{oss}
    Affinché $\mathcal{O}$ e $\mathcal{O}^{\prime}$ abbiano le stesse accelerazioni $\left(\overline{a}_{P}=\overline{a}_{P}^{\prime}\right)$ deve essere $\overline{a}_{O^{\prime}}=\overline{\omega}=\dot{\overline{\omega}}=\overline{0}$, ovvero $\mathcal{O}$ e $\mathcal{O}^{\prime}$ devono essere osservatori relativamente \textbf{inerziali} (moto rettilineo uniforme).
\end{oss}

\begin{cor}[Rivals]
    Dati due punti $P$ e $Q$ di un corpo rigido si ha
    \begin{displaymath}
    \boxed{
        \overline{a}_{P}=\overline{a}_{Q}+\dot{\overline{\omega}} \wedge \overline{QP}+\overline{\omega} \wedge(\overline{\omega} \wedge \overline{QP})
        }
    \end{displaymath}
\end{cor}

\begin{proof}
    È sufficiente usare la legge di trasformazione per le accelerazioni e applicarla al sistema solidale e a quello fisso: entrambi i punti sono fissi nel sistema solidale per cui
    \begin{displaymath}
        \overline{a}_P=\overline{a}_{O'}+\Dot{\overline{\omega}}\wedge\overline{x}_P'+\overline{\omega}\wedge(\overline{\omega}\wedge\overline{x}_P') \quad \quad \quad \overline{a}_Q=\overline{a}_{O'}+\Dot{\overline{\omega}}\wedge\overline{x}_Q'+\overline{\omega}\wedge(\overline{\omega}\wedge\overline{x}_Q')
    \end{displaymath}
    Facendo la differenza si ottiene il risultato.
\end{proof}

\begin{oss}
    Rivals si può ricavare derivando rispetto al tempo la formula fondamentale della cinematica rigida.
\end{oss}

\begin{cor}
    Dato un corpo rigido con $\overline{\omega}$, $\dot{\overline{\omega}} \neq \overline{0}$, $\exists!\;C$ (\textbf{centro delle accelerazioni}) t.c. $\overline{a}_{C}=\overline{0}$ e $\forall P \in \mathcal{C}$
    \begin{displaymath}
        \overline{PC}=\frac{\dot{\overline{\omega}} \wedge \overline{a}_{P}-\overline{\omega}_{P} \wedge\left(\overline{\omega}^{\prime} \wedge \overline{a}_{P}\right)}{\omega^{4}+\dot{\omega}^2}
    \end{displaymath}
\end{cor}

\section{Moti e Atti di moto rigidi}

Riassunto dei moti piani:

\begin{itemize}
    \item traslatorio: atto di moto traslatorio $\forall t$;
    \item rotatorio: atto di moto rotatorio $\forall t$;
    \item rototraslatorio: atto di moto rotatorio $\forall t$ (l'asse di rotazione trasla con velocità $\dot{\overline{\omega}}$);
    \item polare: atto di moto rotatorio con $O$ CIR.
\end{itemize}

\begin{prop}[Composizione delle velocità angolari]
    Dato un corpo rigido $\mathcal{C}$, siano $\overline{\omega}_{\mathcal{C}}$ e $\overline{\omega}_{\mathcal{C}}^{\prime}$ le velocità angolari di $\mathcal{C}$ misurate rispetto a $\mathcal{O}$ e $\mathcal{O}^{\prime}$ ed $\overline{\omega}$ la velocità angolare relativa di $\mathcal{O}^{\prime}$ rispetto ad $\mathcal{O}$, allora:
    \begin{displaymath}
    \boxed{
        \overline{\omega}_{\mathcal{C}}=\overline{\omega}_{\mathcal{C}}^{\prime}+\overline{\omega}
        }
    \end{displaymath}
\end{prop}

\begin{proof}
    La formula fondamentale della cinematica rigida vale sia in $\mathcal{O}$ che in $\mathcal{O}'$, quindi $\forall P, Q \in \mathcal{C}$
    \begin{displaymath}
        \overline{v}_P=\overline{v}_Q+\overline{\omega}_{\mathcal{C}}\wedge\overline{QP} \quad \quad \quad \overline{v}_P'=\overline{v}_Q'+\overline{\omega}_{\mathcal{C}}'\wedge\overline{Q'P'}
    \end{displaymath}
    ma per la legge di trasformazione delle velocità
    \begin{displaymath}
        \overline{v}_P=\overline{v}_P'+\overline{\omega}\wedge\overline{x}_P'+\overline{v}_{O'} \quad \quad \quad \overline{v}_Q=\overline{v}_Q'+\overline{\omega}\wedge\overline{x}_Q'+\overline{v}_{O'}
    \end{displaymath}
    da cui ricaviamo
    \begin{displaymath}
        \overline{v}_P-\overline{v}_Q=\overline{\omega}_{\mathcal{C}}\wedge\overline{QP}=\overline{v}_P'-\overline{v}_Q'+\overline{\omega}\wedge\overline{Q'P'}=(\overline{\omega}_{\mathcal{C}}^{\prime}+\overline{\omega})\wedge\overline{Q'P'}
    \end{displaymath}
    ma $\overline{QP}=\overline{Q'P'}$ e quindi $\overline{\omega}_{\mathcal{C}}=\overline{\omega}_{\mathcal{C}}^{\prime}+\overline{\omega}$, poiché l'identità vale $\forall P, Q \in \mathcal{C}$.
\end{proof}

\begin{oss}
    La legge di composizione delle velocità angolari può essere utile per decomporre il moto di un corpo rigido in moti più semplici, introducendo osservatori mobili.
\end{oss}

\begin{defi}
    Un moto di un corpo rigido $\mathcal{C}$ è una \textbf{precessione} se si possono scegliere  $\mathcal{O}$ e $\mathcal{O}^{\prime}$ in modo che $\hat{e}_{3} \cdot \hat{k}= cost$, con $\hat{e}_{3}$ detto \textbf{asse di rotazione propria} e $\hat{k}$ \textbf{asse di precessione}.
\end{defi}

\begin{prop}
    Un moto di un corpo rigido è una precessione $\iff \overline{\omega}=\lambda\hat{k}+\mu\hat{e}_{3}$.
\end{prop}

\begin{proof}

    \noindent 
    \begin{itemize}
        \item ($\implies$) Se $\hat{e}_3\cdot\hat{k}=\cos{\theta}=cost\implies\Dot{\hat{e}}_3\cdot\hat{k}=(\overline{\omega}\wedge\hat{e}_3)\cdot\hat{k}=\overline{0}\implies\overline{\omega}, \hat{e}_3,\hat{k}$ sono complanari $\implies\overline{\omega}=\lambda\hat{k}+\mu\hat{e}_{3}$
        \item ($\impliedby$) $\frac{d}{dt}(\hat{e}_3\cdot\hat{k})=\Dot{\hat{e}}_3\cdot\hat{k}=(\overline{\omega}\wedge\hat{e}_3)\cdot\hat{k}=\lambda(\hat{k}\wedge\hat{e}_3)\cdot\hat{k}=\overline{0}$.
    \end{itemize}
\end{proof}

\section{Vincoli}

\begin{defi}
    Un \textbf{vincolo} è una qualunque restrizione sulle configurazioni e/o sui moti/atti di moto di un sistema di punti materiali.
\end{defi}

\noindent Assumeremo sempre che per ogni vincolo esista una funzione $F: \mathbb{R}^{dN} \times \mathbb{R}^{dN} \times \mathbb{R} \to \mathbb{R} \in C^{\infty}$ t.c. il vincolo sia scrivibile nella forma $F\left(\overline{x}_{1}, \ldots, \overline{x}_{N}; \; \overline{v}_{1}, \ldots, \overline{v}_{N};\; t\right) \geq 0$.

\noindent Diciamo che un vincolo è:

\begin{itemize}
    \item \textbf{olonomo} se non dipende dalle velocità;
    \item \textbf{anolonomo} se dipende dalle velocità;
    \item \textbf{fisso} se F non dipende esplicitamente dal tempo $t \in \mathbb{R}$;
    \item \textbf{bilatero} se della forma $F = 0$.
    \item \textbf{unilatero} se della forma $F \geq 0$.
\end{itemize}

\begin{oss}
    La distinzione fra vincoli olonomi e anolonomi è a volte artificiosa perchè spesso un vincolo olonomo ha conseguenze sulle velocità e a volte un vincolo anolomono si può riscrivere equivalentemente come vincolo olonomo. Quando questo succede si dice che il vincolo anolonomo è \textbf{integrabile}.
\end{oss}

\noindent \textbf{Problema}: quante coordinate occorrono per determinare univocamente le configurazioni di un sistema con vincoli olonomi bilateri?

\begin{ex}
    Per il corpo rigido sappiamo che la risposta è 3 in 2D e 6 in 3D.
\end{ex}

\begin{defi}
    Le \textbf{coordinate libere} o \textbf{Lagrangiane} $(q_1,\ldots,q_g) \in \mathbb{R}^g$ di un sistema di punti materiali sono un insieme minimale di parametri sufficienti (e necessari) per determinare univocamente la configurazione del sistema. Il loro numero $g \in \mathbb{N}$ si chiama \textbf{numero di gradi di libertà} del sistema.
\end{defi}

\begin{oss}
    Dato un sistema di $N$ punti materiali in assenza di vincoli $g_0=dN$ (indicheremo con $g_0$ per ricordare l'assenza di vincoli). \\Per un corpo rigido $g_0 =
    \begin{cases}
        3 \; \text{in 2D} \\ 6 \; \text{in 3D}
    \end{cases}$.
\end{oss}

\begin{defi}
    Dato un sistema di punti materiali sottoposto a $M$ vincoli olonomi e bilateri \textbf{varietà delle configurazioni} $\Sigma_t$ al tempo $t \in \mathbb{R}$ è:
    \begin{displaymath}
        \Sigma_{t}:=\left\{\left(\overline{x}_{1}, \ldots, \overline{x}_{N}\right) \in \mathbb{R}^{dN} \; | \; F_j\left(\overline{x}_{1}, \ldots, \overline{x}_{N}; t\right)=0, \; \forall j \in \{1,\ldots,M\}\right\}
    \end{displaymath}
\end{defi}

\begin{oss}
    In generale le $M$ equazioni vincolari non saranno indipendenti e quindi dobbiamo capire quante di queste sono necessarie.
\end{oss}

\begin{ex}
    Sappiamo già che in un corpo rigido delle $\frac{1}{2}N(N-1)$ equazioni vincolari solo $2N-3$ (2D) e $3N-6$ (3D) sono indipendenti.
\end{ex}

\begin{defi}
    Diciamo che $M$ vincoli olonomi e bilateri sono al tempo $t \in \mathbb{R}$:
    \begin{itemize}
        \item \textbf{compatibili} se $\Sigma_{t} \neq 0$, ovvero $\exists$ almeno una configurazione del sistema che soddisfa tutti i vincoli;
        \item \textbf{indipendenti} se il rango della matrice jacobiana è massimo, ovvero definendo
        \begin{displaymath}
            \overline{X}:=\left(\overline{x}_1, \ldots, \overline{x}_N\right) \in \mathbb{R}^{dN} \; \text{e} \; J_{ij}:=\frac{\partial F_{i}}{\partial X_{j}}, \quad i \in \{1,\ldots,M\}, \; j \in \{1,\ldots,dN\}
        \end{displaymath}
        richiediamo che
        \begin{displaymath}
            \operatorname{rank}J_{F}=\min \{dN, M\}
        \end{displaymath}
    \end{itemize}
\end{defi}

\begin{teo}
    Dato un sistema di $N$ punti materiali sottoposto a $M$ vincoli olonomi e bilateri, compatibili e indipendenti, fissato $t \in \mathbb{R}$, $\exists$ un intorno $D_t \subset \mathbb{R}^{dN}$ di $\Sigma_t$, un aperto $E_{t} \subset \mathbb{R}^{g}$ con $g=dN-M$ e una funzione $\Phi_t: E_{t} \to D_{t} \in C^{\infty}(E_t)$ e invertibile t.c.
    \begin{displaymath}
    \boxed{
        \overline{F}(\overline{\Phi}(\overline{q}); t)=\overline{0} \quad \forall\overline{q} \in E_t
        }
    \end{displaymath}
\end{teo}

\begin{proof}
    Il teorema è la versione multidimensionale del teorema della funzione implicita (teorema del Dini).
\end{proof}

\begin{oss}
    Il teorema precedente è la versione multidimensionale del teorema della funzione implicita (Dini).
\end{oss}

\begin{oss}
    I gradi di libertà di un sistema di $N$ punti materiali sottoposto a $M$ vincoli compatibili e indipendenti è $g=
    \begin{cases}
        dN - M \quad \text{se} \; M < dN \\ 0 \quad \quad \quad \quad \; \text{altrimenti}
    \end{cases}$.
\end{oss}

\begin{ex}

    \noindent 
    \begin{itemize}
        \item Punto materiale in $\mathbb{R}^d, \; g_0=d$;
        \item punto materiale vincolato nel piano $\hat{i},\hat{j}, \; g=2$ con coordinate libere $(x, y)\in\mathbb{R}^2$;
        \item punto vincolato a $P_0, \; g=0$;
        \item punto materiale in moto lungo una curva regolare $\Gamma,\,g=1$ con coordinata generalizzata $s$ ascissa curvilinea $\in \mathbb{R}^+$.
    \end{itemize}
\end{ex}

\begin{defi}
    Dato un punto materiale $P$, diciamo che $\overline{v}_{P}^{\prime}$ è una sua \textbf{velocità virtuale} se $\left(\overline{x}_{P}, \overline{v}_{P}^{\prime}\right)$ è un atto di moto possibile per $P$.
    
    \noindent Diciamo inoltre che $\delta \overline{x}_{P}^{\prime}$ è un suo \textbf{spostamento virtuale} se $\exists \; \delta t>0$ t.c. $\delta\overline{x}_{P}^{\prime}=\overline{v}_{P}^{\prime} \delta t$.
    
    \noindent L'insieme delle velocità virtuali e degli spostamenti virtuali sono indicati con $V_{P}^{\prime}\left(\overline{x}_{P}\right)$ e $S_{P}^{\prime}\left(\overline{x}_{P}\right)$.
\end{defi}

\begin{ex}
    Consideriamo un punto vincolato a muoversi lungo una guida curvilinea che si muove di moto traslatorio con velocità $\overline{v}$: la velocità virtuale ad ogni istante è tangenziale alla curva mentre $\overline{v}$ è qualunque.
\end{ex}

\begin{oss}
    Spostamenti e velocità virtuali dipendono in modo cruciale dai vincoli a cui è sottoposto $P$ e variano con il variare delle configurazioni di $P$.
\end{oss}

\begin{oss}
    Gli spostamenti e le velocità virtuali hanno a volte poco a che fare con gli spostamenti e le velocità reali.
\end{oss}

\begin{defi}
    Gli spostamenti e le velocità virtuali di un punto materiale sono \textbf{reversibili}, ogni volta che $\delta\overline{x}_{P}^{\prime}$ è uno spostamento virtuale anche $-\delta\overline{x}_{P}^{\prime}$ lo è (per le velocità vale la stessa cosa con $\overline{v}_{P}^{\prime}$).
\end{defi}

\begin{oss}
    Gli spostamenti virtuali sono reversibili $\iff$ le velocità virtuali lo sono.
\end{oss}

\begin{oss}
    Gli spostamenti e le velocità virtuali associati a vincoli bilateri sono \underline{sempre reversibili}.
\end{oss}

\begin{oss}
    Per un sistema di punti materiali gli spostamenti e le velocità virtuali sono la collezione degli spostamenti e delle velocità virtuali per ogni singolo punto.
\end{oss}

\subsection{Tipi di vincolo}

Consideriamo un sistema meccanico con $g_0$ gradi di libertà in assenza di vincoli e supponiamo che sia sottoposto a vincoli olonomi che portino il numero di gradi di libertà a $g=g_0-v$, con $v$ \textbf{gradi di vincolo}.

\noindent Possiamo determinare il \# di gradi di libertà senza calcolare il rango dello jacobiano, ma usando il conteggio empirico del \# vincoli e del grado di vincolo per ciascuno.

\noindent Attenzione: può succedere che i vincoli non siano in realtà indipendenti!

\begin{itemize}
    \item \textbf{Incastro}: $\overline{x}_Q(t)=\overline{x}_Q(0) \; \forall t \in \mathbb{R}$ e nessun moto del corpo relativo a $Q$ è possibile $\Longrightarrow g=g_0-g_0=0$.
    \item \textbf{Cerniera fissa}: $\overline{x}_Q(t)=\overline{x}_Q(0) \; \forall t \in \mathbb{R} \Longrightarrow$ il moto/atto di moto è rotatorio con asse di istantanea rotazione passante per $Q \Longrightarrow g=
    \begin{cases}
        g_0 - 2 \quad \text{se} \; d=2\\ g_0 - 3 \quad \text{se} \; d=3
    \end{cases}$.
    \item \textbf{Cerniera mobile}: $\overline{v}_{A_1}(t)=\overline{v}_{A_2}(t) \; \forall t \in \mathbb{R} \Longrightarrow g=
    \begin{cases}
        g_0 - 2 \quad \text{se} \; d=2\\ g_0 - 3 \quad \text{se} \; d=3
    \end{cases}$.
    \item \textbf{Manicotto}:
    $\overline{v}_{Q}(t)=\overline{v}_{Q}\hat{\tau}_Q(t)$ e nessun moto del corpo relativo a Q è ammesso $\Longrightarrow g=
    \begin{cases}
        g_0 - 2 \quad \text{se} \; d=2 \quad \Longrightarrow \text{se la guida è rettilinea il moto è traslatorio} \\ g_0 - 4 \quad \text{se} \; d=3 \quad \Longrightarrow \text{moto rototraslatorio con} \; \overline{\omega}(t)=\omega(t)\hat{\tau}_Q
    \end{cases}$.
    \item \textbf{Carrello}:
    $\overline{v}_{Q}(t)=\overline{v}_{Q}\hat{\tau}_Q(t)\Longrightarrow g=
    \begin{cases}
        g_0 - 1 \quad \text{se} \; d=2 \\ g_0 - 2 \quad \text{se} \; d=3
    \end{cases}$.
\end{itemize}

\noindent Fin'ora abbiamo visto solo vincoli bilateri e olonomi. Introduciamo ora anche vincoli unilateri e anolonomi.

\begin{itemize}
    \item \textbf{Contatto} (anolonomo e bilatero): quando due sistemi meccanici soddisfano il vincolo di contatto in un punto, una curva o una superficie vuol dire che $\forall P \in \Sigma_{contatto}$ si ha $\overline{v}_{P_1} \cdot \hat{n}=\overline{v}_{P_2} \cdot \hat{n}=0$, ovvero i punti dei due corpi a contatto hanno componente normale delle velocità nulle.
    \item \textbf{Appoggio} (unilatero): due corpi possono rimanere a contatto, ma anche separarsi, ovvero $\forall P \in \Sigma_{appoggio}$ si ha $\overline{v}_{P_1} \cdot \hat{n} \geq 0$.
    \item \textbf{Filo inestensibile} (unilatero e olonomo): due punti collegati da un filo soddisfano il vincolo $|\overline{PQ}| \leq l$, dove $l$ è la lunghezza massima del filo. Se però il filo è in tensione allora il vincolo diventa bilatero e prende la forma $\overline{v}_{P} \cdot \hat{\tau}=\overline{v}_{Q} \cdot \hat{\tau}$ dove $P$, $Q$ sono due punti qualunque lungo il filo e $\hat{\tau}$ è il vettore tangente alla curva del filo.
    \item \textbf{Piolo}: è un elemento meccanico su cui i corpi rigidi possono essere appoggiati (vincolo di appoggio) oppure vincolati con un carrello o un manicotto.
    \item \textbf{Carrucola}: è un elemento meccanico su cui un filo può scorrere (tipicamente senza attrito) per modificare il suo percorso quando in tensione.
\end{itemize}

\subsubsection{Puro rotolamento}

Il puro rotolamento è un vincolo anolonomo che può essere:

\begin{itemize}
    \item bilatero: il disco è vincolato a rotolare lungo la guida e quindi $y_A(t)=0 \; \forall t \in \mathbb{R}$;
    \item unilatero: il disco è appoggiato alla guida e quindi $\overline{v}_A \cdot \Hat{j} \geq 0 \; \forall t \in \mathbb{R} \; | \; y_A(t)=0$.
\end{itemize}

\noindent Il puro rotolamento è quindi dato da un vincolo di contatto (unilatero o bilatero) a cui si aggiunge un vincolo anolonomo che è il vero vincolo di puro rotolamento ("il disco ruota senza strisciare").

\noindent Il punto $A$, infatti, è in realtà la sovrapposizione di un punto del disco $A_{disco}$ con uno della guida $A_{guida}$: i punti coincidono $\forall t \in \mathbb{R}$ (anche se il punto della guida e del disco cambiano!), ovvero
\begin{displaymath}
\boxed{
    \overline{v}_{A_{disco}}(t)=\overline{v}_{A_{guida}}(t)
    }
\end{displaymath}
Se in aggiunta la guida è fissa $\overline{v}_{A_{guida}}(t)=\overline{0} \quad \forall t \in \mathbb{R}$.

\noindent Inoltre
\begin{displaymath}
    g=\underbrace{3}_{g_0}-\underbrace{1}_{\text{contatto}}-\underbrace{1}_{\text{puro rotolamento}}=1
\end{displaymath}

\noindent \underline{Cinematica del puro rotolamento}:
\begin{itemize}
    \item le coordinate del disco $(x_c,\;y_c,\;\theta) \in \mathbb{R}^2 \times \mathbb{R}$, con $\theta$ misurato in senso orario;
    \item $y_A=0$ (appoggio);
    \item $\overline{v}_{A_{disco}}(t)=\overline{v}_{A_{guida}}(t)=\overline{0}$ (puro rotolamento) $\Longrightarrow$ A è il CIR;
    \item $\overline{v}_c=\overline{\omega} \wedge \overline{AC}$ con $\overline{\omega}= - \dot{\theta}\hat{k}$ e $\overline{AC}(t)=(0,\,R,\,0) \Longrightarrow \overline{v}_c = R\dot{\theta}\hat{i}$;
    \item $\overline{x}_c(t)={x}_c(0)+R\left(\theta(t)-\theta(0)\right)\hat{i}$: vincolo olonomo! $\Longrightarrow$ il vincolo di puro rotolamento è integrabile.
\end{itemize}

\begin{oss}
    Nel vincolo di puro rotolamento la guida può in realtà essere un altro corpo rigido. 
\end{oss}

\begin{oss}
    In generale nel vincolo di contatto ci può essere \textbf{strisciamento}, ovvero  $\overline{v}_{A_1} \cdot \hat{\tau} \neq \overline{v}_{A_2} \cdot \hat{\tau}$. Quando questo non succede, ovvero $\overline{v}_{A_1} \cdot \hat{\tau} = \overline{v}_{A_2} \cdot \hat{\tau}$, allora c'è puro rotolamento.
\end{oss}

\begin{oss}
    Come vedremo, il vincolo di puro rotolamento è un vincolo ideale, il che va tipicamente di pari passo con la richiesta che il vincolo non sviluppi attrito, ma non nel caso del puro rotolamento! In assenza di attrito ci sarebbe sempre (puro) strisciamento.
\end{oss}

\subsection{Sistemi olonomi}

\begin{defi}
    Un \textbf{sistema olonomo} è un sistema di punti materiali sottoposto a $M$ vincoli \underline{bilateri}, \underline{olonomi} e \underline{fissi} che assumeremo sempre \underline{compatibili} e \underline{indipendenti}.
\end{defi}

\begin{oss}
    Le velocità e le accelerazioni dei punti sono date da
    \begin{displaymath}
        \overline{v}_i(t) = \sum_{j=1}^{g}\frac{\partial \overline{x}_i}{\partial q_j}\dot{q}_j=\left(J_X\dot{\overline{q}}\right)_i(t) \ \ \ \ \ \overline{a}_i(t) = \sum_{j,k=1}^{g}\left\{\frac{\partial^2 \overline{x}_i}{\partial q_j\partial q_k}\dot{q}_j\dot{q}_k + \frac{\partial \overline{x}_i}{\partial q_j}\ddot{q}_j\right\}
    \end{displaymath}
    con $\dot{q}$ e $\ddot{q}$ \textbf{velocità e accelerazioni generalizzate}.
\end{oss}

\begin{oss}
    Le coordinate libere o Lagrangiane sono per costruzione non soggette a vincoli.
\end{oss}

\begin{oss}
    Gli spostamenti e le velocità virtuali generalizzate $\delta\overline{q}^{\prime}$, $\overline{\nu}^{\prime}$ per un sistema olonomo sono indipendenti gli uni dagli altri, cioè $\delta\overline{q}^{\prime}$, $\overline{\nu}^{\prime} \in \mathbb{R}^g$.
\end{oss}

\begin{prop}
    Dato un sistema olonomo di $N$ punti materiali con $g$ gradi di libertà, si ha
    \begin{displaymath}
        \delta X_i^{\prime} = \sum_{j=1}^{g}\frac{\partial X_i}{\partial q_j}\delta{q}_j^{\prime}=\left(J_X\delta\overline{q}^{\prime}\right)_i \ \ \ \ \ V_i^{\prime} = \sum_{j,k=1}^{g}\frac{\partial X_i}{\partial q_j\partial q_k}\nu_j^{\prime} =\left(J_X\overline{\nu}^{\prime}\right)_i
    \end{displaymath}
    dove abbiamo indicato $\overline{X}=\left(\overline{x}_1,\ldots,\overline{x}_N\right)$ e $\overline{V}=\left(\overline{v}_1^{\prime},\ldots,\overline{v}_N^{\prime}\right)$.
\end{prop}

\begin{proof}
    Segue banalmente dalle leggi del cambiamento di coordinate $\overline{q}\leftrightarrow\overline{X}$.
\end{proof}

\chapter{Leggi della Meccanica}

Nella discussione della cinematica ci siamo posti come obiettivo la descrizione del moto di un sistema meccanico classico. Con l'introduzione delle leggi della meccanica e la trattazione della dinamica in generale, l'obiettivo diventa la previsione del moto di un sistema meccanico a partire da un sistema opportuno di dati iniziali e informazioni sul sistema.

\begin{oss}
    Mentre la cinematica non dipende dall'osservatore, la dinamica richiederà la scelta di $\mathcal{O}$.
\end{oss}

\section{Principi della Meccanica}

\subsection*{I principio della meccanica}

Esiste almeno un sistema (osservatore) di riferimento inerziale (\textbf{principio di inerzia}).

\begin{cor}
    La classe dei sistemi inerziali non è vuota perchè contiene per lo meno quello del I principio e tutti quelli in moto rettilineo uniforme rispetto ad esso.
\end{cor}

\begin{oss}
    A volte si caratterizza la proprietà di essere inerziale di un osservatore $\mathcal{O}$ dicendo che ogni sistema isolato inizialmente in quiete rispetto ad $\mathcal{O}$ rimane in tale stato. La proprietà di essere isolato è tuttavia problematica perchè richiederebbe che il sistema sia infinitamente lontano da ogni altro.
\end{oss}

\begin{oss}
    Il sistema di riferimento dato dalla crosta terrestre può essere considerato inerziale per alcune applicazioni, ma in effetti è non inerziale.
\end{oss}


\subsection*{II principio della meccanica}

Rispetto ad un osservatore inerziale $\mathcal{O}$, $\exists$ \textbf{massa inerziale} $m>0$ e \textbf{forza} o \textbf{risultante} delle forze $\overline{F} \in \mathbb{R}^{3}$ t.c. l'accelerazione del punto materiale $P$ soddisfa la \textbf{legge di Newton}
\begin{displaymath}
\boxed{
    m\overline{a}=\overline{F}
    }
\end{displaymath}

\begin{oss}
    La forza è una grandezza vettoriale e più precisamente un \textbf{vettore applicato}, cioè una coppia $\left(\overline{F}, \overline{x}_P\right) \in \mathbb{R}^{3} \times \mathbb{R}^{3}$ dove $\overline{x}_P$ indica il punto di applicazione della forza.
\end{oss}

\begin{oss}
     La legge di Newton è covariante, cioè è invariante per cambiamento di riferimento inerziale.
\end{oss}

\begin{oss}
    In un sistema di riferimento non inerziale la forza della legge di Newton segue dal II principio e dal teorema di Coriolis:
    \begin{displaymath}
        m\overline{a}^{\prime}=\overline{F}-\underbrace{m\left\{\dot{\overline{\omega}} \wedge \overline{x}^{\prime}+\overline{\omega} \wedge\left(\overline{\omega} \wedge \overline{x}^{\prime}\right)+2\,\overline{\omega} \wedge \dot{\overline{x}'}\right\}}_{\text{forze fittizie o apparenti}}
    \end{displaymath}
\end{oss}

\noindent La legge di Newton fornisce un sistema di equazioni differenziali del secondo ordine in forma normale, o più precisamente, assegnati i dati iniziali $\overline{x}(0)=\overline{x}_0$ e $\dot{\overline{x}}(0)=v_0$ un problema di Cauchy del secondo ordine in $\mathbb{R}^d$ che si può trasformare in un problema equivalente del primo ordine:
\begin{displaymath}
    \begin{cases}
    \ddot{\overline{x}}=\frac{1}{m}\overline{F}\left(\overline{x}; \dot{\overline{x}}; t\right) \\
    \dot{\overline{x}}(0)=\overline{v}_0 \\
    \overline{x}(0)=\overline{x}_0
    \end{cases} \Longrightarrow \quad
    \begin{cases}
    \dot{\overline{v}}=\frac{1}{m}\overline{F}\left(\overline{x}; \overline{v}; t\right) \\
    \dot{\overline{x}}=\overline{v} \\
    \overline{v}(0)=\overline{v}_0 \\
    \overline{x}(0)=\overline{x}_0
    \end{cases}
\end{displaymath}


\subsection*{III principio della meccanica}

L'iterazione fra due punti materiali è data da una coppia di forze $\left(\overline{F}_1, P_1\right)$ e $\left(\overline{F}_2, P_2\right)$ uguali e contrarie e dirette lungo la congiungente:
\begin{displaymath}
\boxed{
    \overline{F}_{1}=-\overline{F}_{2}=f \frac{\overline{P_{1}P_{2}}}{\left|\overline{P_1 P_2}\right|}
    }
\end{displaymath}

\begin{defi}
    Una \textbf{coppia di forze} sono due forze uguali e contrarie:  $\left(\overline{f}, P_1\right)$ e $\left(-\overline{f}, P_2\right)$.
\end{defi}

\section{Forze}

\begin{defi}
    Una forza si dice:
    \begin{itemize}
        \item \textbf{interna} se dovuta all'interazione con altri parti del sistema $\iff$ III principio;
        \item \textbf{esterna} altrimenti.
    \end{itemize}
\end{defi}

\noindent Tipi di forze $\overline{F}(\overline{x}, \dot{\overline{x}}, t)$:
\begin{itemize}
    \item \textbf{costante} se $\overline{F}=\overline{F}_{0} \in \mathbb{R}^{d}$;
    \item \textbf{posizionale} se $\overline{F}=\overline{F}(\overline{x})$;
    \item \textbf{centrale} se è posizionale ed $\exists \; O$ (centro) $\in \mathcal{E}_d$  t.c. $\overline{F}(\overline{x})=f\left(\left|\overline{x}-\overline{x}_{0}\right|\right) \cdot \dfrac{\overline{x}-\overline{x}_{0}}{\left|\overline{x}-\overline{x}_{0}\right|}$.
\end{itemize}

\begin{defi}
    Data una forza posizionale $\overline{F}(\overline{x})$ e una curva $\Gamma$ parametrizzata da $\overline{\gamma}:\left[s_{0}, s_{1}\right] \to \mathbb{R}^{d}$, il \textbf{lavoro} compiuto da $\overline{F}$ lungo $\Gamma$ è
    \begin{displaymath}
    \boxed{
        L=\int_{s_{0}}^{s_{1}} \overline{F}(\overline{\gamma}(s)) \cdot \overline{\gamma}^{\prime}(s)\,ds
        }
    \end{displaymath}
    Se $s$ ascissa curvilinea allora $\overline{\gamma}^{\prime}=\hat{\tau}$.
\end{defi}

\begin{oss}
     Il lavoro infinitesimo di $\overline{F}$ nello spostamento $d\overline{x}$ è
     \begin{displaymath}
         d L=F_{1} d x_{1}+F_{2} d x_{2}+F_{3} d x_{3}
     \end{displaymath}
      cioè è dato da 1-forma differenziale.
\end{oss}

\begin{oss}
    Se la forma differenziale del lavoro in un dominio semplicemente connesso $A$ è esatta e $\overline{F} \in C^1(A)$, allora
    \begin{displaymath}
         \frac{\partial F_{i}}{\partial x_{j}}=\frac{\partial F_{j}}{\partial x_{i}} \quad \forall \overline{x} \in A, \quad \forall i, j \in \{1, 2, 3\}
    \end{displaymath}
\end{oss}

\begin{defi*}
\everymath{\displaystyle}
    Una 1-forma differenziale $\omega=\sum_{j=1}^{3} f_{j}(\overline{x}) dx_{j}$ in $A$ aperto di $\mathbb{R}^{3}$ è:
    \begin{itemize}
    \everymath{\displaystyle}
        \item \textbf{esatta} se $\exists \; g: A \to \mathbb{R}^{3}, \; g \in C^{1}(A)$ t.c. $\omega=dg=\sum_{j=1}^{3} \frac{\partial g}{\partial x_{j}} dx_{j}$;
        \item \textbf{chiusa} se $d\omega=0$ dove $d\omega=\sum_{j=1}^{3} df_j \wedge dx_{j}=\sum_{i, j}^{3}\frac{\partial f_j}{\partial x_i}dx_i \wedge dx_j$
    \end{itemize}
\end{defi*}

\begin{teo*}
    Ogni forma $\omega$ esatta è anche chiusa, ovvero $d^{2} \omega=0$.
\end{teo*}

\begin{teo*}[Lemma di Poincaré]
    In un aperto semplicemente connesso $A \subset \mathbb{R}^{n}$, ogni 1-forma differenziale chiusa è anche esatta.
\end{teo*}

\begin{teo}
    Condizione necessaria affinché $dL$ sia una forma differenziale esatta è che $\overline{F}$ sia posizionale.
\end{teo}

\begin{defi}
    $\overline{F}$ posizionale è \textbf{conservativa} in $A \subset \mathbb{R}^{3}$ se $\exists$ \textbf{potenziale} $U \in C^{2}(A)$ t.c.
    \begin{displaymath}
        \boxed{\overline{F}(\overline{x})=-\nabla U(\overline{x})} \iff dL \; \text{forma esatta}
    \end{displaymath}
\end{defi}

\begin{teo}
    In $A$ aperto semplicemente connesso di $\mathbb{R}^{3} \; \overline{F}$ posizionale è conservativa $\iff \nabla \wedge \overline{F}=\overline{0} \quad \forall \overline{x} \in A$. 
\end{teo}

\begin{proof}

    \noindent 
    \begin{itemize}
        \item ($\implies$) $\overline{F}$ posizionale conservativa $\implies F_j=-\frac{\partial U}{\partial x_j}\implies(\nabla\wedge\overline{F})_i=\sum_{j,k=1}^3\varepsilon_{ijk}\frac{\partial}{\partial x_j}\left(-\frac{\partial U}{\partial x_k}\right)=0$ perché $U\in C^2(A)$ e quindi $\frac{\partial^2U}{\partial x_j \partial x_k}=\frac{\partial^2U}{\partial x_k \partial x_j}$
        \item ($\impliedby$) Se $\nabla\wedge\overline{F}=\overline{0}$ allora $dL$ è una forma chiusa (e anche esatta per Poincaré) in $A$, infatti
        \[
        \begin{split}
            d(dL)&=\sum_{j=1}^3dF_j\wedge dx_j=\sum_{j,k=1}^3\frac{\partial F_j}{\partial x_k}dx_k\wedge dx_j=\sum_{j<k}^3\left(\frac{\partial F_j}{\partial x_k}-\frac{\partial F_k}{\partial x_j}\right)dx_k\wedge dx_i \\
            &=\sum_{j<k}^3\varepsilon_{ijk}(\nabla\wedge\overline{F})_i\,dx_k\wedge dx_i=0
        \end{split}
        \]
    \end{itemize}
\end{proof}

\begin{oss}
    $(\nabla \wedge \overline{F})_{i}=\sum_{j, k=1}^{3} \varepsilon_{i j k} \frac{\partial}{\partial x_{j}} F_{k}$
\end{oss}

\begin{prop}
\everymath{\displaystyle}
    Se $\overline{F}$ è conservativa $\Longrightarrow \int_{\Gamma_0} dL=0 \quad \forall \; \Gamma_0$ curva chiusa.
\end{prop}

\begin{proof}
    Fissiamo $\overline{\gamma}: [0, L] \to \Gamma$ t.c. $\overline{\gamma}(0)=\overline{\gamma}(L)$ allora
    \begin{displaymath}
        \int_0^L\overline{F}(\overline{\gamma}(s))\cdot\hat{\tau}(s)\,ds=\int_0^L\nabla U(\overline{\gamma}(s))\cdot\overline{\gamma}'(s)\,ds=\int_0^L\frac{d}{ds}U(\overline{\gamma}(s))\,ds=U(\overline{\gamma}(L))-U(\overline{\gamma}(0))=0
    \end{displaymath}
\end{proof}

\begin{oss}
    Se il lavoro lungo una qualunque curva chiusa è nullo vuol dire che il lavoro compiuto fra due punti $\overline{x}_1$ e $\overline{x}_2 \in A$ non dipende dalla curva lungo cui è calcolato.
\end{oss}

\noindent Di seguito esempi di forze e relativi potenziali.
\begin{itemize}
    \item \textbf{Forza peso}: $\forall$ punto materiale di massa $m$ sulla Terra è una forza costante $\overline{F}_{g}=m \overline{g}$ con $\overline{g}=-g \hat{j}$.

    \noindent $U_g(\overline{x})=mgy$.
    \item \textbf{Forza elastica}: molla di lunghezza a riposo nulla e costante elastica $k>0$, genera una forza posizionale $\overline{F}_{el}(\overline{x})=-k\overline{x}$.

    \noindent $U_{el}(\overline{x})=\frac{1}{2}k\left|\overline{x}\right|^2$. Se la lunghezza a riposo è $l>0$: $U_{el}(\overline{x})=\frac{1}{2}k\left(\left|\overline{x}\right|-l\right)^2$.
    \item \textbf{Forza gravitazionale}: date due masse (gravitazionali) $m_1$ e $m_2$, esse si attraggono con una forza $\overline{F}_{G}=-G \frac{m_{1} m_{2}}{r^{2}} \hat{r}$. Se poi una delle due ha massa molto più grande dell'altra cioè $m_{1}=m<<M=m_{2}$ allora possiamo assumere $m_2$ fissa e in quel caso la forza è centrale con centro $\overline{x}_{2}$: $\overline{F}_G(\overline{x})=-G \frac{mM}{r^{2}} \hat{r}$ con $\overline{r}=\overline{x}-\overline{x}_2$.

    \noindent $U_G(\overline{r})=-G \frac{mM}{r}$.
    \item \textbf{Forza di attrito} (moto in un mezzo): $\overline{F}=
    \begin{cases}
        -k\dot{\overline{x}} \; \; \quad \text{resistenza viscosa} \\
        -k|\dot{\overline{x}}| \quad \text{resistenza idraulica}
    \end{cases}$
    è posizionale e si oppone al moto.
    \item \textbf{Forza di attrito} (contatto): legge di Coulomb-Morin$: \dfrac{\left|\phi_{\tau}\right|}{\left|\phi_{n}\right|}
    \begin{cases}
        \leq \mu_{s} \quad \textbf{attrito statico} \\
        = \mu_{d} \quad \textbf{attrito dinamico}
    \end{cases}$
\end{itemize}

\begin{oss}
    In presenza di vincoli le forze fisiche che abbiamo introdotto non sono sufficienti a descrivere il moto del sistema: il vincolo è l'idealizzazione di un numero molto grande di interazioni microscopiche, a ciascuno delle quali corrisponde una forza.
\end{oss}

\begin{defi}
    Una forza si dice:
    \begin{itemize}
        \item \textbf{attiva} se è nota a priori la sua dipendenza da $\overline{x}, \dot{\overline{x}}$ e $t$ cioè $\overline{F}=\overline{F}(\overline{x}, \dot{\overline{x}}, t)$;
        \item \textbf{reazione vincolare} se così non è e la forza è dovuta alla presenza di un vincolo.
    \end{itemize}
\end{defi}

\begin{oss}
    In presenza di vincoli la legge di Newton continua a valere, ma dobbiamo introdurre le reazioni vincolari (una per vincolo) per imporre la conservazione del vincolo:
    \begin{displaymath}
    \boxed{
         m\overline{a}=\overline{F}^{(att)}+\overline{\Phi}
         }
    \end{displaymath}
\end{oss}

\chapter{Statica}

\section{Equazioni cardinali della statica}

Nella statica ci proponiamo di studiare le configurazioni di equilibrio di un sistema di punti materiali, ovvero quelle configurazioni che sono preservate dall'evoluzione temporale.

\noindent I problemi di statica sono di tre tipi:
\begin{itemize}
    \item \underline{problema diretto}: assegnate le forze agenti su un sistema, determinare le configurazioni di equilibrio;
    \item \underline{problema inverso}: assegnata la configurazione di equilibrio, determinare le forze che la realizzano;
    \item \underline{stabilità}: assegnata la configurazione di equilibrio, studiare l'effetto di (piccole) perturbazioni su di essa.
\end{itemize}

\begin{defi}
    Le equazioni del moto $\overline{F}=m \overline{a}$ per un punto materiale $P$ ammettono la soluzione di \textbf{quiete} nella configurazione $\overline{x}_{0} \in \mathbb{R}^{d}$ se $\overline{x}(t)=\overline{x}_{0}, \forall t \in \mathbb{R}$ è soluzione delle equazioni del moto.
\end{defi}

\begin{defi}
    Un sistema meccanico composto da $N \in \mathbb{N} \cup \{+\infty\}$ punti materiali $\{P_j\}_{j \in \{1,\ldots,N\}}$ sottoposto alle forze $\left\{\overline{F}_{j}\left(\overline{x}_{1}, \ldots, \overline{x}_{N}; \overline{v}_{1}, \ldots, \overline{v}_{N}; t\right)\right\}_{j \in \{1,\ldots,N\}}$ ha una \textbf{configurazione di equilibrio} $\left\{\overline{x}_j^{(0)}\right\}_{j \in\{1, \ldots, N\}}$ se
    \begin{displaymath}
    \boxed{
        \overline{F}_{j}\left(\overline{x}_{1}^{(0)}, \ldots, \overline{x}_{N}^{(0)}; \overline{0}, \ldots, \overline{0}; t\right)=\overline{0} \quad \forall t \in \mathbb{R}, \quad \forall j \in \{1,\ldots,N\}
        }
    \end{displaymath}
\end{defi}

\begin{oss}
    Il fatto che una configurazione di equilibrio sia di equilibrio è condizione \underline{necessaria} affinché si abbia la soluzione di quiete nella medesima configurazione. D'altra parte sotto ipotesi di Lipschitzianità della forza (che assumeremo sempre), la soluzione delle equazioni del moto è unica, quindi la condizione è anche \underline{sufficiente}.
\end{oss}

\begin{oss}
    Assegnate le forze $\left\{\overline{F}_j\right\}_{j \in\{1, \ldots, N\}}$ agenti sui punti materiali $\left\{P_j\right\}_{j \in\{1, \ldots, N\}}$, esse si possono sempre decomporre come:
   \begin{displaymath}
   \overline{F}_{j}=\overline{F}_{j}^{(ext)}+\overline{F}_{j}^{(int)}=\overline{F}_{j}^{(att)}+\overline{\Phi}_{j}
   \end{displaymath}
\end{oss}

\begin{defi}
    Data una forza $\left(\overline{f}, \overline{x}_{P}\right)$ applicata in $P$, il suo \textbf{momento} rispetto al \textbf{polo} $O$ è dato da
    \begin{displaymath}
        \boxed{
        \overline{m}=\left(\overline{x}_{P}-\overline{x}_{O}\right) \wedge \overline{f}
        }
    \end{displaymath}
\end{defi}

\begin{oss}
    $|\overline{m}|=|\overline{OP}||\overline{f}||\sin{\theta}|$ e $\overline{m}=m \cdot \hat{k}$.
\end{oss}

\begin{defi}
    Un \textbf{sistema di forze} $\mathcal{S}=\left\{\left(\overline{f}_{1}, \overline{x}_{1}\right), \ldots,\left(\overline{f}_{n}, \overline{x}_{n}\right)\right\}$ è una collezione di vettori applicati.
\end{defi}

\begin{defi}
    La \textbf{risultante} $\overline{R}$ e il \textbf{momento risultante} $M_O$ rispetto al polo $O$ di un sistema di forze $\mathcal{S}$ sono date da:
    \begin{displaymath}
    \boxed{
        \overline{R}=\sum_{j=1}^{n} \overline{f}_{j}, \ \ \ \ \ \overline{M}_{O}=\sum_{j=1}^{n}\left(\overline{x}_{j}-\overline{x}_{0}\right) \wedge \overline{f}_{j}
        }
    \end{displaymath}
\end{defi}

\begin{oss}
    Il momento risultante di una coppia di forze \underline{non dipende} dal polo scelto.
\end{oss}

\begin{teo}
    Condizione \underline{necessaria} affinché un sistema meccanico sia in una configurazione di equilibrio è che il sistema di forze \underline{esterne} agenti su di esso soddisfino le \textbf{equazioni cardinali della statica}:
    \begin{displaymath}
    \boxed{
        \overline{R}^{(ext)}=\overline{0}, \ \ \ \ \ \overline{M}_{O}^{(ext)}=\overline{0} \quad \forall \; O \in \mathcal{E}_{3}
        }
    \end{displaymath}
\end{teo}

\begin{lemma}
    Per ogni sistema meccanico il sistema di forze \underline{interne} è tale che:
    \begin{displaymath}
        \overline{R}^{(int)}=\overline{0}, \quad \overline{M}_{O}^{(int)}=\overline{0} \quad \forall \; O \in \mathcal{E}_{3}
    \end{displaymath}
\end{lemma}

\begin{proof}[Dimostrazione (Lemma)]
    Siano $\left\{\overline{F}_{ij}\right\}_{i,j\in\{1,\ldots,N\}}$ le forze interne del sistema dove $\overline{F}_{ij}$ indica al forza esercitata dal punto $j$ sul punto materiale $i$, allora
    \begin{displaymath}
        \overline{R}^{(int)}=\sum_{i\neq j}\overline{F}_{ij}=\sum_{1<j}(\overline{F}_{ij}+\overline{F}_{ji})=\sum_{1<j}(\overline{F}_{ij}-\overline{F}_{ij})=\overline{0}
    \end{displaymath}
    \[
    \begin{split}
        \overline{M}_O^{(int)}&=\sum_{i\neq j}(\overline{x}_i-\overline{x}_O)\wedge\overline{F}_{ij}=\sum_{1<j}\left\{(\overline{x}_i-\overline{x}_O)\wedge\overline{F}_{ij}+(\overline{x}_j-\overline{x}_O)\wedge\overline{F}_{ji}\right\} \\
        &=\sum_{1<j}\left\{(\overline{x}_i-\overline{x}_O)\wedge\overline{F}_{ij}-(\overline{x}_j-\overline{x}_O)\wedge\overline{F}_{ij}\right\}=\sum_{1<j}\left\{(\overline{x}_i-\cancel{\overline{x}_O}-\overline{x}_j+\cancel{\overline{x}_O})\wedge\overline{F}_{ij}\right\} \\
        &=\sum_{1<j}(\overline{x}_i-\overline{x}_j)\wedge\overline{F}_{ij}=\overline{0}
    \end{split}
    \]
\end{proof}

\begin{proof}[Dimostrazione (Teorema)]
    Una configurazione $\left\{\overline{x}_j^{(0)}\right\}_{j\in\{1,\ldots,N\}}$ è di equilibrio $\iff$
    \begin{displaymath}
        \overline{F}_{j}\left(\overline{x}_{1}^{(0)}, \ldots, \overline{x}_{N}^{(0)}; \overline{0}, \ldots, \overline{0}; t\right)=\overline{0} \;\forall t \in \mathbb{R}, \;\forall j \in \{1,\ldots,N\}
    \end{displaymath}
    ma allora
    \begin{displaymath}
        \overline{R}=\sum_{j=1}^N\overline{F}_j=\overline{0}=\overline{R}^{(ext)}+\overline{R}^{(int)}=\overline{R}^{(ext)}
    \end{displaymath}
    Nello stesso modo si ottiene il risultato per il momento.
\end{proof}

\begin{oss}
    Le equazioni cardinali della statica sono condizioni \underline{necessarie}, ma in generale non sufficienti all'equilibrio di un sistema meccanico.
\end{oss}

\begin{oss}
    Le equazioni cardinali della statica per un sistema meccanico composto da vari sottosistemi si applicano a ciascun sottosistema separatamente oltre che al sistema complessivo. Occorre però prestare attenzione a individuare correttamente le forze esterne agenti sul sottosistema perché può accadere che forze interne per il sistema complessivo diventino esterne per il sottosistema.
\end{oss}

\begin{prop}[Trasporto del momento]
    Dato un sistema di forze $\mathcal{S}$ e due poli $O$ e $O^{\prime} \in \mathcal{E}_{d}$ si ha: \begin{displaymath}
    \boxed{
    \overline{M}_{O^{\prime}}=\overline{M}_{O}+\overline{OO^{\prime}} \wedge \overline{R}
    }
    \end{displaymath}
\end{prop}

\begin{proof}
    \[
    \begin{split}
        \overline{M}_{O'}&=\sum_{j=1}^N(\overline{x}_j-\overline{x}_{O'})\wedge\overline{f}_j=\sum_{j=1}^N(\overline{x}_j-\overline{x}_O+\overline{x}_O-\overline{x}_{O'})\wedge\overline{f}_j \\
        &=\overline{M}_O+(\overline{x}_O-\overline{x}_{O'})\wedge\sum_{j=1}^N\overline{f}_j=\overline{M}_O+\overline{O'O}\wedge\overline{R}
    \end{split}
    \]
\end{proof}

\begin{oss}
    Sfruttando la legge del trasporto del momento, si possono rimpiazzare le condizioni sulla risultante con condizioni sul momento.
\end{oss}

\section{Sistemi di forze}

\begin{defi}
    Due sistemi di forze $\mathcal{S}$ e $\mathcal{S}^{\prime}$ si dicono \textbf{equivalenti} $\mathcal{S} \sim \mathcal{S}^{\prime}$ se $\overline{R}=\overline{R}^{\prime}$ e $\overline{M}_{O}=\overline{M}_{O}^{\prime}\; \forall O \in \mathcal{E}_{3}$. 
\end{defi}

\begin{oss}
    L'equivalenza di sistemi di forze è una relazione di equivalenza nel senso matematico del termine, cioè soddisfa le seguenti proprietà:
    \begin{itemize}
        \item riflessiva: $\mathcal{S} \sim \mathcal{S}$;
        \item simmetrica: $\mathcal{S} \sim \mathcal{S}^{\prime} \iff \mathcal{S}^{\prime} \sim \mathcal{S}$;
        \item transitiva: $\mathcal{S} \sim \mathcal{S}^{\prime}$ e $\mathcal{S}^{\prime} \sim \mathcal{S}^{\prime \prime} \Longrightarrow \mathcal{S} \sim \mathcal{S}^{\prime \prime}$.
    \end{itemize}
\end{oss}

\begin{prop}
    Ogni sistema di forze è equivalente al più ad una forza più una coppia di forze.
\end{prop}

\begin{proof}
    Dato $\mathcal{S}$ di risultante $\overline{R}$ e momento risultante $\overline{M}_O$, costruiamo il sistema di forze $\mathcal{S}'=\left\{(\overline{R},\overline{0}), (\overline{f},\overline{0}),(-\overline{f},\overline{x}_P)\right\}$ dove $\overline{f}$ e $P$ sono scelti in modo tale che $-(\overline{x}_P-\overline{x}_O)\wedge\overline{f}=\overline{M}_O$.

    \noindent Rispetto al polo $O$ i due sistemi sono ovviamente equivalenti, ma la legge del trasporto del momento garantisce l'equivalenza rispetto ad ogni altro polo.
\end{proof}

\begin{defi}
    Dato un sistema di forze $\mathcal{S}$ di risultante $\overline{R}$ e momento risultante $\overline{M}_O$ rispetto al polo $O \in \mathcal{E}_3$, l'\textbf{invariante scalare} è
    \begin{displaymath}
        \boxed{
        I=\overline{R} \cdot \overline{M}_{O}
        }
    \end{displaymath}
\end{defi}

\begin{oss}
    L'invariante scalare non dipende dal polo scelto.
\end{oss}

\begin{prop}
    Dato un sistema di forze $\mathcal{S}$ con $\overline{R} \neq \overline{0}, \exists$ una retta $h$ \textbf{asse centrale} tale che $\overline{M}_{P} \parallel \overline{R} \; \forall P \in h$. L'asse centrale ha equazione
    \begin{displaymath}
        h: \overline{x}=\lambda \overline{R}+\frac{\overline{R} \wedge \overline{M}_{A}}{|\overline{R}|^{2}}+\overline{x}_{A}
    \end{displaymath}
    con $A \in \mathcal{E}_{3}$ un punto generico. Nel caso in cui $I=0$ l'asse centrale prende il nome di \textbf{asse di applicazione della risultante} e si ha $\overline{M}_{P}=\overline{0}, \forall P \in h$.
\end{prop}

\begin{proof}
    Sia $\overline{R}\neq\overline{0}$ e $P\in h$. Verifichiamo che $\overline{M}_P\parallel\overline{R}$: si ha
    \begin{displaymath}
        \overline{x}_P-\overline{x}_A=\lambda\overline{R}+\frac{\overline{R}\wedge\overline{M}_A}{|\overline{R}|^2}=\overline{AP}
    \end{displaymath}
    ma la legge di trasporto del momento implica che
    \[
    \begin{split}
        \overline{M}_P&=\overline{M}_A+\overline{PA}\wedge\overline{R}=\overline{M}_A+\left(-\lambda\overline{R}-\frac{\overline{R}\wedge\overline{M}_A}{|\overline{R}|^2}\right)\wedge\overline{R} \\
        &=\overline{M}_A-\frac{(\overline{R}\wedge\overline{M}_A)\wedge\overline{R}}{|\overline{R}|^2}=\cancel{\overline{M}_A}-\cancel{\frac{\overline{R}\cdot\overline{R}}{|\overline{R}|^2}}\cancel{\overline{M}_A}+\frac{\overline{M}_A\cdot\overline{R}}{|\overline{R}|^2}\overline{R}=\frac{I}{|\overline{R}|^2}\overline{R}\parallel\overline{R}
    \end{split}
    \]
    Osserviamo anche che, nel caso $I=0$, il momento si annulla automaticamente.
\end{proof}

\begin{defi}
\everymath{\displaystyle}
    Dato un sistema di punti materiali, il \textbf{centro di massa} o \textbf{baricentro} è il punto di coordinate
    \begin{displaymath}
    \boxed{
        \overline{x}_{CM}=\overline{x}_{G}=\frac{1}{M}\sum_{j=1}^Nm_j\overline{x}_j
        }
    \end{displaymath}
    dove $M=\sum_{j=1}^Nm_j$ è la \textbf{massa totale}.
\end{defi}

\begin{oss}
    Dato un qualunque sistema di punti materiali sottoposto alla forza peso, l'azione su tutti i punti di tale forza è equivalente ad una forza $M\overline{g}$ applicata in $\overline{x}_G$.
\end{oss}

\begin{oss}
    Per un sistema continuo si può definire una funzione $\rho: \mathbb{R}^{3} \to \mathbb{R}^{+}$ detta \textbf{distribuzione di massa} tale che la massa contenuta nella regione $S \subset \mathbb{R}^3$ sia
    $m_{S}=\int_{S} \rho(\overline{x})\,d\overline{x}$ per cui $M=\int_{\mathbb{R}^3}\rho(\overline{x})\,d\overline{x}$. In questo caso
    \begin{displaymath}
        \overline{x}_{CM}=\frac{1}{M} \int_{\mathbb{R}^{3}} \overline{x} \rho(\overline{x})\,d\bar{x}
    \end{displaymath}
\end{oss}

\begin{teo}
    Dato un sistema di forze $\mathcal{S}$ di risultante $\overline{R}$ e momento $\overline{M}_O$ rispetto al polo $O \in \mathcal{E}_3$ e invariante scalare $I=\overline{R} \cdot \overline{M}_O$, allora
    \begin{itemize}
        \item $\overline{R}=\overline{0}, \; \overline{M}_{O}=\overline{0} \iff \mathcal{S} \sim \emptyset$;
        \item $\overline{R}=\overline{0}, \; \overline{M}_{O} \neq \overline{0} \iff \mathcal{S} \sim \left\{(\overline{f}, O), (-\overline{f}, P)\right\}$ (coppia di forze) con $\overline{f}, P$ t.c. $-\overline{OP} \wedge \overline{f}=\overline{M}_O$;
        \item $\overline{R} \neq \overline{0}, \; I=0 \iff \mathcal{S} \sim\left\{(\overline{R}, P)\right\}$ con $P \in h$ retta di applicazione della risultante;
        \item $\overline{R} \neq \overline{0}, \; I \neq \overline{0} \iff \mathcal{S} \sim \left\{(\overline{R}, O), (\overline{f}, O), (-\overline{f}, P)\right\}$ (forza + coppia di forze).
    \end{itemize}
\end{teo}

\begin{lemma}
    Dati due sistemi di forze $\mathcal{S}$ e $\mathcal{S}^{\prime}$ con $\overline{R}=\overline{R}^{\prime}$, si ha $\overline{M}_{O}=\overline{M}_{O}^{\prime}$ per un centro $O \in \mathcal{E}_{3} \iff \mathcal{S} \sim \mathcal{S}^{\prime}$.
\end{lemma}

\begin{proof}[Dimostrazione (Lemma)]
    Il risultato è una conseguenza diretta della legge di trasporto del momento: $\forall P \;\overline{M}_P=\overline{M}_O+\overline{PO}\wedge\overline{R}=\overline{M}_O'+\overline{PO}\wedge\overline{R}'=\overline{M}_O'$
\end{proof}

\begin{proof}[Dimostrazione (Teorema)]
    È sufficiente considerare solo l'implicazione ($\implies$) perché l'altra si ottiene dall'esaurimento dei casi possibili.

    \noindent Inoltre il 4° punto è già stato dimostrato nella proposizione 3.3, mentre il 1° e il 2° sono banali (grazie anche al lemma).

    \noindent Resta quindi da considerare il 3° punto ma per quanto dimostrato nella proposizione 3.3, $\overline{M}_P=\overline{0} \;\forall P \in h$. Quindi $\overline{M}_P'=\overline{0}=\overline{M}_P$ e il lemma ci dà il risultato.
\end{proof}

\section{Vincoli ideali}

\begin{defi}
    Dato un punto materiale che si muove a  velocità $\overline{v}$, la \textbf{potenza istantanea} generata dalle forze $\overline{F}\left(\overline{x}; \overline{v}; t\right)$ è
    \begin{displaymath}
    \boxed{
        \Pi=\overline{F}\left(\overline{x}; \overline{v}; t\right) \cdot \overline{v}
        }
    \end{displaymath}
\end{defi}

\begin{oss}
    Se $\overline{\gamma}(t): [0, T] \to \mathbb{R}^{3}$ di un punto materiale, allora
    \begin{displaymath}
        L=\int_0^{T} \dot{\overline{\gamma}}(t) \cdot \overline{F}\left(\overline{\gamma}(t); \dot{\overline{\gamma}}(t); t\right)\,dt
    \end{displaymath}
\end{oss}

\begin{oss}
    Se $L(t)$ è il lavoro prodotto da $\overline{F}$ lungo il moto $\overline{\gamma}(t)$ di un punto materiale nell'intervallo $[0, t]$ allora $\frac{dL}{dt}= \Pi$.
\end{oss}

\begin{defi}
    Dato un sistema di punti $\left\{P_{j}\right\}_{j \in \{ 1, \ldots, N\}}$ sottoposti alle forze $\left\{\overline{F}_{j}\right\}_{j \in \{1, \ldots, N\}}$, il \textbf{lavoro virtuale} $\delta L^{\prime}$ e la \textbf{potenza virtuale} $\Pi^{\prime}$ sono le quantità
    \begin{gather*}
        \delta L^{\prime}=\sum_{j=1}^{N} \overline{F}_{j} \cdot \delta \overline{x}_{j}^{\prime} \quad \delta \overline{x}_{j}^{\prime} \in S_{j}^{\prime} \quad \forall j \in\{1, \ldots, N\} \\
        \Pi^{\prime}=\sum_{j=1}^{N} \overline{F}_{j} \cdot \overline{v}_{j}^{\prime} \quad \overline{v}_{j}^{\prime} \in V_{j}^{\prime} \quad \forall j \in\{1, \ldots, N\}
    \end{gather*}
\end{defi}

\begin{defi}
    Diciamo che i vincoli a cui è sottoposto un sistema di punti materiali sono \textbf{ideali} se generano \underline{tutte e sole} le reazioni vincolari che producono lavoro virtuale o potenza virtuale non negative:
    \begin{displaymath}
        \sum_{j=1}^{N} \overline{\Phi}_{j} \cdot \delta \overline{x}_{j}^{\prime} \geq 0 \quad \forall \delta \overline{x}_{j}^{\prime} \in S_{j}^{\prime}, \forall j \quad \text{o analogamente} \quad \sum_{j=1}^{N} \overline{\Phi}_{j} \cdot \overline{v}_{j}^{\prime} \geq 0 \quad \forall \overline{v}_{j}^{\prime} \in V_{j}^{\prime}, \forall j
    \end{displaymath}
\end{defi}

\begin{oss}
    Un vicolo bilatero è ideale (e in quel caso si dice \textbf{perfetto}) se e solo se
    \begin{displaymath}
        \sum_{j=1}^{N} \overline{\Phi}_{j} \cdot \delta \overline{x}_{j}^{\prime} = 0 \quad \forall \delta \overline{x}_{j}^{\prime} \in S_{j}^{\prime}, \forall j \quad \text{o} \quad \sum_{j=1}^{N} \overline{\Phi}_{j} \cdot \overline{v}_{j}^{\prime} = 0 \quad \forall \overline{v}_{j}^{\prime} \in V_{j}^{\prime}, \forall j
    \end{displaymath}
\end{oss}

\subsection{Tipi di vincoli ideali}

\begin{defi}
    In generale un vicolo si dice \textbf{liscio} se non produce attrito di contatto nel qual caso la reazione vincolare è sempre $\perp$ alla superficie di contatto.
\end{defi}

\begin{oss}
    In molti casi un vincolo è ideale quando è \underline{liscio}, ma esistono vincoli ideali (puro rotolamento) dove il ruolo dell'attrito è fondamentale.
\end{oss}

\begin{itemize}
    \item \textbf{Guida perfetta}: un punto materiale è vincolato a muoversi lungo una guida $\Gamma$ in assenza di attrito, quindi la reazione vincolare è sempre diretta lungo la normale alla curva $\overline{\Phi}=\overline{\Phi}_{n} \hat{n}(s_P)$.
    \item \textbf{Incastro perfetto} (bilaterale): $\left(\overline{\Phi}, C\right); \overline{M}_{C} \longrightarrow$ momento necessario a cancellare i momenti delle forze non applicate in $C$.
    \item \textbf{Appoggio ideale} (unilatero): un punto materiale si muove senza attrito su una superficie orizzontale $\overline{\Phi}=\overline{\Phi}_{n} \hat{n}(s_P), \overline{\Phi}_n \geq 0$.
    \item \textbf{Puro rotolamento}: un disco rotola senza strisciare su una superficie orizzontale $\overline{\Phi}=\Phi_{\tau} \hat{\tau}(s_H)+\Phi_{n} \hat{n}(s_H)$
    \begin{itemize}
        \item H \underline{appoggiato} su $\hat{i}$ (unilatero): $\Phi_{\tau} \in \mathbb{R}, \Phi_{n} \in \mathbb{R}^{+}$;
        \item H \underline{vincolato} a muoversi lungo $\hat{i}$ (bilatero): $\Phi_{\tau}, \Phi_{n} \in \mathbb{R}$.
    \end{itemize}
    \item \textbf{Cerniera fissa perfetta} (bilatero): $\overline{v}_{C}^{\prime}=\overline{0} \Longrightarrow \left(\overline{\Phi}, C\right)$.
    \item \textbf{Cerniera mobile perfetta} (bilatero): $\left(\overline{\Phi}_{1}, C\right), \left(\overline{\Phi}_{2}, C\right)$ con $\overline{\Phi}_{1}=-\overline{\Phi}_{2}$.
    \item \textbf{Manicotto perfetto} (bilatero): $\overline{\Phi}=\Phi_{n} \hat{n}(s_C)$
    \begin{itemize}
        \item in 2D $\left(\overline{\Phi}, C\right); \overline{M}_{C} \parallel\hat{k}$;
        \item in 3D $\left(\overline{\Phi}, C\right)$.
    \end{itemize}
    \item \textbf{Carrello perfetto} (bilatero): $\left(\overline{\Phi}, C\right), \overline{\Phi}=\Phi_{n} \hat{n}(s_C)$.
\end{itemize}


\section{Principio dei lavori virtuali}

Le equazioni cardinale della statica (in effetti anche quelle della dinamica che vederemo) dipendono dalle forze \underline{esterne} che in generale contengono le \underline{reazioni vincolari}.
Poiché queste ultime non sono assegnate a priori, spesso è impossibile risolvere le equazioni cardinali e determinare la configurazione di equilibrio. Sarebbe molto utile poter ricavare delle equazioni che non contengono le razioni vincolari.

\begin{defi}
    Un'equazione che caratterrizza la statica o la dinamica di un sistema e non contiene reazioni vincolari è detta \textbf{equazione pura}.
\end{defi}

\noindent Quindi siamo alla ricerca di equazioni pure della statica. Vedremo due modi di ricavarle:
\begin{itemize}
    \item principio dei lavori virtuali (PLV);
    \item potenziale (per sistemi olonomi).
\end{itemize}

\begin{teo}[Principio dei lavori virtuali - PLV]
    Dato un sistema di punti materiali $\{P_j\}_{j \in\{1, \ldots, N\}}$ sottoposto a vincoli ideali/perfetti e fissi, la configurazione $\{P_j^{(0)}\}_{j \in\{1, \ldots, N\}}$ è di equilibrio se e solo se
    \begin{displaymath}
    \boxed{
        \delta L^{\prime(att)}=\sum_{j=1}^{N} \overline{F}_{j}^{(att)} \cdot \delta \overline{x}_{j}^{\prime}
        \begin{cases}
            \leq 0 \quad \text{se vincoli ideali} \\
            = 0 \quad \text{se vincoli perfetti}
        \end{cases}
        \quad \forall \delta \overline{x}_{j}^{\prime} \in S_{j}^{\prime}\left(P_1^{(0)},\ldots,P_N^{(0)}\right), \forall j
        }
    \end{displaymath}
\end{teo}

\begin{proof}

    \noindent 
    \begin{itemize}
        \item ($\implies$) $\{P_j^{(0)}\}_{j \in\{1, \ldots, N\}}$ configurazione di equilibrio $\iff$ $\overline{F}_j(\{P_j^{(0)}\})=\overline{0}, \;\forall \; j \in \{1, \ldots, N\} \implies$ 
        \begin{displaymath}
            \sum_{j=1}^N\overline{F}_j\cdot\delta\overline{x}_j'=0=\sum_{j=1}^N\overline{F}_j^{(att)}\cdot\delta\overline{x}_j'+\underbrace{\sum_{j=1}^N\overline{\Phi}_j\cdot\delta\overline{x}_j'}_{\begin{cases}
            \geq 0 \quad \text{se vincoli ideali} \\
            = 0 \quad \text{se vincoli perfetti}
        \end{cases}}=0
        \end{displaymath}
        \item ($\impliedby$) (vincoli ideali) Se
        \begin{displaymath}
            \sum_{j=1}^N\overline{F}_j^{(att)}\cdot\delta\overline{x}_j'\leq 0 \;\forall \,\delta \overline{x}_{j}^{\prime} \in S_{j}^{\prime}(P_1^{(0)},\ldots,P_N^{(0)}), \,\forall j \in \{1, \ldots, N\}
        \end{displaymath} possiamo prendere come reazioni vincolari $\overline{\Phi}_j=-\overline{F}_j^{(att)}, \forall j$, poiché tali reazioni vincolari soddisfano la condizione
        \begin{displaymath}
            \sum_{j=1}^N\overline{\Phi}_j\cdot\delta\overline{x}_j'=-\sum_{j=1}^N\overline{F}_j^{(att)}\cdot\delta\overline{x}_j'\geq 0
        \end{displaymath}
        e quindi sono fra le reazioni un vincolo ideale può generare. Quindi $\overline{F}_j=\overline{0}\;\forall j \in \{1, \ldots, N\}$ e $\{P_j^{(0)}\}_{j \in\{1, \ldots, N\}}$ condizione di equilibrio.
    \end{itemize}
\end{proof}

\begin{oss}
    Equivalentemente si può scrivere la condizione per la potenza virtuale, ovvero
    \begin{displaymath}
        \Pi^{\prime}=\sum_{j=1}^{N} \overline{F}_{j}^{(att)} \cdot \overline{v}_{j}^{\prime}
        \begin{cases}
            \leq 0 \quad \text{se vincoli ideali} \\
            = 0 \quad \text{se vincoli perfetti}
        \end{cases}
        \quad \forall \overline{v}_{j}^{\prime} \in V_{j}^{\prime}\left(P_1^{(0)},\ldots,P_N^{(0)}\right), \forall j
    \end{displaymath}
\end{oss}

\begin{oss}
    Il PLV fornisce un'equazione pura della statica per costruzione.
\end{oss}

\begin{oss}
    La condizione che i vincoli siano fissi è cruciale.
\end{oss}

\section{Statica dei sistemi olonomi}

Ricordiamo che un sistema olonomo è un sistema sottoposto a vincoli \underline{perfetti} (in effetti è sufficiente ideali) \underline{olonomi} e \underline{fissi}.
Per tale sistema abbiamo delle coordinate libere $\left(q_1, \ldots, q_g\right) \in \mathbb{R}^g$ con g= \# g.d.l. e delle mappe $\overline{X}\left(q_1, \ldots, q_g\right)$ che forniscono le coordinate originali in funzione di quelle libere.

\noindent Inoltre per gli spostamenti virtuali $\delta \overline{x}_{j}^{\prime}=\sum_{i=1}^{g} \frac{\partial \overline{x}_{j}}{\partial q_{i}} \cdot \delta q_{i}^{\prime}$ con $\delta q_i^{\prime} \in \mathbb{R}, \; \forall i \in \{1, \ldots, g\}$.


\begin{prop}
\everymath{\displaystyle}
    Il lavoro virtuale delle forze attive per un sistema olonomo è dato da
    \begin{displaymath}
        \delta L^{\prime(att)}=\sum_{k=1}^{g} Q_{k} \, \delta q_{k}^{\prime}
    \end{displaymath}
    dove $\overline{q}= \left(q_1, \ldots, q_g\right)$ sono le coordinate generalizzate e $Q_{k}=\sum_{j=1}^{N} \overline{F}_{j}^{(att)} \frac{\partial \overline{x}_{j}}{\partial q_{k}}$ sono le \textbf{componenti generalizzate delle forze attive}.
\end{prop}

\begin{proof}
    \begin{displaymath}
        \delta L^{\prime(att)}=\sum_{k=1}^{N}\overline{F}_j^{(att)}\cdot\delta\overline{x}_j'=\sum_{k=1}^{N}\overline{F}_j^{(att)}\cdot\sum_{k=1}^{g}\frac{\partial\overline{x}_j}{\partial q_k}\,\delta q_k'=\sum_{k=1}^{g} Q_{k} \, \delta q_{k}^{\prime}
    \end{displaymath}
\end{proof}

\begin{prop}
    Condizione necessaria e sufficiente affinché un sistema olonomo sia in equilibrio in $\overline{q}^{(0)}$ è che 
    \begin{displaymath}
        Q_{k}\left(\overline{q}^{(0)}\right)=0 \quad \forall k \in \left\{1, \ldots, g\right\} 
    \end{displaymath}
\end{prop}

\begin{proof}
    Per il PLV il sistema è in equilibrio se e solo se $\delta L'^{(att)}=0$ ma
    \begin{displaymath}
        0=L'^{(att)}=\sum_{k=1}^g Q_k\,(\overline{q}^{(0)})\,\delta q_k'\iff Q_k\,(\overline{q}^{(0)})=0,\; \forall k \in \left\{1, \ldots, g\right\}
    \end{displaymath}
    poiché gli spostamenti virtuali $\delta q_k$ sono indipendenti uno dall'altro e quindi per ogni $\Bar{k}$ si può scegliere $\delta q_{\Bar{k}}'=1$ e $\delta q_k'=0 \; \forall k\neq\Bar{k}$. 
\end{proof}

\begin{oss}
    Se le forze attive sono \underline{conservative} $\exists \, \widetilde{U}: \mathbb{R}^{dN} \to \mathbb{R}$ t.c. $\overline{F}_{j}^{(att)}=-\nabla_{j} \widetilde{U}\left(\overline{x}_{1}, \ldots, \overline{x}_{N}\right)$, per cui possiamo definire il potenziale in funzione delle coordinate generalizzate
    \begin{displaymath}
    \boxed{
        U\left(\overline{q}\right)=\widetilde{U}\left(\overline{x}_{1}\left(\overline{q}\right), \ldots, \overline{x}_{N}\left(\bar{q}\right)\right)
        }
    \end{displaymath}
\end{oss}

\begin{teo}
    Condizione necessaria e sufficiente affinché un sistema olonomo sottoposto a forze attive conservative sia in equilibrio in $\overline{q}^{(0)}$ è
    \begin{displaymath}
    \boxed{
        \frac{\partial U}{\partial q_{k}}\left(\overline{q}^{(0)}\right)=0, \quad \forall k \in \{1, \ldots, g\}
        }
    \end{displaymath}
\end{teo}

\begin{proof}
    Per la proposizione 3.5 il sistema è in equilibrio $\iff Q_k(\overline{q}^{(0)})=0\;\forall k$, ma
    \begin{displaymath}
        0=Q_k(\overline{q}^{(0)})=\sum_{j=1}^N\left(\overline{F}_j^{(att)}\cdot\frac{\partial\overline{x}_j}{\partial q_k}\right)(\overline{q}^{(0)})=-\sum_{j=1}^N\left(\nabla_j\widetilde{U}\cdot\frac{\partial\overline{x}_j}{\partial q_k}\right)(\overline{q}^{(0)})=-\frac{\partial U}{\partial q_k}(\overline{q}^{(0)})
    \end{displaymath}
    poiché
    \begin{displaymath}
        \frac{\partial U}{\partial q_k}=\frac{\partial}{\partial q_k}\left(\widetilde{U}(\overline{x}_1(\overline{q}),\ldots,\overline{x}_N(\overline{q}))\right)=\sum_{j=1}^N\nabla_j\widetilde{U}\cdot\frac{\partial\overline{x}_j}{\partial q_k}
    \end{displaymath}
\end{proof}

\begin{oss}
    La stazionarietà del potenziale fornisce automaticamente un'equazione pura della statica.
\end{oss}

\begin{prop}[Principio di Torricelli]
    Condizione necessaria e sufficiente affinché un sistema meccanico sottoposto a vincoli ideali e alla forza peso come unica forza attiva sia in equilibrio è che la posizione del baricentro non si alzi né abbassi lungo gli spostamenti virtuali del sistema.
\end{prop}

\noindent Cosa succede se invece i vincoli sono unilateri?

\noindent In questo caso possiamo immaginare di introdurre comunque le coordinate libere $(q_1, \ldots, q_g) \in \mathbb{R}^g$, ma dobbiamo distinguere fra le configurazioni in cui il vincolo si comporta come bilatero (per esempio nell'appoggio quando c'è il contatto) e quelle in cui il vincolo è come se non ci fosse.

\begin{defi}
    Le configurazioni $\overline{q}$ di un sistema olonomo si dicono
    \begin{itemize}
        \item \textbf{configurazioni ordinarie} se $\overline{q} \in \Sigma^{\circ}$ (parte interna di $\Sigma$);
        \item \textbf{configurazione di confine} se $\overline{q} \in \partial \Sigma$
    \end{itemize}
    dove $\Sigma \subset \mathbb{R}^g$ è l'insieme dove vivono le coordinate generalizzate (spazio delle configurazioni ammissibili).
\end{defi}

\begin{oss}
    Lo spazio degli spostamenti virtuali cambia se la configurazione è ordinaria o meno: $S^{\prime}(\overline{q})
    \begin{cases}
         =\mathbb{R}^{g} \text { se } \overline{q} \in \Sigma^{\circ} \\
         \neq \mathbb{R}^{g} \text { se } \overline{q} \in \partial \Sigma
    \end{cases}$.
\end{oss}
\begin{prop}
    Una configurazione $\overline{q}^{(0)}$ ordinaria è di equilibrio $\iff Q_{k}\left(\overline{q}^{(0)}\right)=0 \;\forall k \in \{1, \ldots, g\}$. Per ogni configurazione confine $\exists \, \Bar{k} \in\{1, \ldots, g\}$ t.c. $\delta q_{\Bar{k}}^{\prime} \geq 0$ oppure $\delta q_{\Bar{k}}^{\prime} \leq 0$ e
    
    \noindent $\overline{q}^{(0)}$ è di equilibrio $\iff
    \begin{cases}
        Q_{k}\left(\overline{q}^{(0)}\right)=0 \quad \forall k \text { t.c. } \delta q_{k}^{\prime} \in \mathbb{R} \\
        Q_{k}\left(\overline{q}^{(0)}\right) \geq 0 \quad \forall k \text { t.c. } \delta q_{k}^{\prime} \leq 0 \\
        Q_{k}\left(\overline{q}^{(0)}\right) \leq 0 \quad \forall k \text { t.c. } \delta q_{k}^{\prime} \geq 0
    \end{cases}$ 
\end{prop}

\section{Statica del corpo rigido}

\begin{prop}
    Dato un corpo rigido sottoposto al sistema di forze esterne $\mathcal{S}^{(ext)}$, il lavoro virtuale totale è dato da
    \begin{displaymath}
    \boxed{
        \delta L^{\prime}=\overline{R}^{(ext)} \cdot \delta \overline{x}_{O}^{\prime}+\overline{M}_{O}^{(ext)} \cdot \overline{\varepsilon}
        }
    \end{displaymath}
    dove $O$ è un generico punto del corpo $\mathcal{C}$, $\delta \overline{x}_O^{\prime} \in S_O^{\prime}$ e $\overline{\varepsilon}=\overline{\omega} \delta t \in \mathbb{R}^{3}$.
\end{prop}

\begin{lemma}
    $\forall \; \overline{a}, \overline{b}, \overline{c}\in\mathbb{R}^3, \;\overline{a}\cdot(\overline{b}\wedge\overline{c})=\overline{b}\cdot(\overline{c}\wedge\overline{a})$.
\end{lemma}

\begin{proof}[Dimostrazione (Lemma)]
    \begin{displaymath}
        \overline{a}\cdot(\overline{b}\wedge\overline{c})=\sum_{ijk}a_i\varepsilon_{ijk}b_jc_k=\sum_{ijk}\varepsilon_{jki}a_ib_jc_k=\sum_{j=1}^3b_j\underbrace{\sum_{i,k}\varepsilon_{jki}c_ka_i}_{=(\overline{c}\wedge\overline{a})_j}=\overline{b}\cdot(\overline{c}\wedge\overline{a})
    \end{displaymath}
\end{proof}

\begin{proof}[Dimostrazione (Proposizione)]
    Come già dimostrato in precedenza $\overline{R}^{(int)}=\overline{0}$ e $\overline{M}_O^{(int)}=\overline{0}, \;\forall \;O\in\mathcal{E}_3$. Pertanto
    \[
    \begin{split}
        \delta L'&=\sum_{j=1}^N\overline{F}_j\cdot\delta\overline{x}_k'=\sum_{j=1}^N\overline{F}_j\cdot(\delta\overline{x}_O'+\overline{\varepsilon}\wedge\overline{OP_j}) \\
        &=\left(\sum_{j=1}^N\overline{F}_j\right)\cdot\overline{x}_O'+\overline{\varepsilon}\cdot\sum_{j=1}^N\overline{OP_j}\wedge\overline{F}_j=\overline{R}^{(ext)} \cdot \delta \overline{x}_{O}^{\prime}+\overline{M}_{O}^{(ext)} \cdot \overline{\varepsilon}
    \end{split}
    \]
\end{proof}

\begin{prop}
    Le equazioni cardinali della statica sono necessarie e \underline{sufficienti} per determinare la configurazione di equilibrio di un corpo rigido sottoposto a vincoli ideali.
\end{prop}

\begin{proof}
    Sappiamo già che le equazioni cardinali sono condizione necessaria. Quindi dobbiamo solo provare l'implicazione opposta ($\impliedby$).

    \noindent Se valgono le equazioni cardinali della statica, deve essere, per la proposizione 3.4,
    \begin{displaymath}
        \delta L'=0=\sum_{j=1}^N(\overline{F}_j^{(att)}\underbrace{+\overline{\Phi}_j)\cdot\delta\overline{x}_j'}_{\begin{cases}
            \geq 0 \quad \text{se vincoli ideali} \\
            = 0 \quad \text{se vincoli perfetti}
        \end{cases}} \implies \delta L'^{(att)}
        \begin{cases}
            \leq 0 \quad \text{se vincoli ideali} \\
            = 0 \quad \text{se vincoli perfetti}
        \end{cases}
    \end{displaymath}
    e quindi la configurazione è di equilibrio per il PLV.
\end{proof}

\begin{oss}
    Ai fini delle equazioni cardinali della statica di un corpo rigido si può sostituire $\mathcal{S}^{(ext)}$ con qualunque sistema di forze ad esso equivalente.
\end{oss}

\begin{defi}
    Il \textbf{poligono di appoggio} per un corpo rigido appoggiato su un piano $\pi$ in un numero finito di punti $\left\{A_j\right\}_{j \in \{1, \ldots, n\}}$ è l'unico poligono con le seguenti proprietà:
    \begin{itemize}
        \item ogni vertice è un punto d'appoggio;
        \item il poligono è convesso;
        \item i punti di appoggio che non sono vertici non sono esterni al poligono, ovvero ogni punto di appoggio non è esterno al poligono.
    \end{itemize}
\end{defi}

\begin{prop}
    Condizione necessaria e sufficiente affinché un corpo rigido appoggiato sia in una configurazione di equilibrio è che la proiezione di $G$ su $\pi$ cada dentro al poligono di appoggio.
\end{prop}

\part{Secondo parziale}

\chapter{Dinamica}

L'obiettivo è ora studiare i moti dei sistemi di punti materiali in funzione delle forze che agiscono su di essi.

\section{Quantità meccaniche}

Dato un punto materiale di massa $m>0$, le sue quantità meccaniche sono:
\begin{itemize}
    \item \textbf{quantità di moto} o \textbf{impulso}: $\overline{Q}=m\overline{v}$;
    \item \textbf{energia cinetica}: $T=\dfrac{1}{2}m|\overline{v}|^2$
    \item \textbf{momento angolare} rispetto al polo $O \in \mathcal{E}_3$: $\overline{K}_O = \overline{OP} \wedge \overline{Q}$;
    \item \textbf{potenza}: $\Pi = \overline{F} \cdot \overline{v}$.
\end{itemize}

\begin{oss}
    Le quantità meccaniche dipendono dalla scelta di un osservatore $\mathcal{O}$ e in particolare dal fatto che tale osservatore sia inerziale o meno.
\end{oss}

\begin{oss}
     Le quantità meccaniche sono \textbf{estensive}, cioè se si considera un sistema composto da due sottosistemi, le quantità meccaniche del sistema complessivo sono la somma di quelle dei sottosistemi.
\end{oss}

\begin{prop}[Legge di trasporto del momento angolare]
    Dati $\mathcal{O}$ e $\mathcal{O}^{\prime} \in \mathcal{E}_3$ si ha
    \begin{displaymath}
    \boxed{
        \overline{K}_{O^{\prime}}= \overline{K}_O + \overline{O^{\prime}O} \wedge \overline{Q}
        }
    \end{displaymath}
\end{prop}

\begin{proof}
    \begin{displaymath}
        \overline{K}_{O'}=\sum_{j=1}^Nm_j\overline{O'P_j}\wedge\overline{v}_j=\sum_{j=1}^Nm_j\overline{O'O}\wedge\overline{v}_j+\sum_{j=1}^Nm_j\overline{OP_j}\wedge\overline{v}_j=\overline{O'O}\wedge\sum_{j=1}^Nm_j\overline{v}_j+\overline{K}_O
    \end{displaymath}
\end{proof}

\begin{prop}
    L'impulso complessivo di un sistema di punti materiali è
    \begin{displaymath}
    \boxed{
        \overline{Q}= M \overline{v}_G
        }
    \end{displaymath}
\end{prop}

\begin{proof}
    \begin{displaymath}
        \overline{Q}=\sum_{j=1}^Nm_j\Dot{\overline{x}}_j=\Dot{\left(\sum_{j=1}^Nm_j\overline{x}_j\right)}=\Dot{\left(M\overline{x}_G\right)}=M\overline{v}_G
    \end{displaymath}
\end{proof}

\begin{prop}
    Sia $\mathcal{O}^{\prime}$ un osservatore solidale al centro di massa G (ovvero $O=G$) di un sistema di punti materiali, allora $\overline{Q}^{\prime}=\overline{0}$.
\end{prop}

\begin{proof}
    \begin{displaymath}
        \overline{Q}'=\sum_{j=1}^Nm_j\overline{v}_j'=\sum_{j=1}^Nm_j\left(\overline{v}_j-\overline{v}_G-\overline{\omega}\wedge\overline{GP_j'}\right)
    \end{displaymath}
    dove abbiamo indicato con $\overline{v}_j$ e $\overline{\omega}$ le velocità e la velocità angolare rispetto ad un osservatore inerziale.
    \begin{displaymath}
        \overline{Q}'=\cancel{\overline{Q}-M\overline{v}_G}-\overline{\omega}\wedge\sum_{j=1}^Nm_j\,\overline{GP_j'}=\sum_{j=1}^Nm_j\,(\overline{x}_j'-\overline{x}_G')=M(\overline{x}_G'-\overline{x}_G')=\overline{0}
    \end{displaymath}
\end{proof}

\begin{teo}[Konig 1]
    Per ogni sistema di punti materiali
    \begin{displaymath}
    \boxed{
        T=\frac{1}{2} M |\overline{v}_G|^2 + \frac{1}{2} \sum_{j=1}^N m_j |\overline{v}_j^{\prime}|^2
        }
    \end{displaymath}
    dove $\overline{v}_j^{\prime}$ sono le velocità relative misurate da un osservatore $\mathcal{O}^{\prime}$ solidale con $G$ e in moto traslatorio rispetto a $\mathcal{O}$.
\end{teo}

\begin{proof}
    \begin{displaymath}
        T=\frac{1}{2}\sum_{j=1}^Nm_j|\overline{v}_j|^2=\frac{1}{2}\sum_{j=1}^Nm_j|\overline{v}_j'+\overline{v}_G|^2=\frac{1}{2}\sum_{j=1}^Nm_j|\overline{v}_j'|^2+\frac{1}{2}M|\overline{v}_G|^2\cancel{+\sum_{j=1}^Nm_j\overline{v}_j'\cdot\overline{v}_G}
    \end{displaymath}
    poiché $\sum_{j=1}^Nm_j\overline{v}_j'=\overline{Q}'=\overline{0}$.
\end{proof}

\begin{teo}[Konig 2]
    Per ogni sistema di punti materiali
    \begin{displaymath}
    \boxed{
        \overline{K}_O =\overline{K}_G + M \, \overline{OG} \wedge \overline{v}_G
        }
    \end{displaymath}
    dove $\overline{K}_G = \sum_{j=1}^N m_j \,\overline{CP}_j \wedge \overline{v}_j^{\prime}$ è indipendente dal polo scelto $C$ per calcolarlo e $\overline{v}_j^{\prime}$ sono le velocità relative misurate da un osservatore $\mathcal{O}^{\prime}$ solidale con $G$ e in moto traslatorio.
\end{teo}

\begin{proof}
    \[
    \begin{split}
        \overline{K}_O&=\sum_{j=1}^Nm_j\,\overline{OP_j}\wedge\overline{v}_j=\sum_{j=1}^Nm_j\,\overline{OP_j}\wedge(\overline{v}_j'+\overline{v}_G)=\sum_{j=1}^Nm_j\,\overline{OP_j}\wedge\overline{v}_j'+\left(\sum_{j=1}^Nm_j\,(\overline{x}_j-\overline{x}_O)\right)\wedge\overline{v}_G \\
        &=\overline{K}_G+M\,(\overline{x}_G-\overline{x}_O)\wedge\overline{v}_G=\overline{K}_G + M \, \overline{OG} \wedge \overline{v}_G
    \end{split}
    \]
    Dimostriamo ora che $\overline{K}_G$ è indipendente dal polo scelto:
    \begin{displaymath}
        \sum_{j=1}^Nm_j\,\overline{OP_j}\wedge\overline{v}_j'=\sum_{j=1}^Nm_j\,\overline{CP_j}\wedge\overline{v}_j'+\cancel{\overline{OC}\wedge\sum_{j=1}^Nm_j\overline{v}_j'}=\overline{Q}'=\overline{0}
    \end{displaymath}
\end{proof}

\section{Quantità meccaniche per il corpo rigido}

\begin{defi}
    Dato un asse $r: \overline{x} = \lambda \hat{u} + \overline{x}_Q$, il \textbf{momento di inerzia} di un corpo rigido rispetto a $r$ è:
    \begin{displaymath}
        I_r=
        \begin{dcases}
        \sum_{j=1}^N m_j dist^2(P_j; r) \quad \quad \; \; \; \text{discreto} \\
        \int_{\mathbb{R}^d} d\overline{x} \rho \left(\overline{x}\right) dist^2(P; r) \quad \text{continuo}
        \end{dcases}
    \end{displaymath}
\end{defi}

\begin{oss}
    La distanza di un punto $P$ da una retta $r: \overline{x} = \lambda \hat{u} + \overline{x}_Q$ è data da
    \begin{displaymath}
    \boxed{
        dist(P;r) = \left |\overline{QP}\wedge\hat{u} \right|
        }
    \end{displaymath}
    Si noti che è cruciale che il vettore direttore $\hat{u}$ sia un versore di modulo unitario perché l'identità sia valida.
\end{oss}

\begin{teo}[Huygens-Steiner]
    Dato un asse $r: \overline{x}=\lambda \hat{u} + \overline{x}_Q$ si ha
    \begin{displaymath}
        \boxed{
        I_r=I_{r_G} + M\,dist^2(G;r)
        }
    \end{displaymath}
    dove $r_G$ è l'asse $\parallel r$ passante per il centro di massa G.
\end{teo}

\begin{proof}
    \[
    \begin{split}
        I_r&=\sum_{j=1}^Nm_j\left|\overline{QP_j}\wedge\hat{u}\right|^2=\sum_{j=1}^Nm_j\left|(\overline{QG}+\overline{GP_j})\wedge\hat{u}\right|^2 \\
        &=\sum_{j=1}^Nm_j\left|\overline{QG}\wedge\hat{u}\right|^2+\sum_{j=1}^N2m_j\left(\overline{QG}\wedge\hat{u}\right)\cdot\left(\overline{GP_j}\wedge\hat{u}\right)+\sum_{j=1}^Nm_j\left|\overline{GP_j}\wedge\hat{u}\right|^2
    \end{split}
    \]
    ma
    \begin{itemize}
        \item $\sum_{j=1}^Nm_j\left|\overline{QG}\wedge\hat{u}\right|^2=M\,dist^2(G;r)$;
        \item $\overline{QG}\wedge\hat{u}\cdot\left[\left(\sum_{j=1}^Nm_j\,\overline{GP_j}\right)\wedge\hat{u}\right]=0$ perché $m_j\,\overline{GP_j}=0$ per definizione di centro di massa;
        \item $\sum_{j=1}^Nm_j\left|\overline{GP_j}\wedge\hat{u}\right|^2=I_{r_G}$.
    \end{itemize}
\end{proof}

\begin{defi}
    Il \textbf{tensore di inerzia} di un sistema meccanico rispetto al polo $O \in \mathcal{E}$ è la matrice simmetrica $\mathbb{I} \in M_{sym}(\mathbb{R}^3)$ data da
    \begin{displaymath}
        \mathbb{I}_{lk}^{(O)}=
        \begin{dcases}
            \sum_{j=1}^N m_j\left\{\left|\overline{OP}_j\right|^2 \delta_{lk} - \left(\overline{OP}_j\right)_l\left(\overline{OP}_j\right)_k\right\} \\
            \int_{\mathbb{R}^d} d\overline{x} \rho \left(\overline{x}\right) \left\{\left |\overline{x}-\overline{x}_O\right|^2\delta_{lk} - \left(\overline{x}-\overline{x}_O\right)_l\left(\overline{x}-\overline{x}_O\right)_k\right\}
        \end{dcases}
        k,l \in \{1, 2, 3\}
    \end{displaymath}
\end{defi}

\begin{prop}
\everymath{\displaystyle}
    Dato un asse $r: \overline{x}=\lambda\hat{u}+\overline{x}_Q$ si ha $I_r = \hat{u}\mathbb{I}^{(Q)}\hat{u} = \sum_{l,k=1}^3\hat{u}_l\mathbb{I}^{(Q)}_{lk}\hat{u}_k$.
\end{prop}

\begin{oss}
    $\mathbb{I}^{(O)}$ è una matrice simmetrica e come tale diagonalizzabile con una matrice ortogonale, ovvero $\exists \; I_1,I_2,I_3 \in \mathbb{R}^+$ e una matrice $O \in O_3(\mathbb{R})$ t.c. $O\mathbb{I}^{(O)}O^T=diag(I_1,I_2,I_3)$.
\end{oss}

\begin{defi}
   Gli autovalori di $\mathbb{I}^{(O)}$  sono detti \textbf{momenti principali di inerzia} e i relativi autovettori (ortogonali fra di loro) sono i vettori direttori di 3 assi passanti per $O$ detti \textbf{assi principali di inerzia}.
\end{defi}

\begin{prop}
    $\forall \overline{u}\in\mathbb{R}^3, \mathbb{I}^{(0)}\overline{u}=
    \begin{dcases}
        \sum^n_{j=1}m_j\,\overline{OP}_j\wedge(\overline{u}\wedge\overline{OP}_j) \\
        \int_{\mathbb{R}^3} d\overline{x} \rho \left(\overline{x}\right) \left(\overline{x}-\overline{x}_O\right)\wedge\left[\overline{u}\wedge\left(\overline{x}-\overline{x}_O\right)\right]
    \end{dcases}$
\end{prop}

\begin{prop}
    Dato un generico polo $O \in \mathcal{E}_3$, il momento angolare di un corpo rigido soddisfa le relazioni
    \begin{displaymath}
    \boxed{
        \overline{K}_O=
        \begin{cases}
            M\,\overline{OG}\wedge\overline{v}_G +\mathbb{I}^{(G)}\overline{\omega} \\
            M\,\overline{OG}\wedge\overline{v}_O +\mathbb{I}^{(O)}\overline{\omega}
        \end{cases}
        }
    \end{displaymath}
\end{prop}

\begin{proof}
    \[
    \begin{split}
        \overline{K}_O&=\sum_{j=1}^Nm_j\,\left(\overline{OP_j}\wedge\overline{v}_j\right)\sum_{j=1}^Nm_j\,\overline{OP_j}\wedge\begin{dcases}
            \overline{v}_G+\overline{\omega}\wedge\overline{GP_j} \\
            \overline{v}_O+\overline{\omega}\wedge\overline{OP_j}
        \end{dcases} \\
        &=\begin{dcases}
            \left(\sum_{j=1}^Nm_j\,\overline{OP_j}\right)\wedge\overline{v}_G+\sum_{j=1}^Nm_j\,\overline{OP_j}\wedge\left(\overline{\omega}\wedge\overline{GP_j}\right)+\sum_{j=1}^Nm_j\,\overline{GP_j}\wedge\left(\overline{\omega}\wedge\overline{GP_j}\right) \\
            \left(\sum_{j=1}^Nm_j\,\overline{OP_j}\right)\wedge\overline{v}_O+\sum_{j=1}^Nm_j\,\overline{OP_j}\wedge\left(\overline{\omega}\wedge\overline{OP_j}\right)
        \end{dcases} \\
        &=\begin{dcases}
            M\,\overline{OG}\wedge\overline{v}_G+\cancel{\left(\sum_{j=1}^Nm_j\,\overline{GP_j}\right)}\wedge\overline{v}_G+\overline{OG}\wedge\left(\overline{\omega}\wedge\cancel{\sum_{j=1}^Nm_j\,\overline{GP_j}}\right)+\mathbb{I}^{(G)}\overline{\omega} \\
            M\,\overline{OG}\wedge\overline{v}_O+\left(\sum_{j=1}^Nm_j\,\overline{GP_j}\right)\wedge\overline{v}_O+\mathbb{I}^{(O)}\overline{\omega}
        \end{dcases}
    \end{split}
    \]
\end{proof}

\begin{oss}
     Il momento angolare si può decomporre in quello del centro di massa più quello relativo al centro di massa.
\end{oss}

\begin{oss}
    Se $O=G \Longrightarrow \overline{K}_G=\mathbb{I}^{(G)}\overline{\omega}$.
\end{oss}

\begin{oss}
    Se $O$ coincide con il CIR o è un punto fisso, allora $\overline{K}_O=\mathbb{I}^{(O)}\overline{\omega}$.
\end{oss}

\begin{prop}
    L'energia cinetica di un corpo rigido è data da $T=\dfrac{1}{2}M\left|\overline{v}_G\right|^2 + \dfrac{1}{2}\overline{\omega}\mathbb{I}^{(G)}\overline{\omega}$.
\end{prop}

\begin{proof}
    \[
    \begin{split}
        T&=\frac{1}{2}\sum_{j=1}^Nm_j|\overline{v}_j|^2=\frac{1}{2}\sum_{j=1}^Nm_j\,|\overline{v}_G+\overline{\omega}\wedge\overline{GP_j}|^2=\frac{1}{2}\sum_{j=1}^Nm_j\,\left(|\overline{v}_G|^2+2\overline{v}_G\cdot\overline{\omega}\wedge\overline{GP_j}+|\overline{\omega}\wedge\overline{GP_j}|^2\right) \\
        &=\frac{1}{2}\left(\sum_{j=1}^Nm_j\right)|\overline{v}_G|^2+\overline{v}_G\cdot\omega\wedge\cancel{\left(\sum_{j=1}^Nm_j\,\overline{GP_j}\right)}+\frac{1}{2}\sum_{j=1}^Nm_j\,\left(\overline{\omega}\wedge\overline{GP_j}\right)\cdot\left(\overline{\omega}\wedge\overline{GP_j}\right) \\
        &=\frac{1}{2}M|\overline{v}_G|^2+\frac{1}{2}\overline{\omega}\cdot\sum_{j=1}^Nm_j\,\overline{GP_j}\wedge\left(\overline{\omega}\wedge\overline{GP_j}\right)=\frac{1}{2}M\left|\overline{v}_G\right|^2 + \dfrac{1}{2}\overline{\omega}\mathbb{I}^{(G)}\overline{\omega}
    \end{split}
    \]
\end{proof}

\begin{oss}
    L'energia cinetica di un corpo rigido si decompone nell'energia cinetica "del centro di massa", ovvero di un punto materiale di massa $M$ e velocità $\overline{v}_G$ più l'energia cinetica rotazionale rispetto a $G$.
\end{oss}

\begin{prop}
    Se $C$ è il centro di istantanea rotazione del corpo rigido allora $T=\frac{1}{2}\overline{\omega}\mathbb{I}^{(C)}\overline{\omega}$.
\end{prop}

\begin{cor}
    Se $\omega=\omega(t)\hat{k}$ (moto piano) allora $T=\frac{1}{2}I_r\omega^2(t)$ con $r: \overline{x}=\lambda\hat{k}+\overline{x}_{C}$ e $C$ centro di istantanea rotazione.
\end{cor}

\noindent Rispetto ad un osservatore $\mathcal{O}$ che trasla con il centro di massa $G$ del corpo rigido si ha $T_G=\frac{1}{2}\overline{\omega}\mathbb{I}^{(G)}\overline{\omega}$ e tale quantità in assenza di forze esterne è conservata. Risulta quindi interessante studiare la forma quadratica $\overline{\omega}\mathbb{I}^{(G)}\overline{\omega}$ e la sua relazione con le proprietà del tensore di inerzia (si noti che $T_G=I_{r_{\omega}}\omega^2$), dove $r_{\omega}=\lambda\hat{\omega}+\overline{x}_G$.

\begin{defi}
    L'\textbf{ellissoide di inerzia} associato ad un corpo rigido di tensore di inerzia $\mathbb{I}^{(O)}$ è il luogo geometrico
    \begin{displaymath}
        \mathcal{E}:=\left\{\overline{\omega} \in \mathbb{R}^3\;|\; \overline{\omega}\,\mathbb{I}^{(O)}\overline{\omega} =1\right\}
    \end{displaymath}
\end{defi}

\begin{oss}
    Gli assi principali di inerzia sono gli assi dell'ellissoide che nelle coordinate rispetto a tali assi si può scrivere $\mathcal{E}:=\left\{\overline{\omega} \in \mathbb{R}^3\;|\;I_1\omega_1^2 + I_2\omega_2^2 + I_3\omega_3^2 =1\right\}$ da cui si deduce che i semiassi dell'ellissoide sono lunghi $\frac{1}{\sqrt{I_j}}$, $j=1, 2, 3$.
    
    \noindent Se almeno due di tali assi sono uguali allora l'ellissoide è invariante per rotazioni rispetto al 3° asse e diremo che si tratta di un \textbf{ellissoide di rotazione} ed il corpo rigido associato si chiamerà \textbf{giroscopio}.
\end{oss}

\begin{oss}
     Se $I_1<I_2<I_3$ allora i tre assi sono univocamente determinati e $\hat{e}_1 \perp \hat{e}_2 \perp \hat{e}_3$.
     Se invece $I_1=I_2$ allora ogni asse $\in$ piano $\perp \hat{e}_3$ è un asse di inerzia. Viceversa se si trovano due assi di inerzia non ortogonali allora i rispettivi momenti coincidono.
\end{oss}

\begin{defi}
    Un corpo rigido ha un \textbf{asse di simmetria} di ordine $ n\ge2$ se il corpo rigido si sovrappone a se stesso dopo rotazione di $\frac{2\pi}{n}$ attorno a tale asse. Se l'asse di simmetria è di ordine $n$, $\forall \;n\in\mathbb{N}_0$, allora diremo che è un \textbf{asse di simmetria rotazionale}.
\end{defi}

\begin{oss}
    Se $r$ identifica un asse di simmetria di ordine $n$ per $\mathcal{C} \Longrightarrow r$ asse di simmetria di ordine $\ge n$ per $\mathcal{E}_O$. 
\end{oss}

\begin{oss}
    Alcune proprietà:
    \begin{itemize}
        \item ogni asse principale di inerzia è un asse di simmetria di ordine $\geq 2$ per $\mathcal{E}_O$.
        \item ogni asse di simmetria di ordine $\geq 2$ per $\mathcal{E}_O$ è un asse principale di inerzia.
        \item ogni asse di simmetria di ordine $> 2$ per $\mathcal{E}_O$ è un asse di simmetria rotazionale.
        \item $\exists \; e_1, e_2$ assi di simmetria di ordine 2 per $\mathcal{E}_O$ con angolo fra i due $\neq \frac{\pi}{2} \implies \mathcal{E}_O$ ellissoide di rotazione.
        \item $\exists \; e_1, e_2$ assi di simmetria di ordine 2 per $\mathcal{E}_O \implies \mathcal{E}_O =$ sfera.
        \item $\exists \; e_1, e_2$ assi di simmetria di ordine 2 per $\mathcal{E}_O$ con angolo fra i due $\neq \frac{\pi}{2} \implies \mathcal{E}_O=$ sfera.
    \end{itemize}
\end{oss}

\begin{prop}
    Dato un corpo rigido $\mathcal{C}$:
    \begin{itemize}
        \item $\mathcal{C}$ ha un asse di simmetria $e$ di ordine $> 2 \implies e$ asse principale di inerzia;
        \item $\mathcal{C}$ ha due assi di simmetria di ordine $> 2 \implies e$ ellissoide $=$ sfera e $I_1=I_2=I_3$;
        \item $\mathcal{C}$ ha due assi di simmetria di cui uno di ordine $> 2$ e con angolo $\neq \frac{\pi}{2}\implies e$ ellissoide $=$ sfera e $I_1=I_2=I_3$.
    \end{itemize}
\end{prop}

\begin{proof}
    Il risultato segue dalle analoghe proprietà dell'ellissoide di inerzia.
\end{proof}

\begin{prop}
    Se $\mathcal{C}$ ha un asse di simmetria $e$ allora $G \in e$.
\end{prop}

\begin{proof}
    La posizione del centro di massa deve essere invariante rispetto alle simmetrie di $\mathcal{C} \implies G \in e$ perché $e$ contiene gli unici punti invarianti sotto la simmetria.
\end{proof}

\begin{cor}
    Se $\mathcal{C}$ ha due assi di simmetria distinti allora $G$ è all'intersezione dei due.
\end{cor}

\section{Equazioni cardinali della dinamica}

\begin{teo}
    Lungo un qualunque moto di un sistema di punti materiali sono verificate le \textbf{equazioni cardinali della dinamica}:
    \begin{displaymath}
    \boxed{
        \dot{\overline{Q}}=\overline{R}^{(ext)}, \ \ \ \ \ \dot{\overline{K}}_O=\overline{M}_O^{(ext)}-\dot{\overline{x}}_O\wedge\overline{Q}
        }
    \end{displaymath}
    con $\overline{R}^{(ext)}$ e $\overline{M}_O^{(ext)}$ la risultante e il momento risultante del sistema di forze esterne.
\end{teo}

\begin{proof}
    \begin{displaymath}
        \Dot{\overline{Q}}=\sum_{j=1}^Nm_j\Dot{\overline{v}}_j=\sum_{j=1}^N\overline{F}_j=\overline{R}^{(ext)}
    \end{displaymath}
     perché $\overline{R}^{(int)}=\overline{0}$.
    \[
    \begin{split}
        \Dot{\overline{K}}_O&=\sum_{j=1}^Nm_j\Dot{\left(\overline{OP_j}\wedge\overline{v}_j\right)}=\sum_{j=1}^Nm_j\left[\left(\Dot{\overline{x}}_j-\Dot{\overline{x}}_O\right)\wedge\overline{v}_j+\overline{OP_j}\wedge\Dot{\overline{v}}_j\right] \\
        &=-\overline{v}_O\wedge\overline{Q}+\sum_{j=1}^N\overline{OP_j}\wedge\overline{F}_j=\overline{M}_O^{(ext)}-\overline{v}_O\wedge\overline{Q}
    \end{split}
    \]
     perché $\overline{M}_O^{(int)}=\overline{0} \;\; \forall O \in \mathcal{E}_3$.
\end{proof}

\begin{oss}
     Come le equazioni cardinali della statica, le equazioni cardinali della dinamica sono condizioni necessarie ma non sufficienti a determinare il moto.
\end{oss}

\begin{oss}
    Il polo $O$ della seconda equazione non è necessariamente un punto del sistema.
\end{oss}

\begin{oss}
    $\dot{\overline{x}}_O$ nel secondo termine della seconda equazione non indica la velocità del punto del sistema, ma la velocità del punto che si sposta nello spazio. Ad esempio se $O$ è il punto di contatto in un puro rotolamento lungo una guida fissa  $\dot{\overline{x}}_O \neq \overline{0}$.
\end{oss}

\begin{oss}
    Le equazioni cardinali della dinamica non sono equazioni pure.
\end{oss}

\begin{cor}
    Per un corpo rigido, se scegliamo $O=C$ con $C$ il centro di istantanea rotazione oppure un punto fisso o se prendiamo $O=G$, otteniamo $\dot{\overline{K}}_O=\overline{M}^{(ext)}$.
\end{cor}

\begin{proof}
    Nel caso $O=CIR$ o punti fisso si ha $\overline{v}_O=\overline{0}$ mentre nel caso $O=G$
    
    \noindent $\overline{v}_G\wedge\overline{Q}=M\,\overline{v}_G\wedge\overline{v}_G=\overline{0}$.
\end{proof}

\begin{teo}[Teorema dell'energia cinetica]
    Dato un qualunque sistema meccanico, si ha
    \begin{displaymath}
    \boxed{
        \dot{T}=\Pi
        }
    \end{displaymath}
    con $\Pi$ la potenza complessiva delle forze del sistema.
\end{teo}

\begin{proof}
    \begin{displaymath}
        \Dot{T}=\sum_{j=1}^Nm_j\,\overline{v}_j\cdot\dot{\overline{v}}_j=\sum_{j=1}^N\overline{v}_j\cdot\overline{F}_j=\Pi
    \end{displaymath}
\end{proof}

\begin{cor}
    Se il sistema è sottoposto a vincoli perfetti e fissi allora
    \begin{displaymath}
    \boxed{
        \dot{T}=\Pi^{(att)}
        }
    \end{displaymath}
\end{cor}

\begin{lemma}
    Se un sistema è sottoposto a vincoli perfetti e fissi, allora gli spostamenti reali sono anche spostamenti virtuali.
\end{lemma}

\begin{proof}[Dimostrazione (Corollario)]
    Per definizione di vincoli perfetti $\Pi'=\sum_{j=1}^N\overline{\Phi}_j\cdot\overline{v}_j'=0$ dove $\overline{\Phi}_j$ sono le reazione vincolari e $\overline{v}_j'\in V_j'$ ma gli spostamenti reali sono anche virtuali se i vincoli sono perfetti e fissi per cui $\Pi=\Pi^{(att)}$.
\end{proof}

\begin{oss}
    In presenza di vincoli perfetti e fissi l'equazione data dal teorema dell'energia cinetica è un'equazione pura.
\end{oss}

\begin{oss}
    Per un sistema meccanico generico che non sia rigido le forze interne possono produrre una potenza $\Pi^{(int)} \neq 0$.
\end{oss}

\begin{oss}
    Le forze interne non compaiono direttamente nelle equazioni cardinali e non possono quindi modificare la dinamica complessiva del sistema. Tuttavia possono modificare la forma del corpo (se non è rigido) e indirettamente cambiare la dinamica se le forze esterne dipendono dalla forma (ad esempio il paracadute).
\end{oss}

\begin{teo}[Composizione dell'energia meccanica]
    Dato un sistema meccanico sottoposto solo a forze conservative di potenziale $U(\overline{x}_1, \ldots, \overline{x}_N)$ cioè tali che $\overline{F}_j=-\nabla_j U$,  allora l'\textbf{energia meccanica}
    \begin{displaymath}
    \boxed{
        E=T+U
    }     
    \end{displaymath}
    è un \textbf{integrale primo del moto}, cioè è tale che $\dot{E}=0$ lungo qualunque moto del sistema.
\end{teo}

\begin{proof}
    \begin{displaymath}
        \Dot{E}=\Dot{T}+\Dot{U}=\Pi-\Pi=0
    \end{displaymath}
    perché
    \begin{displaymath}
        \Dot{U}=\frac{d}{dt}U=\sum_{j=1}^N\nabla_jU\cdot\frac{d}{dt}\overline{x}_j=\sum_{j=1}^N\nabla_jU\cdot\Dot{\overline{x}}_j=-\sum_{j=1}^N\overline{F}_j\cdot\overline{v}_j=-\Pi
    \end{displaymath}
\end{proof}

\begin{oss}
    La conservazione dell'energia meccanica si applica quando \underline{tutte} le forze sono conservative.
\end{oss}

\section{Dinamica dei Sistemi Olonomi}

\begin{prop}
    Dato un sistema olonomo con energia cinetica $T\left(\overline{q};\dot{\overline{q}}\right)=\frac{1}{2}\dot{\overline{q}}A\left(\overline{q}\right)\dot{\overline{q}}$ e potenziale $U\left(\overline{q}\right)$, l'energia meccanica è un integrale primo del moto
    \begin{displaymath}
        \boxed{
        E=\frac{1}{2}\dot{\overline{q}}A\left(\overline{q}\right)\dot{\overline{q}}+U\left(\overline{q}\right)
        }
    \end{displaymath}
\end{prop}

\begin{lemma}
    Dato un sistema olonomo sottoposto a vincoli fissi si ha
    \begin{displaymath}
        T=T\left(\overline{q};\dot{\overline{q}}\right)=\frac{1}{2}\dot{\overline{q}}A\left(\overline{q}\right)\dot{\overline{q}}
    \end{displaymath}
    dove $A\left(\overline{q}\right)\in M_g^{sym}(\mathbb{R})$ è una matrice simmetrica $g \times g$ detta \textbf{matrice di massa} data da
    \begin{displaymath}
        \boxed{
        \left[A\left(\overline{q}\right)\right]_{lk}=\sum_{j=1}^N m_j\frac{\partial\overline{x}_j}{\partial q_l} \cdot\frac{\partial\overline{x}_j}{\partial q_k}
        }
    \end{displaymath}
\end{lemma}

\begin{proof}[Dimostrazione (Lemma)]
    \[
    \begin{split}
        T&=\sum_{j=1}^N\frac{1}{2}m_j|\overline{v}_j|^2=\sum_{j=1}^N\frac{1}{2}m_j\frac{d\overline{x}_j}{dt}\cdot\frac{d\overline{x}_j}{dt} \\
        &=\frac{1}{2}\sum_{j=1}^Nm_j\sum_{l,k=1}^g\frac{\partial\overline{x}_j}{\partial q_l}\Dot{q}_l\cdot\frac{\partial\overline{x}_j}{\partial q_k}\Dot{q}_k=\frac{1}{2}\sum_{l,k=1}^g\Dot{q}_l\left[A\left(\overline{q}\right)\right]_{lk}\Dot{q}_k
    \end{split}
    \]
\end{proof}

\begin{oss}
    In caso di vincoli tempo-dipendenti (ma perfetti e olonomi) si può generalizzare la forma dell'energia cinetica:
    \begin{displaymath}
    \boxed{
        T=\frac{1}{2}\dot{\overline{q}}A(\overline{q},t)\dot{\overline{q}} +\overline{b}(\overline{q},t)\cdot\dot{\overline{q}} +c(\overline{q},t)
        }
    \end{displaymath}
    con $b_k=\sum_{j=1}^N m_j\frac{\partial\overline{x}_j}{\partial q_k} \cdot\frac{\partial\overline{x}_j}{\partial t}$ e $c=\sum_{j=1}^N m_j\frac{\partial\overline{x}_j}{\partial t} \cdot\frac{\partial\overline{x}_j}{\partial t}$.
\end{oss}

\begin{prop}
    La matrice di massa $A(\overline{q})$ è tale che $\forall \; \overline{q}\in\mathbb{R}^g$
    \begin{itemize}
        \item $A(\overline{q})>0$;
        \item  $A(\overline{q})$ è invertibile.
    \end{itemize}
\end{prop}

\section{Analisi qualitativa del moto (in 1D)}

Consideriamo un sistema olonomo conservativo 1D ovvero $g=1$ per cui l'energia meccanica $E=\frac{1}{2}m\dot{q}^2-U(q)$ è conservata (vincoli fissi e matrice di massa indipendente da $q$).

\begin{defi}
    Lo \textbf{spazio delle fasi} $\Gamma = \left\{(q, v) \in \mathbb{R}^2\right\}$ è l'insieme delle possibili posizioni e velocità (coppie $(q,\dot{q})$) del sistema.
\end{defi}

\begin{defi}
    Dato un valore di $E\in\mathbb{R}$ dell'energia, la \textbf{curva di livello} corrispondente è
    \begin{displaymath}
        \boxed{
        \Gamma_E:=\left\{(q,v)\in\Gamma \; | \; \frac{1}{2}mv^2+U(q)=E\right\}
        }
    \end{displaymath}
\end{defi}

\begin{prop}
    Le curve di livello soddisfano le seguenti proprietà:
    \begin{itemize}
        \item sono simmetriche rispetto all'asse $q$;
        \item $\Gamma_E\cap\Gamma_{E^{\prime}} = \emptyset$ per $E\neq E^{\prime}$;
        \item $\operatorname{supp}(\Gamma_E)\subset \left\{q\in\mathbb{R} \; |\; U(q) \leq E\right\}$.
    \end{itemize}
    Inoltre l'equazione delle curve di livello è $v=\pm\sqrt{\dfrac{2}{m}\left(E-U(q)\right)}$. 
\end{prop}

\begin{proof}
    L'equazione delle curve di livello è conseguenza diretta della conservazione dell'energia e in particolare implica che $\operatorname{supp}(\Gamma)\subset\{q\;|\; U(q) \leq E\}$ poiché $T\geq0$ e quindi il tero punto è dimostrato.
    \begin{enumerate}
        \item Poiché per ogni soluzione $v$ anche $-v$è soluzione le curve sono simmetriche rispetto all'asse $q$;
        \item non si intersecano perché date due energie $E_1\neq E_2$ se $\exists\;(q_*, v_*)\in\Gamma_{E_1}\cap\Gamma_{E_2}$ si avrebbe
        
        \noindent $E_1=\frac{1}{2}mv_*^2+U(q_*)=E_2$, che è assurdo.
    \end{enumerate}
\end{proof}

\begin{oss}
    Non tutti i valori di $E$ sono ammessi:$E\ge \inf_{q\in\mathbb{R}}U(q)$.
\end{oss}

\begin{defi}
    Le \textbf{orbite} $\gamma_E$ del sistema sono date da
    \begin{displaymath}
        \boxed{
        \gamma_E := \left\{(q(t),\dot{q}(t))\in\Gamma_E \; \text{al variare di} \; t \in \mathbb{R}\right\}
        }
    \end{displaymath}
    dove $q(t)$ è un moto del sistema di energia $E$.
\end{defi}

\begin{prop}

\noindent 
\begin{itemize}
    \item Le orbite sono percorse in senso orario, ovvero da sinistra a destra nel semipiano superiore e da destra a sinistra nel semipiano inferiore;
    \item fissate $q_1<q_2 \in \operatorname{supp}(\gamma_E)$ t.c. $v_1, v_2 \neq 0$ il tempo impiegato a percorrere l'orbita da $q_1$ a $q_2$ è
\end{itemize}
     \begin{displaymath}
         T=\int_{q_1}^{q_2}\sqrt{\frac{m}{2}}\frac{dq}{\sqrt{E-U(q)}}
     \end{displaymath}
\end{prop}

\begin{proof}

\noindent 
    \begin{enumerate}
        \item Segue dal fatto che $\Dot{q}>0$ nel semipiano superiore e $<0$ in quello inferiore;
        \item dalla conservazione di $E$ segue che $\int_{q_1}^{q_2}\frac{dq}{\sqrt{\frac{2}{m}}\sqrt{E-U(q)}}=\int_{t_1}^{t_2}dt$ da cui segue il risultato.
    \end{enumerate}
\end{proof}

\begin{defi}
\everymath{\displaystyle}
    I \textbf{valori critici} $\Lambda_C$ del potenziale $U$ sono:
    \begin{itemize}
        \item $U(q_*)$ tali che $q_*$ punto critico di $U$, cioè $U^{\prime}(q_*)=0$;
        \item gli eventuali asintoti orizzontali di $U$ cioè $\lim_{q\to\pm\infty}U(q)$ (se il limite $\exists$ finito).
    \end{itemize}
\end{defi}

\begin{defi}
    I punti della forma $(q_0,0)\in\gamma_E$ sono detti \textbf{punti di inversione} se $U(q_0)\notin \Lambda_C$.
\end{defi}

\begin{oss}
    In corrispondenza dei punti di inversione il moto cambia verso.
\end{oss}

\begin{defi}
    Diciamo che un'orbita è:
    \begin{itemize}
        \item \textbf{chiusa} quando la curva $\gamma_E$ è chiusa e \textbf{aperta} altrimenti;
        \item \textbf{limitata} quando $\operatorname{supp}(\gamma_E)$ è contenuto in un compatto e \textbf{illimitata} altrimenti;
        \item \textbf{periodica} se $\exists \; T>0 \; t.c. \; q(t+T)=q(t) \; \forall t\in \mathbb{R}$.
    \end{itemize}
\end{defi}

\begin{prop}
    Sia $E\in\mathbb{R} \; t.c. \; E \notin\Lambda_C$, allora ogni punto di $\gamma_E$ è raggiunto in un tempo finito (nel futuro o nel passato).
\end{prop}

\begin{proof}
    Abbiamo già dimostrato che così è per ogni coppia di punti con $v\neq0$, quindi resta solo da studiare i punti di inversione $(q_0,0)$. Il tempo di percorrenza è dato da
    \begin{displaymath}
        \left|\int_{q_0}^{q_1}\frac{dq}{\sqrt{\frac{2}{m}(E-U(q))}}\right|
    \end{displaymath}
    per cui resta solo da dimostrare che l'integrale è finito. Tuttavia il denominatore si annulla nell'estremo $q_0$ per cui è necessario indagare la divergenza dell'integrale: poiché $q_0$ non è un punto stazionario di $U(q), U'(q_0)\neq0$ e quindi in un intorno di $q_0$
    \begin{displaymath}
        U(q)=U(q_0)+U'(q_0)(q-q_0)+o((q-q_0)^2)
    \end{displaymath}
    e l'integrale si comporta come
    \begin{displaymath}
        \frac{1}{\sqrt{\frac{2}{m}(E-U(q))}}\sim\frac{c}{\sqrt{|q-q_0|}}
    \end{displaymath}
    che è integrabile.
\end{proof}

\begin{cor}
    Sia $E$ con le stesse ipotesi e sia $\gamma_E$ un'orbita chiusa e limitata, allora $\gamma_E$ è periodica di periodo
    \begin{displaymath}
        T=2\int_{q_0^{(1)}}^{q_0^{(2)}}\sqrt{\frac{m}{2}}\frac{dq}{\sqrt{E-U(q)}}
    \end{displaymath}
    dove ${q_0^{(i)}}$ sono i due punti di inversione dell'orbita.
\end{cor}

\begin{prop}
    Nei punti di inversione il grafico di $\gamma_E$ è verticale e la funzione $v(q)$ è non derivabile.
\end{prop}

\noindent Fin'ora abbiamo considerato solo valori non \underline{critici} dell'energia. D'ora in poi assumeremo $E\in\Lambda_C$. Cominciamo discutendo il caso $E=U(q_*)$ con $q_*$ tale che $U'(q_*)=0$:

\begin{itemize}
    \item $q_*$ \textbf{minimo isolato} (\textbf{stabile})
    \begin{itemize}
        \item $\exists \; \gamma_*=\{(q_*, 0)\}$ cioè c'è sempre un'orbita che corrisponde al solo punto di equilibrio;
        \item per energia vicine $E=U(q_*)+\varepsilon$ con $\varepsilon\ll1$ si ottengono orbite chiuse limitate e periodiche che si svolgono in un intorno del punto di equilibrio e non si allontanano mai da esso.
    \end{itemize}
    \item $q_*$ \textbf{massimo isolato} (\textbf{instabile})
    \begin{itemize}
        \item $\exists \; \gamma_*=\{(q_*,0)\}$, ma per $E=U(q_*) \; \exists \; \gamma_{*, j} \subset \Gamma_E$ altre orbite, $j=1, \ldots, 4$, dette \textbf{separatrici};
        \item le separatrici $\gamma_{*, j}$ sono disgiunte da $\gamma_*$, cioè $(q_*, 0) \notin \gamma_{*, j}$, quindi in totale sono $5$ componenti connessi;
        \item per ogni valore di $E$ vicino a $U(q_*)$ esistono orbite lungo cui il sistema si allontana dall'intorno di $q_*$.
    \end{itemize}
    \item $q*$ \textbf{flesso} ascendente o discendente (\textbf{instabile})
    \begin{itemize}
        \item $\exists \; \gamma_*=\{(q_*,0)\}$, ma per $E=U(q_*) \; \exists \; \gamma_{*, j} \subset \Gamma_E$, $j=1, 2$ \textbf{separatrici};
        \item le separatrici $\gamma_{*, j}$ sono disgiunte da $\gamma_*$, cioè $(q_*, 0) \notin \gamma_{*, j}$, quindi in totale sono $3$ componenti connessi;
        \item per ogni valore di $E$ vicino a $U(q_*)$ esistono orbite lungo cui il sistema si allontana dall'intorno di $q_*$.
    \end{itemize}
\end{itemize}

\begin{prop}
    Sia $q_*$ un punto di massimo isolato o di flesso di $U(q)$ e sia $E=U(q_*)$. Sia $\gamma_{*, j}$ una separatrice, allora il punto di equilibrio $(q_*, 0)$ è raggiunto da qualunque punto di $\gamma_{*, j}$ in un tempo $\infty$ (nel futuro o nel passato).
\end{prop}

\begin{proof}
    Il tempo di percorrenza è
    \begin{displaymath}
        \left|\int_{q_*}^{q_1}\frac{dq}{\sqrt{\frac{2}{m}(E-U(q))}}\right|
    \end{displaymath}
    ma l'integranda non è mai integrabile perché in un intorno di $q_*$
    \begin{displaymath}
        E-U(q)=E-\cancel{U(q_*)}-\cancel{U'(q_*)}(q-q_*)+o((q-q_*)^2) \implies\frac{1}{\sqrt{\frac{2}{m}(E-U(q))}}\sim\frac{1}{|q-q_*|^{\alpha}}, \quad \alpha \geq 1
    \end{displaymath}
    che è una funzione non integrabile in un intorno di $q_*$.
\end{proof}

\begin{prop}
    La pendenza delle separatrici nel punto di equilibrio non è verticale.
\end{prop}

\section{Stabilità alla Ljapunov}

L'obiettivo è quello di quantificare il diverso comportamento dei punti di equilibrio.

\begin{defi}
    Un \textbf{sistema dinamico} è un problema di Cauchy della forma:
    $\begin{cases}
        \dot{\overline{x}}=\overline{f}(\overline{x},\dot{\overline{x}},t) \\ \overline{x}(0)=\overline{x}_0    
    \end{cases}$
    con $\overline{x}\in\mathbb{R}^n, \overline{f} \in C^{\infty}, \overline{f}: \mathbb{R}^n \to \mathbb{R}^n$.
\end{defi}

\begin{oss}
    Un sistema olonomo conservativo con matrice di massa $A$ (indipendente da $\overline{q}$) è un sistema con $\overline{x}=(\overline{q},\overline{v})$, $\overline{f}=(A^{-1}\overline{v},-\nabla U)$.
\end{oss}

\begin{defi}
    Una configurazione $\overline{x}_*$ è di \textbf{equilibrio} se $\overline{f}(\overline{x}_0,\overline{0},t)=\overline{0}$.
\end{defi}

\begin{defi}[Stabilità alla Ljapunov]
\everymath{\displaystyle}
    Data una configurazione di equilibrio $\overline{x}_*$ di un sistema dinamico, essa è detta
    \begin{itemize}
        \item \textbf{stabile} se $\forall \varepsilon>0, \exists\delta>0 : |\overline{x}(0)-\overline{x}_*|<\delta \implies |\overline{x}(t)-\overline{x}_*|<\varepsilon, \forall t\in \mathbb{R}^+$;
        \item \textbf{asintoticamente stabile} se stabile e $\lim_{t\to +\infty} \overline{x}(t)=\overline{x}_*$;
        \item \textbf{instabile} se non è stabile.
    \end{itemize}
\end{defi}

\begin{oss}
    I punti di equilibrio per un sistema olonomo conservativo sono tutti della forma $\overline{x}_*=(\overline{q}_*, \overline{0}$).
\end{oss}

\begin{oss}
    Una configurazione di equilibrio è instabile se $\exists \varepsilon_0>0$ t.c. $\forall\delta>0, \exists\overline{x}(0) \in \mathbb{R}^n$ e $t_0 \in \mathbb{R}^+$ t.c. $|\overline{x}(0)-\overline{x}_*|<\delta$ ma $|\overline{x}(t_0)-\overline{x}_*|>\delta_0$.
\end{oss}

\begin{oss}
    Nel piano delle fasi le stabilità secondo Ljapunov di un punto di equilibrio si interpreta nel modo seguente: fissato il raggio $\varepsilon>0$ del disco $D_{\varepsilon}$ con centro $q_*, \;\exists$ sempre un disco $D_{\delta}$ centrato in $q*$ e raggio $\delta>0$ t.c. $\forall$ coppia di dati iniziali $(q(0),\dot{q}(0))\in D_{\delta} \implies (q(t),\dot{q}(t))\in D_{\varepsilon} \forall t \in \mathbb{R}^+$, cioè l'orbita non esce mai dal disco $D_{\varepsilon}$.
\end{oss}

\noindent Un modo per ricavare informazioni sulla stabilità del punto di equilibrio è studiare il comportamento del sistema dinamico linearizzato.

\begin{defi}
    Il \textbf{sistema linearizzato} attorno al punto di equilibrio $\overline{x}_*$ del sistema dinamico è il problema di Cauchy lineare
    $\begin{cases}
        \dot{\overline{x}}=A(\overline{x}-{\overline{x_*}}) \\
        \overline{x}(0)=\overline{x}_0    
    \end{cases}$
    dove $A_{ij}=\frac{\partial f_i}{\partial x_j}(\overline{x}_*)$.
\end{defi}

\begin{teo}[Criterio linearizzato]
    Sia $(L)$ il sistema linearizzato del sistema dinamico $(D)$ e siano $\lambda_j, j\in\{1,\ldots,n\}$, gli autovalori di $A$. Se $\exists \; c>0$ t.c.
    \begin{displaymath}
    \boxed{
        \Re(\lambda_j) \leq -c <0 \quad \forall j\in\{1,\ldots,n\}
        }
    \end{displaymath}
    allora $\overline{x}_*$ è un punto di equilibrio asintoticamente stabile e $\exists\;\mathcal{B}(\overline{x}_*$ intorno di $\overline{x}_*$ e $C<\infty$ t.c. $\forall \overline{x}_0\in\mathcal{B}(\overline{x}_*)$
    \begin{displaymath}
        \boxed{
        |\overline{x}(t)-\overline{x}_*|\leq Ce^{-\frac{ct}{2}} |\overline{x}_0-\overline{x}_*|
        }
    \end{displaymath}
\end{teo}

\begin{lemma}[di Gronwall]
    Sia $g:\mathbb{R}\to\mathbb{R}^+$ una funzione $C^1$ non negativa. Se $\exists\;k\in\mathbb{R}$ t.c. $\Dot{g}(t)\leq kg(t)\;\forall t\in [0,T]$ allora $g(t)\leq g(0)e^{kt}$.
\end{lemma}

\begin{oss}
    Il criterio linearizzato fornisce una condizione sufficiente per la stabilità di un punto di equilibrio di un sistema dinamico.
\end{oss}

\begin{cor}
    Se $\exists \; \lambda$ autovalore di A di $(L)$  t.c. $\Re(\lambda)>0$, allora $\overline{x}_*$ è un punto di equilibrio instabile.
\end{cor}

\noindent Il criterio linearizzato non funziona sempre (per esempio se $\Re(\lambda)=0$). C'è quindi necessità di un criterio che vada oltre il linearizzato.

\begin{teo}[Ljapunov]
    Sia $\overline{x}_*$ un punto di equilibrio per il sistema dinamico $(D)$. Se esiste una funzione $W:\mathbb{R}^n\to\mathbb{R}$ (detta \textbf{funzione di Ljapunov}) definita in un intorno $B(\overline{x}_*)$ di $\overline{x}_*$ e di classe $C^1$ t.c.
    \begin{itemize}
        \item $W(\overline{x}_*) = 0, W(\overline{x})>0, \forall \overline{x} \in B(\overline{x}_*)\backslash\{\overline{x}_*\}$;
        \item $\dot{W}(\overline{x})\leq 0, \forall \overline{x} \in B(\overline{x}_*);$
    \end{itemize}
    allora $\overline{x}_*$ è un punto di equilibrio stabile.
\end{teo}

\begin{proof}
    Consideriamo una palla $\mathcal{B}_{\varepsilon}(\overline{x}_*)$ di raggio $\varepsilon$ centrata in $\overline{x}_*$ con $\varepsilon$ tale che $\mathcal{B}_{\varepsilon}(\overline{x}_*)\subset\mathcal{B}(\overline{x}_*)$ e poniamo $\alpha(\varepsilon):=\min_{\overline{x}\in\partial\mathcal{B}_{\varepsilon}(\overline{x}_*)}W(\overline{x})>0$ per la prima ipotesi fatta su $W$. Sia inoltre $U_{\varepsilon}:=\{\overline{x}\in\mathcal{B}_{\varepsilon}(\overline{x}_*)\;|\;W(\overline{x})<\frac{\alpha(\varepsilon)}{2}\}\implies$ poiché $W$ è continua, $U_{\varepsilon}$ è un aperto contenente $\overline{x}_*$ per ipotesi e quindi $\exists \delta_{\varepsilon}>0 \;(\delta_{\varepsilon}<\varepsilon)$ tale che $\mathcal{B}_{\delta_{\varepsilon}}(\overline{x}_*)\subset U_{\varepsilon}$.

    \noindent Vogliamo mostrare che $\forall \overline{x}_0\in\mathcal{B}_{\delta_{\varepsilon}}(\overline{x}_*) \;\overline{x}(t)\in\mathcal{B}_{\varepsilon}(\overline{x}_*), \forall t\in\mathbb{R}^+$, il che non è altro che la definizione di stabilità. Supponiamo per assurdo che $\exists \bar{t}>0$ t.c. fissato $\overline{x}_0\in\mathcal{B}_{\delta_{\varepsilon}}(\overline{x}_*)$ si abbia $\overline{x}(\Bar{t})\in\partial\mathcal{B}_{\varepsilon}(\overline{x}_*)$ allora si avrebbe
    \begin{displaymath}
        0<\frac{\alpha(\varepsilon)}{2}<W(\overline{x}(\Bar{t}))-W(\overline{x}_0)=\int_0^{\Bar{t}}\frac{dW}{d\tau}(\overline{x}(\tau))\,d\tau\leq 0
    \end{displaymath}
    il che è assurdo.
\end{proof}

\begin{oss}
    Se inoltre $\dot{W}(\overline{x})<0, \; \forall \overline{x}\in B(\overline{x}_*)\backslash\{\overline{x}_*\}$ (sotto questa ipotesi si ha automaticamente $\dot{W}(\overline{x}_*)=0$) allora $\overline{x}_*$ è un punto di equilibrio asintoticamente stabile.
\end{oss}

\begin{cor}
    Se, sotto le stesse ipotesi, $\exists$ una funzione  $W:\mathbb{R}^n\to\mathbb{R}$ definita in un intorno $B(\overline{x}_*)$ di $\overline{x}_*$ e di classe $C^1$ t.c.
    \begin{itemize}
        \item $W(\overline{x}_*)=0$ e $W(\overline{x})>0, \forall \overline{x} \in B(\overline{x}_*)\backslash\{\overline{x}_*\}$;
        \item $\dot{W}(\overline{x})\ge 0, \forall \overline{x} \in B(\overline{x}_*)$ e $\dot{W}(\overline{x})>0, \forall \overline{x} \in B(\overline{x}_*)\backslash\{\overline{x}_*\}$;
    \end{itemize}
    allora $\overline{x}_*$ è un punto di equilibrio instabile.
\end{cor}

\begin{oss}
    Si ha $\dot{W}(\overline{x})=\dot{\overline{x}}\cdot \nabla W = \overline{f}(\overline{x};\dot{\overline{x}};t)\cdot \nabla W(\overline{x})$.
\end{oss}

\noindent Torniamo a considerare sistemi olonomi conservativi: $g$ gradi di libertà, energia conservata $E=\frac{1}{2}\dot{\overline{q}}A\left(\overline{q}\right)\dot{\overline{q}}+U\left(\overline{q}\right), \overline{q} \in \mathbb{R}^g$.

\begin{teo}[Dirichlet]
    I punti di minimo isolato di $U$ sono punti di equilibrio stabili del sistema.
\end{teo}

\begin{proof}
    È sufficiente osservare che la funzione $\widetilde{W}(\overline{q};\Dot{\overline{q}})=\frac{1}{2}\dot{\overline{q}}A\left(\overline{q}\right)\dot{\overline{q}}+U\left(\overline{q}\right)-U\left(\overline{q}_*\right)$ è una funzione di Ljapunov per il sistema dinamico
    \begin{displaymath}
        \begin{dcases}
            \Dot{\overline{x}}=\overline{f}(\overline{x};\Dot{\overline{x}};t) \\
            \overline{x}(0)=\overline{x}_0
        \end{dcases}
        \quad\text{ dove } \overline{x}=(\overline{q},\overline{p}), \; \overline{f}=(A^{-1}(\overline{q})\overline{p}, -\nabla U)
    \end{displaymath}
    attorno al punto di equilibrio $\overline{x}_*=(\overline{q}_*,\overline{0})$ con $\overline{q}_*$ punto di minimo isolato di $U(\overline{q})$.
    Infatti
    \begin{itemize}
        \item $W(\overline{x})=\widetilde{W}(\overline{q};A^{-1}(\overline{q})\overline{p})=\frac{1}{2}\overline{p}A^{-1}\left(\overline{q}\right)\overline{p}+U\left(\overline{q}\right)-U\left(\overline{q}_*\right)$ è definita in un intorno di $\overline{x}_*$ ed è di classe $C^\infty$;
        \item $W(\overline{x}_*)=0$ e $W(\overline{x})\geq U\left(\overline{q}\right)-U\left(\overline{q}_*\right)>0, \forall\overline{x}\in\mathcal{B}_{\delta}(\overline{x}_*)-\{\overline{x}_*\}$;
        \item $W(\overline{x})=0, \forall \overline{x}\in\mathbb{R}^{2g}$.
    \end{itemize}
    Quindi $\overline{x}_*$ è un punto di equilbrio stabile.
\end{proof}

\begin{oss}
    La condizione che $\overline{q}_*$ sia un minimo isolato di $U$ è sufficiente ma non necessaria per la stabilità di $\overline{q}_*$.
\end{oss}

\begin{prop}
    Dato un sistema olonomo conservativo unidimensionale di massa m, se q* è un punto $m$ e potenziale $U$, sia $q_*$ un punto di massimo di $U$ con $U^{\prime\prime}(q*)<0$, allora $q_*$ è un punto di equilibrio instabile.
\end{prop}

\chapter{Meccanica Lagrangiana}

\section{Equazioni di Lagrange}

\begin{defi}
    Dato un sistema olonomo sottoposto a forze attive di potenziale $U(\overline{q}; t)$, la \textbf{Lagrangiana} o \textbf{funzione Lagrangiana} è
    \begin{displaymath}
    \boxed{
        \mathcal{L}(\overline{q};\dot{\overline{q}};t):=T(\overline{q};\dot{\overline{q}};t) - U(\overline{q};t) = \frac{1}{2}\dot{\overline{q}}A(\overline{q};t)\dot{\overline{q}} + \overline{b}(\overline{q};t)\cdot\dot{\overline{q}} + \overline{c}(\overline{q};t)-U(\overline{q};t)
        }
    \end{displaymath}
\end{defi}

\begin{teo}[Equazioni di Eulero-Lagrange]
    Dato un sistema olonomo sottoposto a forze attive di potenziale $U(\overline{q}; t)$, $\overline{q}(t):[t_0,t_1]\to\mathbb{R}^g$ è un moto del sistema se e solo se sono soddisfatte le \textbf{equazioni di Lagrange}
    \begin{displaymath}
    \boxed{
        \frac{d}{dt}\frac{\partial\mathcal{L}}{\partial\dot{q}_k} - \frac{\partial\mathcal{L}}{\partial q_k} = 0, \quad \forall k \in \{1,\ldots,g\}, \forall t \in (t_0,t_1)
        }
    \end{displaymath}
\end{teo}

\begin{lemma}
\everymath{\displaystyle}
    Ponendo $\tau_k:=\sum_{j=1}^N m_j\overline{a}_j \cdot \frac{\partial \overline{x}_j}{\partial q_k} $ (\textbf{componenti generalizzate delle forze inerziali}) si ha 
    $\frac{d}{dt}\frac{\partial T}{\partial\dot{q}_k} - \frac{\partial T}{\partial q_k} = \tau_k$ lungo ogni moto del sistema.
\end{lemma}

\begin{proof}[Dimostrazione (Teorema)]
    \begin{displaymath}
        \frac{d}{dt}\frac{\partial\mathcal{L}}{\partial\Dot{q}_k}=\frac{d}{dt}\frac{\partial T}{\partial\Dot{q}_k}=\frac{\partial T}{\partial q_k}+\tau_k=\tau_k+\frac{\partial\mathcal{L}}{\partial q_k}+\frac{\partial U}{\partial q_k}=\frac{\partial\mathcal{L}}{\partial q_k}+\tau_k-Q_k
    \end{displaymath}
    Se quindi dimostriamo che lungo ogni moto del sistema $\tau_k=Q_k$ abbiamo ottenuto il risultato. Ma
    \begin{displaymath}
        \tau_k=\sum_{j=1}^N m_j\overline{a}_j \cdot \frac{\partial \overline{x}_j}{\partial q_k}, \quad Q_k=\sum_{j=1}^N\overline{F}_j\cdot \frac{\partial \overline{x}_j}{\partial q_k} \implies \tau_k-Q_k=\sum_{j=1}^N\left(m_j\overline{a}_j-\overline{F}_j\right) \cdot \frac{\partial \overline{x}_j}{\partial q_k}=0
    \end{displaymath}
\end{proof}

\begin{oss}
    Le equazioni di Lagrange sono equazioni pure della dinamica.
\end{oss}

\begin{oss}
    Se le forze attive non sono conservative le equazioni di Lagrange prendono la forma
    \begin{displaymath}
        \frac{d}{dt}\frac{\partial T}{\partial\dot{q}_k} - \frac{\partial T}{\partial q_k} = Q_k, \quad \forall k \in \{1,\ldots,g\}
    \end{displaymath}
    con $Q_k$ le componenti generalizzate delle forze.
\end{oss}

\begin{prop}[Conservazione dell'energia]
    Se $\mathcal{L}(\overline{q};\dot{\overline{q}})=T(\overline{q};\dot{\overline{q}})-U(\overline{q})$ non dipende esplicitamente da $t\in\mathbb{R}$, allora
    \begin{displaymath}
        \boxed{
        E=T(\overline{q};\dot{\overline{q}})+U(\overline{q})
        }
    \end{displaymath}
    è un integrale primo del moto.
\end{prop}

\begin{proof}
    $\Dot{E}=\frac{d\mathcal{L}}{dt}+2\frac{dU}{dt}$, ma
    \begin{displaymath}
        \frac{d\mathcal{L}}{dt}=\sum_k\left\{\frac{\partial\mathcal{L}}{\partial q_k}\Dot{q}_k+\frac{\partial\mathcal{L}}{\partial\Dot{q}_k}\Ddot{q}_k\right\}=\sum_k\left\{\frac{d}{dt}\left(\frac{\partial\mathcal{L}}{\partial\Dot{q}_k}\right)\Dot{q}_k+\frac{\partial\mathcal{L}}{\partial\Dot{q}_k}\Ddot{q}_k\right\}=\sum_k\frac{d}{dt}\left(\frac{\partial\mathcal{L}}{\partial\Dot{q}_k}\right)=2\Dot{T}
    \end{displaymath}
    poiché $T$ è quadratica nelle $\Dot{q}_k\implies\Dot{E}=2\Dot{E}\implies\Dot{E}=0$.
\end{proof}

\section{Simmetrie}

\begin{defi}
\everymath{\displaystyle}
    Diciamo che una coordinata $q_k, k \in \{1,\ldots,g\}$, è \textbf{ciclica} per la Lagrangiana $\mathcal{L}(\overline{q};\dot{\overline{q}},t)$ se $\mathcal{L}$ non dipende da $q_k$ ovvero $\frac{\partial\mathcal{L}}{\partial q_k}=0$.
\end{defi}

\begin{prop}
\everymath{\displaystyle}
    Sia $q_k$ una coordinata ciclica per la Lagrangiana $\mathcal{L}(\overline{q};\dot{\overline{q}},t)$, allora il \textbf{momento coniugato} $p_k=\frac{\partial \mathcal{L}}{\partial \dot{q}_k}$ è un integrale primo del moto, cioè $\dot{p}_k=0$.
\end{prop}

\begin{proof}
\everymath{\displaystyle}
    $0=\frac{\partial\mathcal{L}}{\partial q_{k_*}}=\frac{d}{dt}\frac{\partial\mathcal{L}}{\partial\Dot{q}_k}$.
\end{proof}

\begin{defi}
    Diciamo che una trasformazione $\overline{\Phi}:\mathbb{R}^g\to \mathbb{R}^g \; C^{\infty}$ invertibile è una \textbf{simmetria} del sistema se $\mathcal{L}(\overline{q};\dot{\overline{q}};t)=\mathcal{L}(\overline{q}^{\prime};\dot{\overline{q}'};t)$ dove $\overline{q}'=\overline{\Phi}(\overline{q})$.
\end{defi}

\begin{defi}
    Una famiglia di diffeomorfismi $\left\{\overline{\Phi}_{\varepsilon}\right\}_{\varepsilon\in\mathbb{R}}:\mathbb{R}^g\to\mathbb{R}^g$ (trasformazioni $C^\infty$ invertibili) è un \textbf{gruppo ad un parametro di simmetrie} del sistema se è derivabile in $\varepsilon$ e
    \begin{itemize}
        \item $\overline{\Phi}_0(\overline{q})=\overline{q}, \forall \overline{q}\in\mathbb{R}^g$;
        \item $\overline{\Phi}_{\varepsilon_1}\left(\overline{\Phi}_{\varepsilon_2}(\overline{q})\right)=\overline{\Phi}_{\varepsilon_1+\varepsilon_2}(\overline{q}), \forall \varepsilon_1, \varepsilon_2\in\mathbb{R}$.
    \end{itemize}
\end{defi}

\begin{teo}[Noether]
    Se un sistema di Lagrangiana $\mathcal{L}(\overline{q};\dot{\overline{q}})$ ammette una famiglia ad 1 parametro di simmetrie $\left\{\overline{\Phi}_{\varepsilon}\right\}_{\varepsilon\in\mathbb{R}}$ allora le quantità seguenti sono integrali primi del moto:
    \begin{displaymath}
    \boxed{
        I(\overline{q};\dot{\overline{q}})=\nabla\dot{\overline{q}}\mathcal{L}\cdot\frac{\partial\Phi_{\varepsilon}}{\partial\varepsilon}\bigg|_{\varepsilon=0}=\sum_{k=1}^g\frac{\partial\mathcal{L}}{\partial\dot{q}_k}\cdot \frac{\partial(\Phi_{\varepsilon})_k}{\partial\varepsilon}\bigg|_{\varepsilon=0}
        }
    \end{displaymath}
\end{teo}

\begin{proof}
    L'invarianza di $\mathcal{L}$ garantisce che se $\overline{q}(t)$ è soluzione delle equazioni di Lagrange allora anche $\overline{\Phi}_\varepsilon(\overline{q}(t))$ lo è $\forall\;\varepsilon\in\mathbb{R}$, ovvero
    \begin{displaymath}
        \frac{d}{dt}\left(\frac{\partial\mathcal{L}}{\partial\Dot{q}_k'}(\overline{q}';\dot{\overline{q}'})\right)-\frac{\partial\mathcal{L}}{\partial q_k'}(\overline{q}';\dot{\overline{q}'})=0\quad\text{dove } \overline{q}'=\overline{\Phi}_\varepsilon(\overline{q})
    \end{displaymath}
    ma per definizione di simmetria e invarianza di $\mathcal{L}$
    \begin{displaymath}
        0=\frac{\partial}{\partial\varepsilon}\mathcal{L}(\overline{q}';\dot{\overline{q}'})=\frac{\partial\mathcal{L}}{\partial\Dot{q}_k'}\cdot\frac{\partial\Dot{q}_k'}{\partial\varepsilon}+\frac{\partial\mathcal{L}}{\partial q_k'}\cdot\frac{\partial q_k'}{\partial\varepsilon}
    \end{displaymath}
    da cui otteniamo moltiplicando per $\frac{\partial q_k}{\partial\varepsilon}$:
    \[
    \begin{split}
        0&=\frac{d}{dt}\left(\frac{\partial\mathcal{L}}{\partial\Dot{q}_k'}\right)\cdot\frac{\partial q_k}{\partial\varepsilon}-\frac{\partial\mathcal{L}}{\partial q_k'}\cdot\frac{\partial q_k'}{\partial\varepsilon}=\frac{d}{dt}\left(\frac{\partial\mathcal{L}}{\partial\Dot{q}_k'}\right)\cdot\frac{\partial q_k}{\partial\varepsilon}+\frac{\partial\mathcal{L}}{\partial\Dot{q}_k'}\cdot\frac{\partial\Dot{q}_k'}{\partial\varepsilon} \\
        &=\frac{d}{dt}\left(\frac{\partial\mathcal{L}}{\partial\Dot{q}_k'}\cdot\frac{\partial q_k'}{\partial\varepsilon}\right)\left(\overset{\varepsilon=0}{=}\Dot{I}\right)
    \end{split}
    \]
\end{proof}

\begin{oss}
    Esistono integrali primi del moto che non derivano dalla presenza di simmetrie, cioè non tutti gli integrali primi sono momenti conservati. Un esempio banale è dato dalla energia meccanica $E$ che non è il momento coniugato di nessuna coordinata.
\end{oss}

\section{Moti Centrali}

Consideriamo un sistema di due corpi di masse $m_1$ e $m_2$ in mutua interazione con una forza del tipo
\begin{displaymath}
    \overline{F}_{ij}=f(|\overline{x}_1-\overline{x}_2|)\,\hat{x}_{ij}
\end{displaymath}
con $\hat{x}_{ij}=\frac{\overline{x}_i-\overline{x}_j}{|\overline{x}_i-\overline{x}_j|}$, $i,j=1,2$, $\overline{x}_j\in\mathbb{R}^3$ e $f$ una funzione $C^\infty(\mathbb{R}^+)$, cioè il coefficiente della forza $f$ dipende solo dalla distanza tra i corpi del sistema.

\begin{prop}
\everymath{\displaystyle}
    Nelle coordinate $\overline{X}:=\frac{m_1\overline{x}_1+m_2\overline{x}_2}{m_1+m_2}$ e $\overline{r}:=\overline{x}_1-\overline{x}_2$, la Lagrangiana del sistema è
    \begin{displaymath}
        \mathcal{L}(\overline{X};\dot{\overline{X}};\overline{r};\dot{\overline{r}})=\frac{1}{2}(m_1+m_2)|\dot{\overline{X}}|^2+\frac{1}{2}\mu|\dot{\overline{r}}|^2 -U(|\overline{r}|)
    \end{displaymath}
    dove $\mu=(\frac{1}{m_1}+\frac{1}{m_2})^{-1}$ è la massa ridotta e $U$ una primitiva di $-f$ (cioè $U'=-f$).
\end{prop}

\begin{oss}
    La Lagrangiana è ciclica rispetto a $\overline{X} \implies \overline{P}_X:=(m_1+m_2)\dot{\overline{X}}$ (momento totale) è un integrale primo del moto. Possiamo quindi considerare un osservatore $\mathcal{O}'$ che è solidale al centro di massa (osservatore inerziale) e rispetto a tale osservatore
    \begin{displaymath}
    \boxed{
        \mathcal{L}=\mathcal{L}(\overline{r};\dot{\overline{r}})=\frac{1}{2}\mu|\dot{\overline{r}}|^2 - U(|\overline{r}|)
        }
    \end{displaymath}
    che è quindi la Lagrangiana che descrive il moto in un campo centrale.
\end{oss}

\begin{oss}
    $\mathcal{L}$ non dipende esplicitamente dal tempo $\implies E=\frac{1}{2}\mu|\dot{\overline{r}}|^2 + U(|\overline{r}|)$ è integrale primo.
\end{oss}

\begin{prop}
     Il momento angolare $\overline{L}:=m\overline{r}\wedge\Dot{\overline{r}}$ è un integrale primo (vettoriale = 3 scalari) del moto.
\end{prop}

\begin{proof}
    Fissato un qualunque asse $r$, le rotazioni rispetto ad esso formano un gruppo ad un parametro di simmetrie di $\mathcal{L}\implies$ il momento angolare lungo $r$ è conservato per Noether. Prendendo i 3 assi di un sistema di riferimento otteniamo la conservazione delle 3 componenti di $\overline{L}$ e dunque di $\overline{L}$ stesso.
\end{proof}

\begin{prop}
    Il moto per un sistema in un campo centrale è:
    \begin{itemize}
        \item rettilineo se $\overline{L}(0)=\overline{0}$;
        \item piano se $\overline{L}(0)\neq \overline{0}$ (nel piano $\pi \perp \overline{L}(0)$).
    \end{itemize}
\end{prop}

\begin{proof}
    Poiché $\overline{L}$ è un integrale primo del moto, il suo valore al tempo iniziale è conservato:
    \begin{enumerate}
        \item $\overline{L}(0)=\overline{0}\implies\overline{r}(0)\wedge\Dot{\overline{r}}(0)=\overline{0}\implies\overline{r}(0)\parallel\Dot{\overline{r}}(0)$ ovvero $\exists \hat{k}\in\mathbb{R}^3$ t.c. $\overline{r}(0)=r_0\hat{k}, \;\Dot{\overline{r}}(0)=v_0\hat{k}$, con $r_0, v_0 \in \mathbb{R}$ (possono anche essere nulli) ma allora le equazioni del moto ammettono una soluzione del tipo $\overline{r}(t)=r(t)\hat{k}$ infatti, poiché $\overline{L}$ è conservato, deve anche essere $\overline{r}(t)\parallel\Dot{\overline{r}}(t)$ ovvero $\exists \hat{k}\in\mathbb{R}^3$ e $a(t)$ e $b(t) \in \mathbb{R}$ t.c. $\overline{r}(t)=a(t)\hat{k}(t),\; \Dot{\overline{r}}(t)=b(t)\hat{k}$ e $\hat{k}(0)=\hat{k},\; a(0)=a, \;b(0)=b$. Ma $\Dot{\overline{r}}(t)=\Dot{a}(t)\hat{k}(t)+a(t)\Dot{\hat{k}}$ e $\Dot{\hat{k}}(t)\cdot\hat{k}(t)=0$ se $\hat{k}$ è un versore per cui $\Dot{\hat{k}}=\overline{0}$ oppure $\Dot{a}(t)=b(t)$, che implica ancora $\hat{k}(t)=0$.
        \item $\overline{L}$ è conservato ma $\overline{r}, \Dot{\overline{r}}\perp\overline{L}\;\forall t\implies$ il moto si svolge nel piano $\perp\overline{L}$.
    \end{enumerate}
\end{proof}

\begin{oss}
    Le equazioni del moto sono: $\mu\Ddot{\overline{r}}=-U'(|\overline{r}|)\hat{r}$.
\end{oss}

\begin{teo}[II legge di Keplero]
    In un campo centrale la velocità areolare $\Dot{A}(t)$ è costante.
\end{teo}

\begin{proof}
    Conviene anzitutto fare un cambio di coordinate e utilizzare le coordinate polari nel piano $\perp\overline{L}$ assumendo $\overline{L}\neq\overline{0}$ (se $\overline{L}=\overline{0}$ il moto si svolge lungo una retta e la velocità areolare è nulla $\forall t\in\mathbb{R}$), quindi $\overline{r}\to(r,\vartheta,z)\in\mathbb{R}^+\times[0,2\pi)\times\mathbb{R}$. In coordinate polari si ha
    \begin{displaymath}
        A(t)=\frac{1}{2}\int_{\theta(0)}^{\theta(t)}r^2(\theta)\,d\theta=\frac{1}{2}\int_0^tr^2(t)\Dot{\theta}(t)\,d\tau\implies\Dot{A}(t)=\frac{1}{2}r^2(t)\Dot{\theta}(t)
    \end{displaymath}
    ma
    \begin{displaymath}
        L_3=\mu(\overline{r}\wedge\Dot{\overline{r}})_3=\mu(x\Dot{y}-\Dot{x}y)=\mu\left\{\rho\cos{\theta}(\cancel{\Dot{\rho}\sin{\theta}}-\rho\Dot{\theta}\cos{\theta})-\rho\sin{\theta}(\cancel{\Dot{\rho}\cos{\theta}}-\rho\Dot{\theta}\sin{\theta})\right\}=\mu\rho^2\Dot{\theta}
    \end{displaymath}
    cioè $L_3=\mu\rho^2\Dot{\theta}=2\mu\Dot{A}=cost$ per la conservazione del momento angolare.
\end{proof}

\begin{defi}
    La \textbf{velocità areolare} $\Dot{A}(t)$ è la derivata rispetto al tempo della funzione $A(t)$ che dà l'area coperta dalla traiettoria al variare di $t$. 
\end{defi}

\begin{oss}
    In coordinate polari $\mathcal{L}(\rho;\theta;\dot{\rho};\dot{\theta}) = \frac{1}{2}\mu (\dot{\rho}^2+\rho^2 \dot{\theta}^2) -U(\rho)$.
\end{oss}

\begin{prop}
\everymath{\displaystyle}
    Ogni moto centrale con $L\neq0$ si decompone in coordinate polari in 2 moti indipendenti unidimensionali:
    \begin{itemize}
        \item $\rho(t)$ è soluzione di un moto in $\mathbb{R}^+$ con energia conservata $E=\frac{1}{2}\mu\Dot{\rho}^2 +U_{eff}(\rho)$ dove il \textbf{potenziale efficace} $U_{eff}$ è dato da
        \begin{displaymath}
             U_{eff}(\rho)=\frac{L^2}{2\mu\rho^2}+U(\rho)
        \end{displaymath}
        dove $L=L_3$, cioè $\overline{L}=L\hat{k}$;
        \item $\theta(t)$ risolve $\Dot{\theta}(t)=\frac{L}{\mu\rho^2(t)}$, finché $\rho(t)>0$;
        \item l'equazione della traiettoria nel piano $\perp \overline{L}$ è soluzione di $\frac{d\rho}{d\theta}=\pm \frac{\mu\rho^2}{L_3(0)}\sqrt{\frac{2}{\mu}(E-U_{eff})}$.
    \end{itemize}
\end{prop}

\begin{oss}
    Se $E\in\Lambda_C(U_{eff})$ (non asintoto orizzontale), allora un moto radiale è $\rho(t)=\rho_*$ punto di equilibrio radiale e il moto complessivo è circolare uniforme $\theta(t)=\theta_0+\omega t$ con $\omega=\frac{L}{\mu\rho_*^2}$.
\end{oss}

\begin{oss}
    Le orbite del moto radiale possono essere per $E\notin\Lambda_C(U_{eff})$
    \begin{itemize}
        \item chiuse, limitate, periodiche;
        \item aperte e illimitate.
    \end{itemize}
\end{oss}

\begin{prop}
    Se $E\notin\Lambda_C(U_{eff})$ e $\limsup_{\rho\to^+}\rho^2U(\rho)=0 \implies \forall L \neq 0 \;\operatorname{supp}\gamma_{E, \rho}$ (orbita radiale) $\subset [\rho_m,+\infty)$ con $\rho_m$ più piccola radice $\in \mathbb{R}^+$ di $E-U_{eff}(\rho)$.
\end{prop}

\begin{prop}
     Un moto centrale è periodico se e solo se
     \begin{itemize}
         \item  il moto radiale è periodico di periodo $T$ oppure di quiete;
         \item la variazione di $\theta$ in un periodo del moto soddisfa
         \begin{displaymath}
             \Delta\theta := \theta(T)-\theta(0) = \int_0^T\frac{L^2}{\mu\rho^2(\tau)}d\tau=\frac{p}{q}2\pi, \quad p,q\in\mathbb{Z}^+
         \end{displaymath}
         e in tal caso il periodo complessivo è $nT$ con $n$ il più piccolo $q\in\mathbb{Z}$ t.c. la condizione è soddisfatta.
     \end{itemize}
\end{prop}

\begin{proof}
    Il moto complessivo è periodico $\iff \exists\Bar{T}>0$ t.c. $\rho(t+\Bar{T})=\rho(t)$ e $\theta(t+\Bar{T})=\theta(t)+2\pi\Bar{k}, \Bar{k}\in\mathbb{Z}, \forall t\in\mathbb{R}$. La prima condizione è verificata per ogni moto periodico o di quiete per $\Bar{T}=qT, \forall q\in\mathbb{Z}_0^+$. La seconda condizione coincide con ($*$):
    \[
    \begin{split}
        \theta(t+\Bar{T})&=\theta_0+\int_0^{t+\Bar{T}} \frac{L}{\mu\rho^2}\,d\tau=\theta_0+\int_0^t \frac{L}{\mu\rho^2}\,d\tau+\int_t^{t+qT} \frac{L}{\mu\rho^2}\,d\tau \\
        &=\underbrace{\theta_0+\int_0^t \frac{L}{\mu\rho^2}\,d\tau}_{=\theta(t)}+\underbrace{\int_0^{qT} \frac{L}{\mu\rho^2}\,d\tau}_{\text{per periodicità di $\rho$}} =\theta(t)+q\Delta\theta=\theta(t)+2\pi\Bar{k} \iff (*)
    \end{split}
    \]
\end{proof}

\begin{oss}
     In generale ogni orbita radiale che non corrisponde a valori critici dell'energia è periodica ma la condizione non è soddisfatta, allora nel quel caso l'orbita è \underline{limitata}, ma \underline{aperta}.
\end{oss}

\subsection{Campo Gravitazionale}

Nel campo gravitazionale (Coulombiano attrattivo):
\begin{itemize}
\everymath{\displaystyle}
    \item $U(\rho)=-\frac{k}{\rho}, k>0$;
    \item $U_{eff}(\rho)=\frac{L^2}{2m\rho^2}-\frac{k}{\rho}$;
    \item $\lim_{\rho\to0^+}U_{eff}(\rho)=+\infty$;
    \item $\lim_{\rho\to+\infty}U_{eff}(\rho)=0^-$;
    \item $\min U_{eff}(\rho)=U_{eff}(\rho_*)=-\frac{k^2m}{2L^2}=E_*$;
    \item $\rho_*=\frac{L^2}{km}$ punto di minimo assoluto;
    \item i valori ammissibili dell'energia sono $E\in[E_*,+\infty)$ e quelli critici $\Lambda_C$=\{0\} (asintoto orizzontale). 
\end{itemize}

\begin{oss}
    Si noti che la descrizione in coordinate polari di qualunque moto centrale può essere problematica se il moto raggiunge l'origine in un tempo finito (caduta nel centro), perché da quel momento in poi le equazioni del moto perdono senso (singolarità polare).
\end{oss}

\begin{defi}
\everymath{\displaystyle}
    La quantità positiva $e:=\sqrt{1-\frac{E}{E_*}} \in \mathbb{R}^+$ è detta \textbf{eccentricità} dell'orbita.
\end{defi}

\begin{oss}
    Si noti che poiché $E\geq E_*$, l'eccentricità $e$ è ben definita.
\end{oss}

\noindent Discutiamo ora il comportamento dell'orbita a seconda dei valori dell'energia $E$:
\begin{itemize}
\everymath{\displaystyle}
    \item $E=E_*$: l'orbita radiale corrisponde al solo punto di equilibrio stabile $\rho(t)=\rho_*$;
    \item $E_*<E<0$: l'orbita radiale è limitata, chiusa e periodica; si svolge fra $\rho_-$ e $\rho_+$ soluzioni di $E=U_{eff}(\rho)$ che sono date da $\rho_{\pm}=\frac{\rho_*}{1\mp e}$;
    \item $E=0$ (separatrice): corrisponde ad un valore critico dell'energia (dato che $0$ è un asintoto orizzontale) e si ottiene un'unica orbita illimitata aperta $[\rho_-, +\infty)$ (si noti che $\rho_-$ è un punto di inversione).
    \item $E>0$: l'orbita radiale è illimitata ed aperta, con un unico punto di inversione in $\rho_-$.
\end{itemize}

\begin{prop}
     L'equazione delle orbite in un campo gravitazionale è
     \begin{displaymath}
         \rho(\theta)=\frac{\rho_*}{1+e\cdot \cos(\theta)}
     \end{displaymath}
     Di conseguenza le orbite sono:
     \begin{itemize}
         \item ellissi se $e<1$ (circonferenze per $e=0$);
         \item parabole se $e=1$ (separatrice);
         \item iperboli se $e>1$.
     \end{itemize}
     In particolare, le orbite limitate ($e<1$) sono tutte periodiche di periodo $T$ uguale a quello del moto radiale.
\end{prop}

\begin{teo}[Leggi di Keplero]
    Per ogni moto in un campo di forze gravitazionali:
    \begin{itemize}
        \item le orbite (limitate) sono ellissi, in cui il centro della forza occupa uno dei due fuochi;
        \item la velocità areolare è costante;
        \item il quadrato del periodo dell'orbita è proporzionale al cubo del semiasse maggiore ($T^2\propto a^3$).
    \end{itemize}
\end{teo}

\begin{oss}
    Nella forma originaria delle leggi di Keplero la prima legge recitava "le orbite dei pianeti del sistema solare sono ellissi di cui il sole occupa uno dei due fuochi".
\end{oss}

\begin{oss}
    Per descrivere le orbite dei pianeti con precisione sarebbe necessario tenere conto dell'influenza di tutti i pianeti del sistema solare: occorre applicare una teoria perturbativa (KAM) per poter approssimare le traiettorie e studiare ad esempio la stabilità.
\end{oss}

\begin{oss}
    Le traiettorie ellittiche hanno uno dei due fuochi in $O$, per cui sono asimmetriche rispetto ad $O\implies T=T_{radiale}$ e $\Delta\theta=2\pi$.
\end{oss}

\begin{oss}
    Nel caso dell'oscillatore armonico, tutte le orbite sono ellissi con il centro in $O$, ovvero sono simmetriche $\implies T=2T_{radiale}$ e $\Delta\theta=\pi$.
\end{oss}

\begin{oss}
    Sia per il campo gravitazionale che per l'oscillatore armonico, tutte le orbite limitate sono anche periodiche (ellissi e circonferenze), ma questo è un fatto molto speciale!
\end{oss}

\begin{teo}[Bertrand]
    In un campo centrale tutte le orbite limitate sono chiuse se e solo se
    \begin{displaymath}
        U(\rho)=
        \begin{dcases}
        \frac{1}{2}k\rho^2 \quad \text{oscillatore armonico} \\
        -\frac{k}{\rho} \quad \; \; \; \text{campo gravitazionale}
    \end{dcases}
    \end{displaymath}
\end{teo}

\begin{oss}
     Per un potenziale centrale generico (che non sia gravitazionale o armonico) l'insieme dei dati iniziali che danno origine a orbite chiuse ha misura di Lebesgue nulla.
\end{oss}

\noindent Cos'hanno di così tanto speciale il campo gravitazionale e l'oscillatore armonico? Sono potenziali che hanno più simmetrie dei campi centrali generici. Una manifestazione della maggiore simmetria del potenziale gravitazionale è data dalla presenza di una quantità conservata indipendente da $\overline{L}$ e $E$.

\begin{prop}
     In ogni moto in un campo gravitazionale il \textbf{vettore di Runge-Lenz}
     \begin{displaymath}
     \boxed{
         \overline{A}= m\left(\dot{\overline{r}}\wedge\overline{L}-k\hat{r}\right)
         }
     \end{displaymath}
     è un integrale primo del moto.
\end{prop}

\begin{proof}
    \[
    \begin{split}
        \frac{\Dot{\overline{A}}}{m}&=\Ddot{\overline{r}}\wedge\overline{L}-\frac{k\Dot{\overline{r}}}{|\overline{r}|}+\frac{k\overline{r}\cdot\Dot{\overline{r}}}{|\overline{r}|^3}\overline{r}=\{\text{usiamo l'equazione del moto $m\Ddot{\overline{r}}=-\frac{k}{|\overline{r}|^2}\hat{r}$}\} \\
        &=-\frac{k}{m|\overline{r}|^2}\hat{r}\wedge\overline{L}-\frac{k\Dot{\overline{r}}}{|\overline{r}|}+\frac{k\overline{r}\cdot\Dot{\overline{r}}}{|\overline{r}|^3}\overline{r}=-\frac{k}{m\rho^2}\hat{e}_{\rho}\wedge L\hat{k}-\cancel{\frac{k\Dot{\rho}}{\rho}\hat{e}_{\rho}}-k\Dot{\theta}\hat{e}_{\theta}+\cancel{\frac{k\Dot{\rho}}{\rho}\hat{e}_{\rho}} \\
        &=\left(\frac{k}{m\rho^2}L-k\Dot{\theta}\right)\hat{e}_{\theta}=\left(k\Dot{\theta}-k\Dot{\theta}\right)\hat{e}_{\theta}=\overline{0}
    \end{split}
    \]
\end{proof}

\begin{oss}
    Si tratta di $3$ quantità indipendenti conservate che però non derivano da una simmetria del tipo Noether.
\end{oss}

\section{Piccole Oscillazioni}

Consideriamo un sistema olonomo conservativo di Lagrangiana $\mathcal{L}(\overline{q};\dot{\overline{q}})=\frac{1}{2}\dot{\overline{q}}A(\overline{q})\Dot{\overline{q}} -U(\overline{q}), \overline{q}\in\mathbb{R}^g$.

\begin{prop}
    Il sistema dinamico linearizzato attorno  ad una posizione di equilibrio stabile $\overline{q}_*$ del sistema olonomo conservativo ha equazioni del moto
    \begin{displaymath}
    \boxed{
        A\Ddot{\overline{u}}= -B\overline{u}
        }
    \end{displaymath}
    dove $A=A(\overline{q}_*)$, $B_{ij}= \frac{\partial^2U}{\partial q_i\partial q_j}(\overline{q}_*)$ (hessiana di $U(\overline{q}_*)$), $\overline{u}=\overline{q}-\overline{q}_*$
    ed è quindi descritto dalla Lagrangiana
    \begin{displaymath}
    \boxed{
        \widetilde{\mathcal{L}}(\overline{u};\dot{\overline{u}})=\frac{1}{2}\Dot{\overline{u}}A\Dot{\overline{u}} -\frac{1}{2}\overline{u}B\overline{u}
        }
    \end{displaymath}
\end{prop}

\begin{proof}
    Il sistema dinamico associato è dato da $\Dot{\overline{x}}=\overline{f}(\overline{x})$ dove $\overline{x}=(\overline{q}, \overline{p}), \overline{f}=\left(A^{-1}(\overline{q})\overline{p}, -\nabla U\right)$ e il punto di equilibrio diventa $\overline{x}_*=(\overline{q}_*,\overline{0})$. Il linearizzato è il sistema $\Dot{\overline{y}}=C\overline{y}$ dove $\overline{y}=\overline{x}-\overline{x}_*$ e $C_{ij}=\frac{\partial f_i}{\partial x_j}(\overline{x}_*)$, da cui otteniamo
    \begin{displaymath}
        \begin{dcases}
            \overline{y}=(\overline{q}-\overline{q}_*, \overline{p}) \\
            C_{ij}=\begin{pmatrix}
                0 & A^{-1}(\overline{q}_*)\\
                -Hess_U(\overline{q}_*) & 0
            \end{pmatrix}
        \end{dcases}
        \quad\implies\quad
        \begin{dcases}
            \Dot{q}_i=A^{-1}(\overline{q}_*)p_i \\
            \Dot{p}_i=-\sum_{j=1}^g\frac{\partial^2U}{\partial q_i\partial q_j}(\overline{q}_*)(\overline{q}-\overline{q}_*)_j
        \end{dcases}
    \end{displaymath}
    cioè $A\Ddot{\overline{u}}= -B\overline{u}$.
\end{proof}

\begin{defi}
    Chiamiamo \textbf{piccoli oscillazioni} rispetto ad una posizione di equilibrio stabile $\overline{q}_*$ i moti descritti dalla Lagrangiana $\widetilde{\mathcal{L}}$ del linearizzato. Ogni singola oscillazione è anche detta \textbf{modo normale} e la relativa frequenza è la \textbf{frequenza propria} dei modi normali.
\end{defi}

\begin{teo}[Piccole oscillazioni]
    Dato il sistema linearizzato $\widetilde{\mathcal{L}}$ con $B>0$, i modi normali sono
    \begin{displaymath}
        \boxed{
        z_i(t)=z_i(0)\cos\omega_it+\frac{\dot{z}_i(0)}{\omega_i}\sin\omega_it, \quad i\in\{1,\ldots,g\}
        }
    \end{displaymath}
    e le relative frequenze proprie $\omega_i$ sono soluzioni dell'\textbf{equazione caratteristica}
    \begin{displaymath}
        \boxed{
        \det(B-\omega^2A)=0
        }
    \end{displaymath}
    Più precisamente $\exists$ matrice $C=\left(\overline{\xi}_1|\cdots|\overline{\xi}_g\right)$, dove $\overline{\xi}_i$ son gli autovettori di  $(B-\omega_i^2A)$, tale che  la soluzione dell'equazione del moto è data da
    \begin{displaymath}
        \boxed{
        \overline{u}(t)=C\overline{z}(t)
        }
    \end{displaymath}
\end{teo}

\begin{oss}
     Le piccole oscillazioni hanno la forma
     \begin{displaymath}
     \boxed{
         \overline{u}(t)=\sum_{i=1}^g\left(z_i(t)=z_i(0)\cos\omega_it+\frac{\dot{z}_i(0)}{\omega_i}\sin\omega_it\right)\overline{\xi}_i
         }
     \end{displaymath}
     cioè sono combinazioni lineari di modi normali.
\end{oss}

\chapter{Meccanica Hamiltoniana}

\section{Principi variazionali}

Vogliamo mostrare che la descrizione Lagrangiana di un sistema olonomo conservativo è in effetti derivabile a partire da un principio variazionale senza passare per le equazioni di Newton. Consideriamo allora un sistema olonomo conservativo con $g$ gradi di libertà.

\begin{defi}
    Lo \textbf{spazio delle traiettorie} per un sistema olonomo conservativo con $g$ gradi di libertà è lo spazio affine
    \begin{displaymath}
        \boxed{
        \mathcal{M}_{\overline{q}_0,\overline{q}_T,T}:=\left\{\overline{q}(t):[0,T]\to \mathbb{R}^d \; | \; \overline{q}(0)=\overline{q}_0, \overline{q}(T)= \overline{q}_T, \overline{q}\in C^1([0,T])\right\}
        }
    \end{displaymath}
\end{defi}

\begin{defi}
    Lo \textbf{spazio delle deformazioni} è lo spazio vettoriale
    \begin{displaymath}
    \boxed{
        \mathcal{M}_{0,T}:=\left\{\overline{\eta}(t):[0,T]\to \mathbb{R}^d \; |\; \overline{\eta}(0)=\overline{\eta}(T)=0, \overline{\eta}\in C^2([0,T])\right\}
        }
    \end{displaymath}
\end{defi}

\noindent Nello spazio delle traiettorie vogliamo definire un funzionale che assuma un valore critico fra tutte le possibili traiettorie su quelle che sono moti del sistema. Dobbiamo però prima di tutto dire cos'è un punto critico per un funzionale (funzione su uno spazio di funzioni $\infty$ dimensionale).

\begin{defi*}
    Sia $X$ uno spazio vettoriale normato, un \textbf{funzionale} $\mathcal{F}[f]$ è un'applicazione $X\to\mathbb{R}$.
\end{defi*}

\begin{defi*}
    Diciamo che un funzionale $\mathcal{F}: X\to\mathbb{R}$ ha un \textbf{estremale} (massimo o minimo) in $f_0$ se $\exists$ un intorno $U$ di $f_0$ t.c. sgn$(\mathcal{F}[f]-\mathcal{F}[f_0])$ è lo stesso $\forall f \in U$.
\end{defi*}

\begin{oss}
    Fra gli estemali possiamo distinguere i massimi e minimi (relativi) a seconda che il segno sia $\leq0$ o $\geq 0$.
\end{oss}

\begin{defi*}
    Un funzionale continuo $\mathcal{F}: X\to\mathbb{R}$ è \textbf{differenziabile} in $f \in X$
    \begin{displaymath}
        \Delta\mathcal{F}[\eta]:= \mathcal{F}[f+\eta]-\mathcal{F}[f] = \delta\mathcal{F}[f](\eta) + \varepsilon(\eta)\lVert\eta\rVert, \forall \eta \in X
    \end{displaymath}
    dove $\delta\mathcal{F}$ è una funzione lineare detto \textbf{differenziale} o \textbf{variazione} di $\mathcal{F}$ in $f$ e $\lim_{\lVert\eta\rVert\to 0}\varepsilon(\eta)=0$.
\end{defi*}

\begin{teo*}
    Se $\mathcal{F}$ ha un estremale in $f_0\implies\delta\mathcal{F}[f_0](\eta)=0, \forall \eta \in X$, cioè $f_0$ è un \textbf{punto stazionario} di $\mathcal{F}$.
\end{teo*}

\noindent Consideriamo ora i funzionali della forma $\mathcal{F}[f]:=\int_a^bF(f;f';x)\,dx$ dove $F\in C^2(\mathbb{R}\times\mathbb{R}\times[a, b])$ e $f\in X=\left\{f\in C^1[a,b] \; | \; f(a)=f(b)=0\right\}$.

\begin{teo*}[Equazioni di Eulero-Lagrange]
    $\mathcal{F}$ differenziabile in X ha un punto stazionario in $f_0\in X$ se e solo se
    \begin{displaymath}
        \boxed{
        \left[\frac{d}{dx}\frac{\partial F}{\partial f'}-\frac{\partial F}{\partial f}\right]\bigg|_{f=f_0}=0 \quad \forall x \in [a,b]
        }
    \end{displaymath}
\end{teo*}

\begin{defi}
    Il \textbf{funzionale d'azione} è il funzionale $\mathcal{A}$ sullo spazio affine $\mathcal{M}_{\overline{x}_0,\overline{x}_T,T}$ dato da
    \begin{displaymath}
        \boxed{
        \mathcal{A}(\overline{q}):=\int_0^T\mathcal{L}(\overline{q};\dot{\overline{q}};t)\,dt
        }
    \end{displaymath}
    definito sullo spazio delle funzioni $\overline{q}:[0,T]\to\mathbb{R}^g$.
\end{defi}

\begin{teo}[Principio variazionale di Hamilton]
    Assumendo che $T$ e $U$ siano funzioni $C^2(\mathbb{R}^g\times\mathbb{R}^g)$, ogni moto di un sistema olonomo conservativo rende stazionario il funzionale d'azione, cioè se $\overline{q}_{\mathcal{L}}(t)$ per $t\in[0,T]$ è un moto del sistema allora $\delta\mathcal{A}[\overline{q}_{\mathcal{L}}]=0$.
\end{teo}

\begin{teo}[Principio di minima azione]
     Assumendo che $T$ e $U$ siano funzioni $C^2(\mathbb{R}^g\times\mathbb{R}^g)$e che $\frac{\partial^2T}{\partial\dot{q}_j\partial\dot{q}_k}>0$ (come matrice) $\exists T>0 $ t.c. se $\overline{q}(t)$ rende stazionario $\mathcal{A}[\overline{q}]$ nell'intervallo $[0,T]$ allora $\mathcal{A}[\overline{q}]$ ha un minimo relativo in $\overline{q}$.
\end{teo}

\begin{teo*}[Moltiplicatori di Lagrange]
\everymath{\displaystyle}
    Sia $X_{\mathcal{G}}:=\left\{\overline{f}\in C^2([a,b]) \; |\; \overline{f}(a)=\overline{A}, \overline{f}(b)=\overline{B}, \mathcal{G}[\overline{f}]=l\right\}$ dove $l\in\mathbb{R}$ e
    \begin{displaymath}
        \mathcal{G}[\overline{f}]:=\int_a^bG(\overline{f};\overline{f}';x)dx \; \text{per un certo} \; G\in C^2(\mathbb{R}^g\times\mathbb{R}^g\times\mathbb{R}) \; (\textbf{vincolo})
    \end{displaymath}
    Allora $\overline{f}_0$ è un punto stazionario di $\mathcal{F}[\overline{f}]:=\int_a^bF(\overline{f};\overline{f}';x)\,dx$ su $X_{\mathcal{G}} \iff \overline{f}_0$ punto stazionario di $\mathcal{F}[\overline{f}]+\lambda\mathcal{G}[\overline{f}]$, $\lambda\in\mathbb{R}$ (\textbf{moltiplicatore di Lagrange}).
\end{teo*}

\section{Equazioni di Hamilton}

\begin{defi}
    Data una funzione \underline{convessa} $\left(\iff \det\frac{\partial^2f}{\partial x_j\partial x_k}>0\right)$ $f(\overline{x})$ di classe $C^2$, la sua \textbf{trasformata di Legendre} è la funzione
    \begin{displaymath}
        \boxed{
        g(\overline{p}):=\sup_{\overline{x}\in\mathbb{R}^d} \left\{\overline{p}\cdot\overline{x}-f(\overline{x})\right\}
        }
    \end{displaymath}
\end{defi}

\begin{oss}
    Se il $\sup$ è anche un massimo allora $g(\overline{p})=\overline{p}\cdot\overline{x}_0-f(\overline{x}_0)$ per un certo $\overline{x}(\overline{p})$ punto di massimo di $\overline{p}\cdot\overline{x}-f(\overline{x})$, quindi tale che $\nabla(\overline{p}\cdot\overline{x}-f(\overline{x}))|_{\overline{x}_0}=0$, ovvero tale che
    \begin{displaymath}
        \overline{p}=(\nabla f)\left(\overline{x}(\overline{p})\right)
    \end{displaymath}
\end{oss}

\begin{oss}
    Per $g=1$ quindi la trasformata di Legendre $p=f'(x(p))$ si può riassumere nella costruzione della retta tangente di pendenza $p$ nell'ascissa $x(p)$.
\end{oss}

\begin{oss}
     Se $f$ è $C^2$ e strettamente convessa (cioè vale la condizione $\det\frac{\partial^2f}{\partial x_j\partial\ x_k}>0$ e non ci sono asintoti obliqui) $\forall \overline{x}\in\mathbb{R}^d$ allora il $\sup$ è in effetti un $\max$.
\end{oss}

\begin{prop}
\everymath{\displaystyle}
     La trasformata di Legendre è una funzione $C^2$ convessa ed è una trasformata \textbf{involutiva} cioè se $g(\overline{p})$ è la trasformata di $f$ allora $f(\overline{x})=\sup_{\overline{p}\in\mathbb{R}^g}\left\{\overline{p}\cdot\overline{x}-g(\overline{p})\right\}$.
\end{prop}

\begin{proof}
    Supponiamo per semplicità che il $\sup$ sia un massimo e consideriamo solo il caso $g=1$. Derivando la definizione abbiamo che
    \begin{displaymath}
        g'(p)=x+px'(p)-f'(x(p))x'(p)=x(p)+\cancel{px'(p)}-\cancel{px'(p)}
    \end{displaymath}
    da cui ricaviamo $g'(p)=\left(f'\right)^{-1}(p)$ e quindi $g''(p)=\frac{1}{f''\left( f(f^{-1}(p)\right)}=\frac{1}{f''\left(x(p)\right)}>0$.

    \noindent Sia ora $\Tilde{f}(z)$ la trasformata di Legendre di $g(p)$, allora $g(p)=px(p)-f(x(p))$ dove $p=f'(x(p))$ e $\Tilde{f}(z)=zp(z)-g(p(z))=zp(z)-p(z)x(p(z))+f(x(p(z)))$ dove $z=g'(p(z))$. Quindi è sufficiente dimostrare che $x(p(z))=z$, ma
    \begin{displaymath}
        x(p(z))=x\left((g')^{-1}(z)\right)=(f^{-1})\left((g')^{-1}(z)\right)=z \text{ perché $(g')^{-1}=f'$}
    \end{displaymath}
\end{proof}

\begin{defi}
    Data una Lagrangiana $\mathcal{L}(\overline{q};\Dot{\overline{q}};t)$, assumendo che $T, U \in C^2$ e che $T>0$, l'\textbf{Hamiltoniana} corrispondente è la trasformata di Legendre di $\mathcal{L}$ rispetto a $\dot{\overline{q}}$, ovvero
    \begin{displaymath}
    \boxed{
        \mathcal{H}(\overline{q};\overline{p};t):=\sup_{\overline{\eta}\in\mathbb{R}^g}\left\{\overline{p}\cdot\overline{\eta}-\mathcal{L}(\overline{q};\overline{\eta};t)\right\}
        }
    \end{displaymath}
    Chiamiamo inoltre \textbf{momento coniugato} $p_1,\ldots,p_k$ le variabili definite tramite la trasformata di Legendre ovvero
    \begin{displaymath}
        \boxed{
        p_k=\frac{\partial\mathcal{L}}{\partial\dot{q}_k}(\overline{q};\dot{\overline{q}}(\overline{p});t)
        }
    \end{displaymath}
    e le variabili $(\overline{q}, \overline{p})$ \textbf{variabili canoniche}.
\end{defi}

\begin{prop}
    Sia $\mathcal{L}(\overline{q};\dot{\overline{q}})=\frac{1}{2}\dot{\overline{q}}A(\overline{q})\dot{\overline{q}}-U(\overline{q})$ con $A, U\in C^2$ e $A>0$, allora
    \begin{displaymath}
    \boxed{
        \mathcal{H}(\overline{q};\overline{p})=\frac{1}{2}\overline{p}A^{-1}(\overline{q})\overline{p}+U(\overline{q})
        }
    \end{displaymath}
\end{prop}

\begin{proof}
    \begin{displaymath}
        p_k=\frac{\partial\mathcal{L}}{\partial\Dot{q}_k}\left(\overline{p};\overline{\eta}(\overline{p})\right)=\left(A\overline{\eta}\right)_k\implies\overline{\eta}(\overline{p})=A^{-1}(\overline{q})\overline{p}
    \end{displaymath}
    da cui
    \begin{displaymath}
        \mathcal{H}(\overline{q},\overline{p})=\overline{p}\cdot\left(A^{-1}(\overline{q})\overline{p}\right)-\frac{1}{2}\overline{p}\cancel{A^{-1}(\overline{q})}\cancel{A(\overline{q})}\left(A^{-1}(\overline{q})\overline{p}\right)+U(\overline{q})=\frac{1}{2}\overline{p}A^{-1}(\overline{q})\overline{p}+U(\overline{q})
    \end{displaymath}
\end{proof}

\begin{defi}
    Le \textbf{equazioni di Hamilton} sono il sistema di equazioni differenziali
    \begin{equation*}
    \boxed{
        \begin{dcases}
            \dot{q}_k=\frac{\partial\mathcal{H}}{\partial p_k} \\
            \dot{p}_k=-\frac{\partial\mathcal{H}}{\partial q_k}
        \end{dcases}
            }
    \end{equation*}
    con dati iniziali $\overline{q}(0)=\overline{q}_0$, $\overline{p}(0)=\overline{p}_0$.
\end{defi}

\begin{prop}
     Sia $(\overline{q}(t),\overline{p}(t))$ una qualunque soluzione delle equazioni di Hamilton, allora 
     \begin{displaymath}
     \boxed{
         \frac{d\mathcal{H}}{dt}(\overline{q}(t);\overline{p}(t);t)=\frac{\partial\mathcal{H}}{\partial t}(\overline{q}(t);\overline{p}(t);t)
         }
     \end{displaymath}
\end{prop}

\begin{proof}
    \begin{displaymath}
        \frac{d\mathcal{H}}{dt}=\sum_{k=1}^d\left\{
        \frac{\partial\mathcal{H}}{\partial q_k}\Dot{q}_k+\frac{\partial\mathcal{H}}{\partial p_k}\Dot{p}_k\right\}+\frac{\partial\mathcal{H}}{\partial t}
    \end{displaymath}
\end{proof}

\begin{cor}
    Se $\mathcal{H}$ non dipende esplicitamente da $t$ (ovvero se $\mathcal{L}$ non dipende esplicitamente da $t$), allora $\mathcal{H}$ è integrale primo.
\end{cor}

\begin{defi}
     La funzione $t\to (\overline{q}(t),\overline{p}(t))$ è detta \textbf{flusso Hamiltoniano}.
\end{defi}

\begin{oss}
     Le equazioni di Hamilton sono formalmente equivalenti alle equazioni di Lagrange.
\end{oss}

\begin{oss}
     L'insieme in cui variano le coppie $(\overline{q},\overline{p})$ è detto \textbf{spazio delle fasi}.
\end{oss}

\begin{oss}
    Così come le equazioni di Lagrange so ottengono dalla stazionarietà dell'azione, le equazioni di Hamilton si ottengono dalla stazionarietà di $\mathcal{\widetilde{A}}\left[(\overline{q},\overline{p})\right]=\int_0^T\left\{\overline{p}\cdot\dot{\overline{q}}(\overline{p}) - \mathcal{H}(\overline{p};\dot{\overline{q}})\,dt\right\}$ nello spazio delle traiettorie $\mathcal{N}_{\overline{q}_0;\overline{q}_T;T}:=\left\{(\overline{q},\overline{p}) \; | \; \overline{q}(0)=\overline{q}_0, \overline{q}(T)=\overline{q}_T\right\}$.
\end{oss}

\begin{oss}
    $\mathcal{H}$ è definita a meno di $C$ additiva mentre $\mathcal{L}$ è definita a meno di una $\dot{f}(t)$ additiva.
\end{oss}

\begin{teo}[Liouville]
    Il flusso Hamiltoniano conserva il volume nello spazio delle fasi, cioè $\forall \mathcal{D}\subset\Gamma, |\mathcal{D}|=|\mathcal{D}_t|$ dove $\mathcal{D}_t:=\left\{\overline{x}(t)=(\overline{q}(t),\overline{p}(t)) \; | \; \exists \; (\overline{q}_0,\overline{p}_0)\in \mathcal{D} \text{ e } (\overline{q}(0),\overline{p}(0))=(\overline{q}_0,\overline{p}_0)\right\}$.
\end{teo}

\begin{proof}
    \begin{displaymath}
        \int_{\mathcal{D}_t}d\overline{x}=\int_{\mathcal{D}}\det{A(t)}\,d\overline{y}
    \end{displaymath}
    cambio di variabili $\overline{x}(t)=\overline{x}(\overline{y}; t)$ dove $\overline{y}$ sono i dati iniziali e $A_{jk}(t)=\dfrac{\partial x_j(\overline{y}; t)}{\partial y_k}$.

    \noindent Si noti che per le proprietà dell'evoluzione temporale se il risultato è vero $\forall t <<1$ allora è vero $\forall t\in\mathbb{R}$ poiché $\overline{x}(\overline{y};t)=\overline{x}(\overline{x}(\overline{y};\Tilde{t});t-\Tilde{t})$. In particolare, questo implica che è sufficiente dimostrare che $\frac{d}{dt}\left|\mathcal{D}_t\right|\big|_{t=0}=0$. Per $t<<1$ abbiamo
    \begin{displaymath}
        \overline{x}(\overline{y};t)=\overline{y}+\Dot{\overline{x}}(\overline{y});t)|_{t=0}t+o(t^2)\implies\overline{x}(\overline{y};t)=\overline{y}+\overline{f}(\overline{y});t)t+o(t^2)
    \end{displaymath}
    da cui
    \begin{gather*}
        \frac{\partial x_j}{\partial y_k}(\overline{y};t)=\delta_{jk}+\frac{\partial f_j}{\partial y_k}(\overline{y};0)t+o(t^2)\implies A(t)=\mathbbm{1}+J_{\overline{f}}(\overline{y};0)t+o(t^2) \\
        \implies\det{A}=1+\operatorname{Tr}{J_f}(\overline{y};0)t+o(t^2)\implies\frac{d}{dt}(\det{A})\big|_{t=0}=\operatorname{Tr}{J_{\overline{f}}(\overline{y};0}) \\
        \implies\frac{d}{dt}\left|\mathcal{D}_t\right|\big|_{t=0}=\int_{\mathcal{D}}\operatorname{Tr}J_{\overline{f}}(\overline{y};0)\,d\overline{y}=\int_{\mathcal{D}}\sum_{j=1}^{2g}\frac{\partial f_j}{\partial y_j}(\overline{y}; 0)\,d\overline{y}=\int_{\mathcal{D}}\nabla\cdot\overline{f}(\overline{y};0)\,d\overline{y}=0
    \end{gather*}
    perché il campo vettoriale Hamiltoniano ha divergenza nulla.
\end{proof}

\begin{oss}
    Si può affermare che ogni campo vettoriale a divergenza nulla genera flusso che conserva il volume, ovvero il flusso $\dot{\overline{x}}=\overline{f}(\overline{x})$ conserva il volume se $\nabla\cdot\overline{f}=0$.
\end{oss}

\begin{oss}
    Il teorema di Liouville implica che in un sistema Hamiltoniano non possono esistere punti di equilibrio asintoticamente stabili.
\end{oss}

\begin{teo}[Eterno ritorno di Poincarè]
    Dato $\Omega \subset \Gamma$ compatto, sia $\overline{x}(t)$ un flusso Hamiltoniano t.c. $\overline{x}(\cdot;t): \Omega \to \Omega$, allora $\forall B\subset\Omega$ palla aperta $\forall t>0 \; \exists\;\bar{t}>t$ t.c. $B_{\bar{t}}\cap B \neq \emptyset$.
\end{teo}

\begin{proof}
    Consideriamo gli insiemi $\mathcal{B}_{nt}$ allora $\exists n_0$ e $k\in\mathbb{N}$ t.c. $\mathcal{B}_{n_0t}\cap\mathcal{B}_{(n_0+k)t}\neq\emptyset$ perché per il teorema di Liouville il volume di tutte le palle è lo stesso e sono tutte contenute in $\Omega$ per ipotesi, ma $\Omega$ è compatto e non può contenere un numero $\infty$ di insiemi disgiunti di volume $>0$ fissato. Consideriamo ora $\mathcal{B}_{n_0}\cap\mathcal{B}_{(n_0+k)t}$: se $n_0=0$ la dimostrazione è completa, altrimenti prendendo un punto in $\mathcal{B}_{n_0}\cap\mathcal{B}_{(n_0+k)t}$ e tornando indietro con il flusso inverso $n_0$ volte troviamo un punto in $\mathcal{B}\cap\mathcal{B}_{kt}$ che quindi è $\neq\emptyset$.
\end{proof}

\begin{cor}
    $\forall \overline{x}_0\in\mathcal{D}$ (con al più l'eccezione di un insieme a misura nulla), la traiettoria deve tornare in $\mathcal{D}$ un numero $\infty$ di volte.
\end{cor}

\begin{proof}
    Il teorema ci dà il risultato per $\mathcal{D}=\mathcal{B}$, ma in realtà non è necessario richiedere che $\mathcal{B}$ sia una palla ma solo che abbia misura $>0$.
\end{proof}

\noindent Il teorema dell'eterno ritorno di Poincarè ha importanti implicazioni e pone problemi ad esempio ai fondamenti della descrizione statistica/termodinamica dei sistemi classici.

\chapter{Esempi di sistemi meccanici}

\section{Moti rigidi liberi}

Consideriamo un corpo rigido con un punto fisso $O$: se c'è un punto fisso, identifichiamo $O$ con quello, altrimenti possiamo metterci nel sistema di riferimento del centro di massa che in assenza di forze esterne si muove di moto rettilineo uniforme (inerziale) e quindi prendiamo $O=C_M$.

\begin{prop}
    Dato un corpo rigido con punto fisso $O$, in assenza di forze esterne, $\dot{\overline{K}'}_O=\overline{K}_O'\wedge\overline{\omega}$ dove $\overline{K}_O'$ indica il momento angolare rispetto ad $O$ misurato nel sistema solidale
\end{prop}

\begin{proof}
    In assenza di forze esterne la seconda equazione cardinale della dinamica dà $\Dot{\overline{K}}_O=\overline{0}$ ma la legge di trasformazione della derivata rispetto al tempo dei vettori implicava che nel sistema solidale $\Dot{\overline{K}'}_O$ soddisfa $\Dot{\overline{K}}_O=\Dot{\overline{K}'}_O+\overline{\omega}\wedge\overline{K}_O'$ da cui il risultato.
\end{proof}

\begin{cor}
    Usando come sistema solidale quello fissato dagli assi principali di inerziali le equazioni del moto prendono la forma delle \textbf{equazioni di Eulero}
    \begin{displaymath}
        \begin{dcases}
        I_1\dot{\omega}_1=(I_2-I_3)\omega_2\omega_3 \\
        I_2\dot{\omega}_2=(I_3-I_1)\omega_1\omega_3\\
        I_3\dot{\omega}_3=(I_1-I_2)\omega_1\omega_2
    \end{dcases}
    \end{displaymath}
\end{cor}

\begin{oss}
    Se il sistema non è libero, ma sottoposto a forze esterne di momento $\overline{M}_0$, le equazioni di Eulero vanno modificate come segue:
    \begin{displaymath}
        \begin{dcases}
            I_1\dot{\omega}_1=(I_2-I_3)\omega_2\omega_3 + M_1 \\
            I_2\dot{\omega}_2=(I_3-I_1)\omega_1\omega_3 +M_2\\
            I_3\dot{\omega}_3=(I_1-I_2)\omega_1\omega_2 +M_3
        \end{dcases}
    \end{displaymath}
    dove $\overline{M}_0=M_1\hat{e}_1+M_2\hat{e}_2+M_3\hat{e}_3$.
\end{oss}

\begin{prop}
\everymath{\displaystyle}
    Le equazioni di Eulero ammettono 2 integrali primi del moto:
    \begin{itemize}
        \item l'\textbf{energia} $E=\frac{1}{2}\sum_{j=1}^3I_j\omega_j^2 = \frac{1}{2}\sum_{j=1}^3\frac{K_j^2}{I_j}$
        \item il \textbf{momento angolare} $|\overline{K}'|=\sum_{j=1}^3 K_j^2=\sum_{j=1}^3I_j^2\omega_j^2$
    \end{itemize}
\end{prop}

\begin{cor}
    I moti si svolgono sull'intersezione dell'ellissoide di inerzia $\overline{K}'\mathbb{I}^{-1}\overline{K}'=2E$ e la sfera $|\overline{K}'|^2=cost$.
\end{cor}

\begin{defi}
    Chiamiamo \textbf{rotazioni stazionarie} ogni moto in cui $\overline{\omega}=cost$.
\end{defi}

\begin{prop}
    Ogni corpo rigido vincolato ad un punto ammette una rotazione stazionaria attorno a ciascuno dei suoi assi principali di inerzia (in assenza di forze esterne).
\end{prop}

\begin{proof}
    Vogliamo dimostrare che $\overline{\omega}=\omega_j\hat{e}_j,\;j=1, 2, 3$ è soluzione delle equazioni di Eulero ma si vede facilmente che per $i\neq j$ le equazioni di Eulero diventano $\Dot{\omega}_i=0\implies\omega_i=0 \;\forall i\neq j$. Inoltre $\Dot{\omega}_j=0$ (perché $\omega_i=0, \forall i\neq j$) da cui troviamo $\omega_j(t)=\omega_j(0)$.
\end{proof}

\noindent Discutiamo ora i moti possibili in assenza di forze esterne:
\begin{itemize}
    \item $I_1<I_2<I_3$ (caso generale)

    I semiassi dell'ellissoide sono in questo caso $a_1=\sqrt{2EI_1}<a_2=\sqrt{2EI_2}<a_3=\sqrt{2EI_3}$. $|K'|\in\left[\sqrt{2EI_1},\sqrt{2EI_3}\right]$ altrimenti non ci sono intersezioni fra sfera e ellissoide.
    \begin{itemize}
        \item $K'=\pm\sqrt{eRI_3}$: l'intersezione è data da due punti, gli antipodi dell'ellissoide che corrispondono a 2 rotazioni stazionarie;
        \item $\sqrt{2EI_2}<|K'|<\sqrt{2EI_3}$: due curve chiuse attorno ai punti antipodali trovati in precedenza;
        \item $\sqrt{2EI_1}<|K'|<\sqrt{2EI_2}$: due curve chiuse attorno ai punti antipodali più piccoli;
        \item $K'=\pm\sqrt{2EI}$: due punti antipodali;
        \item $K'=\pm\sqrt{2EI_2}$: due punti antipodali e due curve chiuse (circonferenze) che contengono i punti.
    \end{itemize}
    \item $I_1=I_2<I_3$ (trottola asimmetrica)

    Il comportamento è lo stesso di prima per $K=\pm\sqrt{2EI_3}$ e $\sqrt{2EI_2}<K<\sqrt{2EI_3}$, con la differenza che le curve sono tutte circonferenze, mentre per $K=\pm\sqrt{2EI_1}=\pm\sqrt{2EI_2}$ l'intersezione è una circonferenza i cui punti sono tutti di equilibrio.
    \item $I_1=I_2=I_3$
    
    L'ellissoide è anch'esso una sfera, quindi o ci sono $\infty$ intersezioni che coincidono con la sfera stessa e in tal caso ogni moto è una rotazione stazionaria stabile, oppure non ci sono intersezioni.
\end{itemize}

\begin{teo}
    Se $I_1<I_2<I_3$ le rotazioni stazionarie attorno a $\hat{e}_1$ e $\hat{e}_3$ sono stabili mentre le rotazioni stazionarie attorno a $\hat{e}_2$ sono instabili.
\end{teo}

\begin{proof}
    Per quanto riguarda i valori antipodali $\pm\sqrt{2EI_1}$ e $\pm\sqrt{2EI_3}$, le orbite di $\overline{\omega}'$ per valori di $K$ vicini sono curve chiuse in un intorno dei punti antipodali e in particolare sono orbite periodiche limitate $\implies$ tali punti di equilibrio sono stabili. In effetti si può ottenere il risultato osservando che se ad esempio $K=\pm\sqrt{2EI_1}+\varepsilon$ allora
    \begin{displaymath}
        2E=\frac{K_1^2}{I_1}+\frac{K_2^2}{I_2}+\frac{K_3^2}{I_3}\implies\cancel{K_1^2}+K_2^2+K_3^2+o(\varepsilon)=\cancel{K_1^2}+\frac{I_1}{I_2}K_2^2+\frac{I_1}{I_3}K_3^2\implies K_2, K_3=o(\varepsilon)
    \end{displaymath}
    Pertanto $\omega_2,\omega_3=o(\varepsilon)$ che implica stabilità perché se $\overline{\omega}(0)=(\omega_1,\omega_2,\omega_3)$ con $\omega_2,\omega_3=o(\varepsilon)$ allora $\overline{\omega}(t)=\left(\omega_1+o(\varepsilon),\omega_2+o(\varepsilon),\omega_3+o(\varepsilon)\right) \;\forall t\in\mathbb{R}$.

    \noindent Per quanto riguarda invece $\pm\sqrt{2EI_2}$ abbiamo visto che i punti di equilibrio stanno all'intersezione di 4 curve (4 semicirconferenze) su cui i punti di equilibrio non potranno essere raggiunti se non asintoticamente (separatrici) per unicità delle equazioni del moto.

    \noindent Equivalentemente si può osservare che il linearizzato attorno alla configurazione di equilibrio $\overline{\omega}'=(0, \omega_2, 0)$ è $\Dot{\overline{\omega}'}=A\overline{\omega}$ con
    \begin{displaymath}
        A=\begin{pmatrix}
            0 & 0 & I_1^{-1}(I_2-I_3)\omega_2 \\
            0 & 0 & 0 \\
            I_3^{-1}(I_1-I_2)\omega_2 & 0 & 0
        \end{pmatrix}
    \end{displaymath}
    i cui autovalori sono $\lambda_1=0, \lambda_{2,3}=\pm\sqrt{\frac{(I_2-I_1)(I_3-I_2)}{I_1I_3}}$ per cui c'è un autovalore $>0$ e quindi il moto di $\overline{\omega}'$ è instabile per il criterio del linearizzato.
\end{proof}

\begin{oss}
    I moti del corpo rigido che corrispondono a rotazioni stazionarie rispetto agli assi di inerzia con momenti maggiori e minori non sono stabili a differenza dei corrispondenti moti di $\overline{\omega}$. Il motivo è che anche piccole variazioni nella velocità angolare possono portare a sfasamenti grandi nelle rotazioni a tempi successivi.
\end{oss}

\begin{teo}
    Se $I_1=I_2<I_3$ ogni soluzione delle equazioni di Eulero che non contenga punti di equilibrio è periodica.
\end{teo}

\section{Dinamica relativa}

Fin'ora abbiamo pensato la Lagrangiana come scritta rispetto ad un osservatore $\mathcal{O}$ inerziale ma talvolta è conveniente descrivere il sistema rispetto ad altri osservatori $\mathcal{O}'$ che possono anche essere non inerziali. Come cambia la forma di $\mathcal{L}$?

\begin{teo}
    Sia $\mathcal{O}'$ un osservatore qualunque che si muove rispetto ad $\mathcal{O}$ con velocità angolare $\overline{\omega}$ e velocità di traslazione $\overline{v}_{O'}$ e accelerazione $\overline{a}_{O'}$, allora la Lagrangiana di un punto materiale di massa $m$ rispetto a $\mathcal{O}$ è
    \begin{displaymath}
    \boxed{
        \mathcal{L}'(\overline{q}';\dot{\overline{q}'})=\frac{1}{2}m|\dot{\overline{q}'}|^2 + m\dot{\overline{q}'}\cdot\overline{\omega}\wedge\overline{q}' + \frac{1}{2}m|\overline{\omega}\wedge\overline{q}'|^2 -m\overline{a}_{O'}\cdot\overline{q}'-U(\overline{q}')
        }
    \end{displaymath}
    dove $\overline{q}'$ sono le coordinate Lagrangiane del punto rispetto ad $\mathcal{O'}$.
\end{teo}

\chapter{Meccanica dei Continui}

\section{Deformazioni}

L'obiettivo è quello di costruire un modello che ci permette di descrivere corpi continui \underline{deformabili} sia dal punto di vista statico che dinamico.

\begin{defi}
    Chiameremo \textbf{configurazione di riferimento} $\mathcal{C}^*$ di un corpo continuo la regione dello spazio $\mathbb{R}^3$ occupata dal corpo e assumeremo sempre che $\partial \mathcal{C}^*$ sia sufficientemente regolare e che $\exists\rho: \mathbb{R}^3\times\mathbb{R}\to\mathbb{R}^+$ \textbf{densità di massa} funzione integrabile t.c. la massa contenuta in $\mathcal{D}\subset\mathbb{R}^3$ sia
    \begin{displaymath}
        m(\mathcal{D})=\int_{\mathcal{D}}\rho(\overline{x};t)\,d\overline{x}
    \end{displaymath}
\end{defi}

\begin{defi}
    Chiameremo \textbf{deformazione} un qualunque diffeomorfismo (funzione di classe $C^1$ invertibile con inversa $C^1$)
    \begin{displaymath}
        \overline{\chi}: \mathcal{C}^*\to \mathcal{C}
    \end{displaymath}
    che individua lo spostamento dei punti materiali del corpo. Indicheremo con $\overline{\Pi}: \mathcal{C}\to\mathcal{C}^*$ la \textbf{deformazione inversa}, cioè $\overline{\Pi}=\overline{\chi}^{-1}$.
\end{defi}

\begin{oss}
    Per distinguere i punti della configurazione di riferimento useremo $*$ e quindi $\overline{x}=\overline{\chi}(\overline{x}^*) \in \mathcal{C}$.
\end{oss}

\begin{oss}
     Le richieste sulla funzione che descrive la deformazione sono in effetti ereditate da ipotesi fisiche:
     \begin{itemize}
         \item $\overline{\chi}$ è $C^1$ perché la deformazione non può generare discontinuità (fratture).
         \item $\overline{\chi}$ iniettiva perché si suppone che due punti materiali non possano occupare la stessa posizione.
     \end{itemize}
\end{oss}

\begin{defi}
    Lo \textbf{spostamento} del punto materiale $\overline{x}^*$ è dato da $\overline{u}(\overline{x}^*)=\overline{\chi}(\overline{x}^*)-\overline{x}^*$.
\end{defi}

\begin{oss}
    La biunivocità di $\overline{\chi}$ garantisce che si possa calcolare equivalentemente $\overline{u}=\overline{x}-\overline{\Pi}(\overline{x})$.
\end{oss}

\begin{defi}
    Il \textbf{gradiente di deformazione} è il tensore di secondo ordine dato dal differenziale di $\overline{\chi}$ ovvero la trasformazione lineare $d\overline{\chi}:\mathcal{C}^*\to\mathbb{R}^3 $ data da
    \begin{displaymath}
        (d\overline{\chi})_{jk}=\frac{\partial\chi_j}{\partial x_k}
    \end{displaymath}
\end{defi}

\begin{oss}
    Si ha quindi $\overline{\chi}(\overline{x}^*+\overline{h})=\overline{\chi}(\overline{x}^*)+d\overline{\chi}(\overline{x}^*)\overline{h} + o(\overline{h})$ per $\overline{h}\to \overline{0}$.
\end{oss}

\begin{oss}
    Richiederemo sempre che $J(\overline{x}^*):=\det(d\overline{\chi})(\overline{x}^*)>0 \;\forall \overline{x}^* \in \mathcal{C}^*$.
\end{oss}

\begin{oss}
    É immediato dimostrare che $d\overline{\Pi}(\overline{x})=(d\overline{\chi})^{-1}(\overline{x})$.
\end{oss}

\begin{defi}
    Le \textbf{deformazioni omogenee} sono quelle descritte da un diffeomorifismo $\mathcal{\chi}$ t.c. $d\overline{\chi}=D$ è costante e
    \begin{displaymath}
        \overline{\chi}(\overline{x}^*)=\overline{\chi}(\overline{y}^*)+D(\overline{x}^*-\overline{y}^*)
    \end{displaymath}
\end{defi}

\begin{oss}
    Per definizione e linearità del differenziale tutte le deformazioni sono almeno localmente omogenee.
\end{oss}

\noindent Esempi di deformazioni:
\begin{itemize}
    \item \textbf{traslazioni}: $D=\mathbbm{1}$ quindi $\overline{\chi}(\overline{x}^*)=\overline{\chi}(\overline{y}^*)+\overline{x}^*-\overline{y}^* \implies \overline{u}(\overline{x}^*)=\overline{u}(\overline{y}^*)=\overline{u}$ cioè lo spostamento non dipende dal punto e $\forall \; \overline{x}^* \in \mathcal{C}^* \; \overline{\chi}(\overline{x}^*)=\overline{x}^*+\overline{u}$ cioè stiamo traslando tutti i punti del vettore $\overline{u}$ fissato;
    \item \textbf{deformazioni con punto fisso}: per assegnare una deformazione omogenea t.c $\exists\;\overline{x}^*\in\mathcal{C}^*$ punto fisso cioè $\overline{\chi}(\overline{x}^*)=\overline{x}^*$ è sufficiente assegnare il gradiente $D$: $\overline{\chi}(\overline{y}^*)=\overline{x}^*+D(\overline{y}^*-\overline{x}^*)$.
    \item \textbf{rototralazione}: $D=R\in O_3$. Una deformazione omogenea con $D$ una rotazione, ovvero $\overline{\chi}(\overline{x}^*)=\overline{\chi}(\overline{z}^*)+R(\overline{x}^*-\overline{z}^*)$;
    \item \textbf{rotazione}: è una rototraslazione con un punto fisso $\overline{\chi}(\overline{x}^*)=\overline{z}^*+R(\overline{x}^*-\overline{z}^*)$;
\end{itemize}

\begin{prop}
    Ogni deformazione omogenea $\overline{\chi}$ è ottenibile come combinazione di una deformazione con punto fisso e una traslazione (prima o dopo).
\end{prop}

\begin{proof}
    Fissato un punto $\overline{z}^*\in\mathcal{C}^*$ definiamo la deformazione omogenea $\overline{\xi}(\overline{x}^*)=\overline{z}^*+D(\overline{x}^*-\overline{z}^*)$ e le traslazioni $\overline{t}_1(\overline{x}^*)=\overline{x}^*+\overline{\chi}(\overline{z}^*)-\overline{z}^*$, $\overline{t}_2(\overline{x}^*)=\overline{x}^*+D^{-1}(\overline{\chi}(\overline{z}^*)-\overline{z}^*)$, allora si ha
    \begin{displaymath}
        \overline{t}_1\left(\overline{\xi}(\overline{x}^*)\right)=\overline{\xi}(\overline{x}^*)+\overline{\chi}(\overline{z}^*)-\overline{z}^*=\cancel{\overline{z}^*}+D(\overline{x}^*-\overline{z}^*)+\overline{\chi}(\overline{z}^*)-\cancel{\overline{z}^*}=\overline{\chi}(\overline{x}^*)
    \end{displaymath}
    e analogamente $\overline{\xi}\left(\overline{t}_2(\overline{x}^*)\right)=\overline{\chi}(\overline{x}^*)$.
\end{proof}

\begin{prop}
    Date due deformazioni $\overline{\chi}_1$ e $\overline{\chi}_2$, il gradiente di $\overline{\chi}_1\circ\overline{\chi}_2$ è: $d(\overline{\chi}_1\circ\overline{\chi}_2)=d\overline{\chi}_1\cdot d\overline{\chi}_2$.
\end{prop}

\begin{oss}
    Data una rotazione con punto fisso $\overline{z}^*$, i punti della retta $\overline{x}^*=\lambda\hat{e}+\overline{z}^*$ sono lasciati fissi dove $\hat{e}$ è l'asse della rotazione $R$.
\end{oss}

\begin{oss}
    Date due rotazioni la composizione $\overline{\chi}_1\circ\overline{\chi}_2$ è una rotazione di gradiente $R_1\cdot R_2$.
\end{oss}

\begin{lemma}[Decomposizione polare]
    $\forall \;D\in L^+(\mathbb{R}^3),\; \exists!\; R \in O_3 \text{ e } U,V\in L_{sym}^+(\mathbb{R}^3) \text{ t.c. }D=RU=VR$.
\end{lemma}

\begin{defi}
    Una deformazione omogenea generica cioè con $D\in L_{sym}^+(\mathbb{R}^3)$ (gathertori lineari simmetrici positivi) è detta \textbf{deformazione pura}. 
\end{defi}

\begin{prop}
    Ogni deformazione omogenea con punto fisso e gradiente $D$ si scrive $\overline{\chi}=\overline{\rho}\circ\overline{\xi}_1=\overline{\xi}_2\circ\overline{\rho}$ dove $\overline{\rho}$ è una rotazione di gradiente $R$ e $\overline{\xi}_1 \text{e }\overline{\xi}_2$ sono delle deformazioni pure di gradienti $U$ e $V$.
\end{prop}

\begin{defi}
    Uno \textbf{stiramento} di intensità $\lambda\in\mathbb{R}^+$ nella direzione $\hat{e}\in\mathbb{R}^3$ è la deformazione pura  $\overline{\chi}(\overline{x}^*)=\overline{x}^*+(\lambda-1)[(\overline{x}^*-\overline{y}^*)\cdot\hat{e}]\,\hat{e}$.
\end{defi}

\noindent Data una qualunque deformazione localmente in un intorno di $\overline{x}^*$ si può approssimare con una deformazione omogenea
\begin{displaymath}
    \overline{\chi}(\overline{x}^*+\overline{h})=\overline{\chi}(\overline{x}^*)+d\overline{\chi}(\overline{x}^*)\overline{h} +o(\overline{h})
\end{displaymath}
per cui è naturale studiare la decomposizione polare di $d\overline{\chi}$.

\begin{defi}
    Data una deformazione $\overline{\chi}$ il \textbf{tensore di Cauchy-Green} destro e sinistro sono rispettivamente
    \begin{displaymath}
        \boxed{B:=d\overline{\chi}\cdot d\overline{\chi}^T} \ \ \ \ \ \boxed{C:=d\overline{\chi}^T\cdot d\overline{\chi}}
    \end{displaymath}
\end{defi}

\begin{oss}
    Sia $B$ che $C \in L_{sym}^+(\mathbb{R}^3)$ e data la decomposizione polare $d\overline{\chi}=RU=VR$ si ha
    \begin{itemize}
        \item $B=U^2$, $C=V^2$, $B=RCR^T$;
        \item autovalori di $B =$ autovalori di $C$;
        \item $I_j(B)=I_j(C)\;\forall \;j\in \{1,2,3\}$, dove $I_j$ sono gli invarianti ortogonali.
    \end{itemize}
\end{oss}

\begin{defi*}
    Un \textbf{invariante ortogonale} è una funzione $f:L(\mathbb{R}^3)\to\mathbb{R} \text{ t.c. } f(RA)=f(A) \;\forall R\in O_3 \text{ e } \forall A \in L(\mathbb{R}^3)$.
\end{defi*}

\begin{prop*}
    Le seguenti funzioni sono invarianti ortogonali:
    \begin{itemize}
        \item $I_1(A):=tr(A)$;
        \item $I_2(A):=\dfrac{1}{2}[(tr(A))^2-tr(A^2)]$
        \item $I_3(A):=\det(A)$.
    \end{itemize}
\end{prop*}

\begin{defi*}
    Dato uno spazio vettoriale $V$, un \textbf{tensore} di ordine $n\in\mathbb{N}_0$ è un'applicazione multilineare $T: V^n\to\mathbb{R}$.
\end{defi*}

\begin{oss}
    Il polinomio caratteristico di un tensore in $L(\mathbb{R}^3)$ si esprime come $p(\lambda)=-\lambda^3+I_1\lambda^2-I_2\lambda+I_3=0$ e in effetti $\forall \;A\in L(\mathbb{R}^3)$ $-A^3+I_1(A)A^2-I_2(A)A+I_3(A)=0$.
\end{oss}

\begin{defi*}
    Dato uno spazio vettoriale $V$ di dimensione $m\in\mathbb{N}$ e due vettori $\overline{a}, \overline{b}\in V$, il loro \textbf{prodotto tensoriale} $\overline{a}\otimes\overline{b}$ è un tensore di ordine $2: V^n \to \mathbb{R}$ t.c. $\left(\overline{a}\otimes\overline{b}\right)_{ij}=a_ib_j$.
\end{defi*}

\begin{oss}
    Il gradiente di deformazione $d\overline{\chi}$ può essere pensato come trasformazione: $\mathcal{C}^*\to\mathcal{C}$ nel senso che $dx_i=\sum_{j=1}^3(d\overline{\chi})_{ij} dx^*_j$.
\end{oss}

\begin{oss}
    Allo stesso modo $d\overline{\chi}^T$ può essere pensato come applicazione lineare: $\mathcal{C}\to\mathcal{C}^*$ e quindi i tensori di Cauchy-Green $B: \mathcal{C}\to\mathcal{C} \text{ e } C: \mathcal{C}^*\to\mathcal{C}^*$.
\end{oss}

\noindent Data una deformazione omogenea $\overline{\chi}$ e una curva $\overline{\gamma}:\mathbb{R}\to\mathcal{C}^*$ t.c.
$\begin{cases}
    \overline{\gamma}(s^*)=\overline{x}^*\\
    \overline{\gamma}'(s^*)=\hat{e}
\end{cases}$
possiamo costruire una curva deformata
$\begin{cases}
    \widetilde{\overline{\gamma}}(s)=\overline{\chi}(\overline{\gamma}(s))\\
    \widetilde{\overline{\gamma}'}(s^*)=D\hat{e}
\end{cases}$
ovvero il vettore tangente alla curva deformata in $\overline{x}=\overline{\chi}(\overline{x}^*)=\overline{\chi}(\overline{\gamma}(s^*))$ è $D\hat{e}$.

\begin{defi}
    Il rapporto fra la lunghezza dell'arco della curva deformata e l'arco della curva nella configurazione di riferimento è $\delta(\hat{e}):=\frac{d\tilde{s}}{ds}=|D\hat{e}|$ ed è detto \textbf{stiramento} in direzione $\hat{e}$ mentre $\lambda(\hat{e}) :=\delta(\hat{e})-1$ è la \textbf{deformazione longitudinale}.
\end{defi}

\begin{prop}
    Data una deformazione $\overline{\chi}$, si ha
    \begin{displaymath}
        \lim_{\varepsilon\to 0^+} \frac{|\overline{\chi}(\overline{x}^*+\varepsilon\hat{e})-\overline{\chi}(\overline{x}^*)|^2}{\varepsilon^2}=\hat{e}C\hat{e}
    \end{displaymath}
\end{prop}

\noindent Consideriamo ora due versori ortogonali $\hat{e}_1$ e $\hat{e}_2$ e chiediamoci cosa succede ad essi dopo la deformazione: prese due curve $\overline{\gamma}_{1,2}:\mathbb{R}\to\mathcal{C}^*$ con $\overline{\gamma}_1(s^*)=\overline{\gamma}_2(s^*)=\overline{x}^*$ e $\overline{\gamma}_1'(s^*)=\hat{e}_1$, $\overline{\gamma}_2'(s^*)=\hat{e}_2$ avremo che $\overline{\chi}'(\overline{\gamma}_1(s^*))=d\overline{\chi}\cdot\hat{e}_1$, $\overline{\chi}'(\overline{\gamma}_2(s^*))=d\overline{\chi}\cdot\hat{e}_2$ e quindi $d\overline{\chi}\hat{e}_1$ e $d\overline{\chi}\hat{e}_2$ formeranno in genere un angolo $\neq \frac{\pi}{2}$.

\begin{defi}
    Si chiama \textbf{angolo di scorrimento} la quantità $\gamma(\hat{e}_1,\hat{e}_2):= \frac{\pi}{2}-\theta(\hat{e}_1,\hat{e}_2)$ ovvero $\frac{\pi}{2}$ meno l'angolo formato tra $d\overline{\chi}\hat{e}_1$ e $d\overline{\chi}\hat{e}_2$.
\end{defi}

\begin{prop}
    Data una deformazione $\overline{\chi}$, si ha
    \begin{displaymath}
        \sin\gamma(\hat{e}_1,\hat{e}_2)= \frac{\hat{e}_1 C \hat{e}_2}{\sqrt{\hat{e}_1 C \hat{e}_1} \sqrt{\hat{e}_2 C \hat{e}_2}}
    \end{displaymath}
\end{prop}

\begin{prop}
    Data una deformazione $\overline{\chi}, \;\forall\overline{x}^* \in \mathcal{C}^* \; \exists$ terna ortonormale $\{\hat{e}_1, \hat{e}_2, \hat{e}_3\}$ dipendente da $\overline{x}^*$ t.c. $C$ è diagonale in tale terna e $\overline{\chi}$ rispetto alla terna è data da 3 stiramenti lungo i 3 assi $\{\hat{e}_1, \hat{e}_2, \hat{e}_3\}$ che individuano le \textbf{direzioni principali di deformazione}, ovvero rispetto a tale base $C=\operatorname{diag}\left\{{\delta}^2_1, {\delta}^2_2, {\delta}^2_3\right\}$, dove $\{\delta_i\}_{i\in\{1,2,3\}}$ sono detti \textbf{stiramenti principali}. 
\end{prop}

\noindent Vediamo ora come le deformazioni cambiano i volumi ccupati dal corpo e più in generale le funzioni (campi) dei punti materiali.

\begin{oss}
    Sia $\mathcal{D}^*\subset\mathcal{C}^* $ un insieme misurabile,  $\overline{\chi}$ una deformazione e $\mathcal{D}=\left\{ \overline{x}\in\mathcal{C} \; | \; \exists\; \overline{x}^* \in \mathcal{D}^*, \overline{x}=\overline{\chi}(\overline{x}^*)\right\}$ l'insieme deformato, allora
    \begin{displaymath}
        \left|\mathcal{D}\right|=\int_{\mathcal{D}^*}\det\left(d\overline{\chi}(\overline{x}^*)\right)d\overline{x}^*
    \end{displaymath}
\end{oss}

\begin{oss}
    Le deformazioni che non cambiano i volumi (\textbf{isocore}) sono tutte e sole quelle con $\det(d\overline{\chi})=1$.
\end{oss}

\begin{oss}
\everymath{\displaystyle}
    Data una qualunque funzione $\Phi:\mathbb{R}^3\to\mathbb{R}$ (\textbf{campo scalare}),
    \begin{displaymath}
        \int_{\mathcal{D}}\Phi(\overline{x})d\overline{x} = \int_{\mathcal{D}^*} \det(d\overline{\chi})  \Phi(\overline{x}^*)d\overline{x}^*
    \end{displaymath}
\end{oss}

\begin{defi}
    Data una \textbf{deformazione infinitesima}, cioè t.c. $\overline{u}(\overline{x}^*)=\overline{\chi}(\overline{x}^*) - \overline{x}^* <<1$, il \textbf{gradiente di deformazione infinitesimo} è
    \begin{displaymath}
        E:=\frac{1}{2}(d\overline{u}+d\overline{u}^T)=\frac12 (d\overline{\chi} + d\overline{\chi}^T)-\mathbbm{1}
    \end{displaymath}
\end{defi}

\begin{oss}
    Data una deformazione infinitesima con $\overline{u}(\overline{x}^*)=\varepsilon\overline{u}_0(\overline{x}^*)$ e $\varepsilon<<1$ si ha
    \begin{itemize}
        \item $C=d\overline{\chi}^T\cdot d\overline{\chi}=\mathbbm{1}+2E+O(\varepsilon^2)$;
        \item $U=\mathbbm{1}+E+O(\varepsilon^2)$ poichè $U^2=C$;
        \item $R=\mathbbm{1}+d\overline{u}-E+O(\varepsilon^2)=\mathbbm{1}+W+O(\varepsilon^2)$ dove $W:=\frac{1}{2} (d\overline{u}-d\overline{u}^T)$ è la componente antisimmetrica di $d\overline{u}$.
    \end{itemize}
\end{oss}

\section{Moti}

Vogliamo ora vedere come descrivere il moto di un continuo.

\begin{defi}
    Il \textbf{moto} di un sistema continuo è dato da una famiglia di deformazioni $\{\overline{\chi}(\cdot\ ,t)\}_{t\in\mathbb{R}}$ (dipendenti dal tempo)  dove $\overline{\chi}(\cdot\ ,\cdot):\mathcal{C}^*\times\mathbb{R}\to\mathcal{C}_t$ (\textbf{configurazione attuale} $\overline{\chi}(\mathcal{C}^*;t)$).
    La \textbf{traiettoria} del punto materiale $\overline{x}^*$ è data da $\overline{y}(t)= \overline{\chi}(\overline{x}^*;t), \forall t \in \mathbb{R}$, mentre la \textbf{traiettoria del sistema} è
    \begin{displaymath}
        \mathcal{C}_{\overline{\chi}}=\left\{ (\overline{x},t)\in\mathbb{R}^3\times\mathbb{R} \; |\; \exists\ \overline{x}^*\in \mathcal{C}^* , \overline{x}=\overline{\chi}(\overline{x}^*,t)\right\}
    \end{displaymath}
\end{defi}

\begin{oss}
    Considereremo solo moti sufficientemente regolari per cui $\overline{\chi}(\overline{x}^*;t)$ è almeno $C^2$ su $\mathcal{C}^*\times\mathbb{R}$.
\end{oss}

\begin{oss}
    L'invertibilità di $\overline{\chi}$ garantisce che $\forall t \in \mathbb{R}$ è definita $\overline{\Pi}(\cdot;t): \mathcal{C}_t\to\mathcal{C}^*$ t.c. $\overline{\chi}(\overline{\Pi}(\overline{x};t);t)=\overline{x}$ e $\overline{\Pi}(\overline{\chi}(\overline{x}^*;t);t)=\overline{x}^*$.
\end{oss}

\begin{defi}
    Dato un punto $\overline{x}^*\in\mathcal{C}^*$ e un moto $\overline{\chi}$, la \textbf{velocità} e l'\textbf{accelerazione} del punto materiale sono date da
    \begin{displaymath}
        \dot{\overline{x}}=\dfrac{\partial \overline{\chi}}{\partial t}(\overline{x}^*;t) \quad \text{e} \quad \ddot{\overline{x}}=\dfrac{\partial^2\overline{\chi}}{\partial t^2} (\overline{x}^*;t)
    \end{displaymath}
\end{defi}

\begin{oss}
    Con questa definizione velocità e accelerazione sono funzioni del punto $\overline{x}^*\in\mathcal{C}^*$ ma possono essere facilmente trasformate in funzioni di $\overline{x}\in\mathcal{C}_t$ via $\overline{\Pi}: \overline{v}(\overline{x};t)=\dot{\overline{x}}(\overline{\Pi}(\overline{x};t),t), \overline{a}(\overline{x};t)=\ddot{\overline{x}}(\overline{\Pi}(\overline{x};t),t)$.
\end{oss}
\begin{oss}
    Le due diverse espressioni di velocità e accelerazione prendono il nome di descrizione \textbf{Lagrangiana} o \textbf{materiale} e \textbf{Euleriana} o \textbf{spaziale}. Nel primo caso stiamo seguendo il moto dei punti materiali al variare del tempo mentre nel secondo fissiamo il tempo $t$ e guardiamo la distribuzione di velocità e accelerazione nello spazio. Analogamente chiameremo \textbf{campo materiale} una qualunque funzione $\overline{\Psi}(\overline{x}^*;t):\mathcal{C}^*\times\mathbb{R}\to\mathbb{R}^n$ e \textbf{campo spaziale} una funzione $\overline{\Phi}(\overline{x};t):\mathcal{C}_t\times\mathbb{R}\to\mathbb{R}^n$.
\end{oss}

\begin{oss}
    É sempre possibile passare da un campo materiale a uno spaziale e viceversa:
    \begin{displaymath}
        \overline{\Psi}_s(\overline{x};t):= \overline{\Psi}\left(\overline{\Pi}(\overline{x};t);t\right) \ \ \ \ \ \overline{\Phi}_m(\overline{x}^*;t):= \overline{\Phi}\left(\overline{\chi}(\overline{x}^*;t);t\right)
    \end{displaymath}
\end{oss}

\begin{prop}
    Indicando con $\dot{\overline{x}}, \ddot{\overline{x}}$ e $\overline{v},\overline{a}$ rispettivamente i campi materiali e spaziali di velocità e accelerazione, si ha
    \begin{displaymath}
        d\dot{\overline{x}}=d\overline{v}\, d\overline{\chi}= d\dot{\overline{\chi}} \ \ \ \ \ d\ddot{\overline{x}}=d\overline{a}\,d\overline{\chi}= d\ddot{\overline{\chi}}
    \end{displaymath}
\end{prop}

\begin{proof}
    Si ha scambiando l'ordine di derivazione $d\Dot{\overline{x}}=\Dot{(d\overline{\chi})}$ e
    \begin{displaymath}
        \left(d_{\overline{x}^*}\Dot{\overline{x}}\right)_{ij}=\left(d_{\overline{x}^*}\left[\overline{v}\left(\overline{\chi}(\overline{x}^*;t);t\right)\right]\right)_{ij}=\sum_k(d_{\overline{x}}\overline{v})_{ik}(d\overline{\chi})_{kj}
    \end{displaymath}
    Per $d\Ddot{\overline{x}}$ la dimostrazione è analoga.
\end{proof}
%Oss: Di conseguenza valgono\ \ \ $d\overline{v}=d\dot{\overline{\chi}}d\overline{\chi}^{-1} \ \ \ d\overline{a}=d\ddot{\overline{\chi}}d\overline{\chi}^{-1}$\\ \\
%
%
%
%Oss: Posti \ \ \ $(d_*\overline{\Psi})_{ij}=\frac{\partial \Psi_i}{\partial x^*_j} \ \ \ (d\overline{\Phi})_{ij}=\frac{\partial \Phi_i}{\partial x_j}$, \ \ \ è ovvio che \ $d_*\overline{\Psi}=d\overline{\Psi}_sd_*\overline{\chi} \ \ e \ \ d\overline{\Phi}=d_*\overline{\Phi}_md\overline{\chi}^{-1}$ \\
%
%
%Oss: Analogamente per divergenze \ \ $\nabla_*\cdot\overline{\Psi}:=tr(d_*\overline{\Psi}) \ \ \ \nabla\cdot\overline{\Phi}:=tr(d\overline{\Phi}) $\\ \\
%
%
%
%Oss: Per i campi spziali vale\ \ $\dot{\overline{\Phi}}(\overline{x};t) = d\overline{\Phi}\cdot\overline{v} + \frac{\partial\overline{\Phi}}{\partial t}$ \ e in particolare per l'accelerazione \ $\overline{a}(\overline{x};t)= d\overline{v}\cdot\overline{v} + \frac{\partial\overline{v}}{\partial t}$\\
%
%
%
\begin{defi}
    Un campo spaziale è \textbf{stazionario} se $\dfrac{\partial \overline{\Phi}}{\partial t}=\overline{0}$.
\end{defi}

\begin{defi}
    Dato un campo spaziale di velocità, il suo gradiente si decompone $d\overline{v}=D+W$ dove
    \begin{displaymath}
        \underbrace{D=\frac{1}{2}(d\overline{v}+d\overline{v}^T)}_{\textbf{tensore velocità di deformazione}} \ \ \ \ \ \underbrace{W=\frac{1}{2}(d\overline{v}-d\overline{v}^T)}_{\textbf{tensore di vorticità}}
    \end{displaymath}
\end{defi}

\begin{oss}
    Poiché $W$ antisimmetrico $\exists\;\overline{w}\in \mathbb{R}^3$ (\textbf{vorticità}) t.c. $W\overline{u}=\frac{1}{2}\overline{w}\wedge\overline{u}$ e si ha che $\overline{w}=\nabla\wedge\overline{v}$.
\end{oss}

\noindent Associate alla doppia descrizione materiale e spaziale del moto, abbiamo anche una doppia tipologia di curve che rappresentano il moto:
\begin{itemize}
    \item le curve che descrivono le traiettorie dei singoli punti materiali si dicono \textbf{linee di corrente}, cioè le curve $\overline{x}(t)=\overline{\chi}(\overline{x}^*;t)$;
    \item fissato il tempo $t\in\mathbb{R}$ possiamo però costruire un'altra famiglia di curve chiamate \textbf{linee di flusso} che siano in ogni punto tangenti al campo di velocità, ovvero tali che $\overline{x}'(s)=\overline{v}(\overline{x}(s);t)$
\end{itemize}

\begin{oss}
    Se il campo spaziale delle velocità è stazionario, allora le linee di flusso e le linee di corrente coincidono.
\end{oss}

\noindent Fin'ora abbiamo descritto il moto dei corpi continui ma non abbiamo mai escluso che tali corpi fossero rigidi, pertanto il moto dei corpi rigidi deve essere ricompreso nel modello costruito.

\begin{teo}
    Il moto di un corpo continuo è rigido $\iff D=0$.
\end{teo}

\begin{proof}

    \noindent
    \begin{itemize}
        \item ($\implies$) Come sappiamo un moto è rigido se $\forall$ coppia di punti $(\overline{x}-\overline{y})\cdot(\overline{v}(\overline{x};t)-\overline{v}(\overline{y};t))=0$ e per la formula fondamentale della cinematica rigida deve essere $\overline{v}(\overline{x};t)=\overline{v}(\overline{y};t)+\overline{\omega}\wedge(\overline{x}-\overline{y})$ che può essere riscritta come $\overline{v}(\overline{x};t)-\overline{v}(\overline{y};t)=\widetilde{W}(\overline{x}-\overline{y})$ per un certo tensore antisimmetrico $(\Dot{R}R^T)$. Quindi per definizione di differenziale
    \begin{displaymath}
        (D+W)\overline{h}=d\overline{v}\overline{h}=\lim_{\varepsilon\to0}\frac{[\overline{v}(\overline{x}+\varepsilon\overline{h})-\overline{v}(\overline{x})]}{\varepsilon}=\widetilde{W}(t)\overline{h}
    \end{displaymath}
    e quindi $D=0$ perché la parte simmetrica è nulla.
    \item ($\impliedby$) Viceversa se $D=0$, $d\overline{v}=W$ e vedremo che $W$ deve essere costante in $\mathcal{C}_t$:
    \[
    \begin{split}
        (dW)_{ijk}&=\frac{\partial W_{ij}}{\partial x_k}=\frac{\partial^2v_i}{\partial x_j\partial x_k}=\frac{\partial^2v_i}{\partial x_k\partial x_j}=\frac{\partial W_{ik}}{\partial x_j}=-\frac{\partial W_{ki}}{\partial x_j}=-\frac{\partial W_{kj}}{\partial x_i} \\
        &=\frac{\partial W_{jk}}{\partial x_i}=\frac{\partial W_{ji}}{\partial x_k}=-\frac{\partial W_{ij}}{\partial x_k}=-W_{ijk}=-(dW)_{ijk}
    \end{split}
    \]
    \begin{gather*}
        \implies dW=0\implies\overline{v}(\overline{x};t)-\overline{v}(\overline{y};t)=W(t)(\overline{x}-\overline{y}) \\
        \implies(\overline{x}-\overline{y})\cdot(\overline{v}(\overline{x};t)-\overline{v}(\overline{y};t))=(\overline{x}-\overline{y})W(t)(\overline{x}-\overline{y})=0
    \end{gather*}
    \end{itemize}
    
\end{proof}

\noindent Vediamo ora qual è il significato geometrico del tensore velocità di deformazione e del tensore di vorticità.

\begin{prop}
    Fissata una direzione $\hat{e}$ si ha $\dfrac{\dot{\delta}(t)}{\delta(t)}=\hat{e}D\hat{e}$ dove ${\delta}(t)$ è lo stiramento in direzione $\hat{e}$ al variare di $t\in\mathbb{R}$.
\end{prop}

\noindent Vediamo ora cosa succede agli angoli di scorrimento.

\begin{prop}
    Dati due versori $\hat{e}_1', \hat{e}_2'$ ortogonali nella configurazione attuale $\mathcal{C}_t$ si ha che $\dot{\theta}_{12}(t)=-2\hat{e}_1D\hat{e}_2$ (detta \textbf{velocità di scorrimento}) dove $\theta_{12}$ è l'angolo formato dai versori $d\overline{\chi}\hat{e}_1$ e $d\overline{\chi}\hat{e}_2$ al variare di $t\in\mathbb{R}$.
\end{prop}

\noindent Vediamo ora come i moti del corpo influenzano i volumi e gli integrali di volume.

\begin{prop}
    Sia $\mathcal{D}^* \subset\mathcal{C}^*$ un insieme misurabile e $\mathcal{D}_t$ il suo evoluto lungo il moto $\left\{\overline{\chi}(\cdot;t)\right\}_{t \in \mathbb{R}}$, allora
    \begin{displaymath}
        \frac{d}{dt}\int_{\mathcal{D}_t}d\overline{x}=\int_{\mathcal{D}_t}\nabla\cdot\overline{v}\,d\overline{x}
    \end{displaymath}
    dove $\overline{v}$ è il campo spaziale delle velocità.
\end{prop}

\begin{proof}
    \[
    \begin{split}
        \frac{d}{dt}\int_{\mathcal{D}_t}d\overline{x}&=\frac{d}{dt}\left\{\int_{\mathcal{D}^*}\det [d\overline{\chi}(\overline{x};t)]\,d\overline{x}^*\right\}=\int_{\mathcal{D}^*}\Dot{\left(\det [d\overline{\chi}(\overline{x};t)]\right)}\,d\overline{x}^* \\
        &\overset{\text{lemma}}{=}\int_{\mathcal{D}^*}\det(d\overline{\chi})\operatorname{tr}(d\Dot{\overline{\chi}}d\overline{\chi}^{-1})\,d\overline{x}^*=\int_{\mathcal{D}_t}\operatorname{tr}(d\overline{v})\,d\overline{x}=\int_{\mathcal{D}_t}\nabla\cdot\overline{v}\,d\overline{x}
    \end{split}
    \]
\end{proof}

\begin{lemma}
    Per ogni  tensore invertibile e derivabile $T(t)$ vale l'identità $\dot{[\det T(t)]} = \det(T)\cdot \operatorname{tr}(\dot{T}T^{-1})$.
\end{lemma}

\begin{oss}
    Affinché un moto non modifichi i volumi del corpo ovvero l'evoluzione sia isocora deve essere $\nabla\cdot\overline{v}=0$. Questa condizione per i fluidi diventerà la richiesta che il fluido sia \underline{incomprimibile}: Quindi se $D=0$ il moto è rigido mentre se $\operatorname{tr}D=0$ il corpo è incomprimibile.
\end{oss}

\begin{teo}[Trasporto di Reynolds]
    Sia $\overline{\Phi}(\overline{x},t)$ un campo spaziale regolare e sia $\mathcal{D}_t\subset\mathcal{C}_t$ un insieme con bordo $\partial\mathcal{D}_t$ regolare, allora
    \begin{displaymath}
        \frac{d}{dt}\int_{\mathcal{D}_t}\overline{\Phi}(\overline{x};t)\,d\overline{x} = \int_{\mathcal{D}_t}\frac{\partial \overline{\Phi}}{\partial t}\,d\overline{x} + \int_{\partial \mathcal{D}_t}\overline{v}\cdot\hat{n}\,\overline{\Phi}\,d\overline{\sigma}
    \end{displaymath}
    dove $\hat{n}$ è la normale uscente a $\partial\mathcal{D}_t$.
\end{teo}

\begin{proof}
    Dimostriamo il risultato per un campo scalare. Il risultato segue dall'applicazione di quanto provato a ciascuna componente di $\overline{\Phi}$. Sia allora $\Phi(\overline{x};t):\mathcal{C}\to\mathbb{R}$,
    \[
    \begin{split}
        \frac{d}{dt}\left\{\int_{\mathcal{D}_t}\Phi(\overline{x};t)\,d\overline{x}\right\}&=\frac{d}{dt}\left\{\int_{\mathcal{D}^*}\det(d\overline{\chi})\Phi(\overline{\chi}(\overline{x}^*;t);t)\,d\overline{x}^*\right\}=\int_{\mathcal{D}^*}\left\{\Dot{(\det d\overline{\chi})}\Phi+\det (d\overline{\chi})\Dot{\Phi}\right\}d\overline{x}^* \\
        &=\int_{\mathcal{D}^*}\left\{\nabla\cdot\overline{v}\,\Phi+\Dot{\Phi}\right\}\det (d\overline{\chi})\,d\overline{x}^*=\int_{\mathcal{D}_t}\left\{\nabla\cdot\overline{v}\,\Phi+\Dot{\Phi}\right\}d\overline{x} \\
        &=\int_{\mathcal{D}_t}\left\{\nabla\cdot\overline{v}\,\Phi+\frac{\partial\Phi}{\partial t}+d\Phi\overline{v}\right\}d\overline{x}=\int_{\mathcal{D}_t}\left\{\nabla\cdot(\Phi\overline{v})+\frac{\partial\Phi}{\partial t}\right\}d\overline{x} \\
        &=\int_{\mathcal{D}_t}\frac{\partial\Phi}{\partial t}\,d\overline{x}+\int_{\partial\mathcal{D}_t}\hat{n}\cdot\overline{v}\Phi\,d\overline{\sigma} \text{\quad per il teorema della divergenza}
    \end{split}
    \]
\end{proof}

\section{Leggi di bilancio}

Vediamo un'applicazione immediata di quanto visto riguardo l'evoluzione degli integrali di campi spaziali.

\noindent Ricordiamo che è assegnata una funzione $\rho_*: \mathcal{C}^*\to\mathbb{R}$ t.c. $m(\mathcal{D}^*)=\int_{\mathcal{D}^*}\rho_*(\overline{x}^*)\,d\overline{x}^*$ è la massa del corpo contenuta in $\mathcal{D}^*$.

\noindent Allo stesso modo assumeremo l'esistenza di un campo scalare spaziale $\rho(\overline{x};t)$ che fornisca la massa in $\mathcal{C}_t$, ovvero dato $\mathcal{D}_t\subset\mathcal{C}_t$, $m(\mathcal{D}_t)=\int_{\mathcal{D}_t}\rho(\overline{x};t)\,d\overline{x}$ sia la massa in $\mathcal{D}_t$.

\begin{prop}
    Condizione necessaria e sufficiente affinché ogni parte del corpo in $\mathcal{C}^*$ e $\mathcal{C}_t$ abbia la stessa massa è che $\rho_*=\det(d\overline{\chi})\rho_m$.
\end{prop}

\begin{teo}
    La massa di un corpo continuo è conservata lungo il moto se e solo se vale l'\textbf{equazione di continuità}
    \begin{displaymath}
    \boxed{
        \frac{\partial\rho}{\partial t} + \nabla\cdot(\rho\overline{v})=0
        }
    \end{displaymath}
\end{teo}

\begin{proof}
    Possiamo imporre la conservazione della massa richiedendo che la massa contenuta in ogni parte del corpo si conservi ovvero che $\forall\;\mathcal{D}_t\subset\mathcal{C}_t$ si abbia
    \begin{displaymath}
        \frac{d}{dt}\int_{\mathcal{D}_t}\rho(\overline{x};t)\,d\overline{x}=0=\int_{\mathcal{D}_t}\left\{\frac{\partial\rho}{\partial t}+\nabla\cdot(\rho\overline{v})\right\}d\overline{x}
    \end{displaymath}
    da cui l'equazione di continuità segue dal lemma.
\end{proof}

\begin{lemma}[di localizzazione]
    Sia $f$ una funzione continua: $\mathcal{C}\to\mathbb{R}$ e sia $\int_{\mathcal{D}}f(\overline{x})d\overline{x}=\overline{0} \quad\forall \mathcal{D}\subset \mathcal{C}$ misurabile $\implies f=0$.
\end{lemma}

\begin{oss}
    Se il continuo è incomprimibile si ha $\frac{\partial \rho}{\partial t} + \overline{v}d\rho=0$.
\end{oss}

\begin{cor}
    Sia $\overline{\Phi}$ un campo spaziale, allora
    \begin{displaymath}
        \frac{d}{dt}\int_{\mathcal{D}_t}\rho(\overline{x};t)\overline{\Phi}(\overline{x};t)\,d\overline{x}=\int_{\mathcal{D}_t}\rho(\overline{x};t)\frac{d\overline{\Phi}}{dt}(\overline{x};t)\,d\overline{x}
    \end{displaymath}
\end{cor}

\begin{oss}
    Le quantità meccaniche per un continuo deformabile sono:
    \begin{itemize}
        \item la \textbf{quantità di moto} $\overline{Q}=\int_{\mathcal{C}_t}\rho(\overline{x};t)\overline{v}(\overline{x};t)\,d\overline{x}$;
        \item il \textbf{momento angolare} $\overline{K}_O=\int_{\mathcal{C}_t}\rho(\overline{x};t)(\overline{x}-\overline{x}_O)\wedge\overline{v}\,d\overline{x}$.
    \end{itemize}
\end{oss}

\begin{oss}
    Le forze esterne che agiscono su un continuo possono essere di vario tipo ma restringeremo l'attenzione a
    \begin{itemize}
        \item \textbf{forze di volume} determinate da una densità $\overline{f}_v(\overline{x};t)$ così che la forza esercitata sulla componente  $\mathcal{D}_t\subset\mathcal{C}_t$ del corpo sia $\overline{F}(\mathcal{D}_t)=\int_{\mathcal{D}_t}\overline{f}_v(\overline{x};t)\,d\overline{x}$;
        \item \textbf{forze di superficie} che agiscono su $\partial\mathcal{C}_t$ e sono determinate da una densità superficiale $\overline{f}_s(\overline{x};t)$ così che la forza esercitata su $\Sigma_t\subset\partial\mathcal{C}_t$ sia $\overline{F}(\Sigma_t)=\int_{\Sigma_t}\overline{f}_s(\overline{x};t)\,d\overline{\sigma}$.
    \end{itemize}
\end{oss}

\noindent Come agiscono invece le forze interne?

\begin{pos}[Ipotesi di Cauchy]
    $\exists$ una densità superficiale di forza $\overline{f}_{s,int}(\overline{x},\hat{n})$ che fornisce la forza esercitata su $\mathcal{C}_t^-$ da $\mathcal{C}_t^+$
    \begin{displaymath}
        \overline{F}(\mathcal{C}_t^+,\mathcal{C}_t^-)=\int_{\Sigma}\overline{f}_{s,int}(\overline{x},\hat{n})\,d\overline{\sigma}
    \end{displaymath}
\end{pos}

\begin{teo}[Cauchy]
    Siano $\overline{\Phi}$ e $\overline{s}(\overline{x};\hat{n})$ due campi spaziali vettoriali che siano continui nei loro argomenti $\overline{x}\in\mathcal{C}_t$ e $\hat{n}\in\mathcal{N}$. Se $\forall \mathcal{D}_t\subset\mathcal{C}_t$ con bordo regolare
    \begin{displaymath}
    \boxed{
        \int_{\mathcal{D}_t}\overline{\Phi}(\overline{x})\,d\overline{x} + \int_{\partial\mathcal{D}_t}\overline{s}(\overline{x};\hat{n}(\overline{x}))\,d\overline{\sigma}=0
        }
    \end{displaymath}
    dove $\hat{n}(\overline{x})$ è la normale uscente da $\partial \mathcal{D}_t$, allora $\exists$ un campo tensoriale $T(\overline{x})$ t.c.
    \begin{displaymath}
    \boxed{
    \overline{s}(\overline{x};\hat{n})=T(\overline{x})\hat{n}
    }
    \end{displaymath}
\end{teo}

\begin{oss}
    Il teorema di Cauchy è una manifestazione del principio di azione e reazione a livello locale, perché $\overline{s}(\overline{x};-\hat{n})=-\overline{s}(\overline{x};\hat{n})$.
\end{oss}

\begin{teo}
    Per ogni corpo continuo,
    \begin{itemize}
        \item $\exists \; T(\overline{x};t)\in L(\mathbb{R}^3)$ detto \textbf{tensore degli sforzi} t.c. $\overline{f}_s(\overline{x};\hat{n};t)=T(\overline{x},t)\hat{n}$;
        \item vale la \textbf{I equazione indefinita} (quantità di moto): $\operatorname{div}T + \overline{f}_v=\rho\overline{a}$;
        \item vale la \textbf{II equazione indefinita} (momento angolare): $T=T^T$.
    \end{itemize}
\end{teo}

\begin{oss}
    $\operatorname{div}T(\overline{x};t):=\operatorname{tr}(dT(\overline{x};t))$.
\end{oss}

\begin{oss}
    Il motivo per cui le equazioni sono dette indefinite è che $T$ è a priori un tensore ignoto e finché non viene messo in relazione con le altre proprietà fisiche del corpo non si può chiudere l'equazione.
\end{oss}

\begin{oss}
    Il tensore degli sforzi deve però soddisafare una \underline{condizione al contorno} ovvero $T(\overline{x};t)\hat{n}\big|_{\partial\mathcal{C}_t}=\overline{f}_s(\overline{x},\hat{n}(\overline{x});t)$ dove  $f_s$ è la densità di forza su $\partial\mathcal{C}_t$ prodotta dalle forze esterne sul sistema.
\end{oss}

\begin{oss}
    Nel tensore degli sforzi è codificata un'informazione molto importante perché $T(\overline{x})$ permette di calcolare la forza agente sulle superfici del corpo passanti per $\overline{x}$.
\end{oss}

\begin{oss}
    Per poter applicare il teorema di Cauchy abbiamo implicitamente assunto che i campi di densità, accelerazione e sforzi siano continui, il che è un'ipotesi nascosta del teorema.
\end{oss}

\begin{oss}
    Data una generica superficie $\Sigma$ passante per $\overline{x}$ con normale $\hat{n}(\overline{x})$, la forza suprficiale è $T(\overline{x})\hat{n}(\overline{x})$  ma non è detto che sia $\parallel \hat{n}(\overline{x})$!
\end{oss}

\begin{defi}
   Lo \textbf{sforzo normale} è la quantità $\overline{s}_n=(\hat{n}T\hat{n})\hat{n}$, lo \textbf{sforzo di taglio} è la quantità $\overline{s}_{\tau}=T\hat{n}-\overline{s}_n$.
\end{defi}

\begin{oss}
    $\forall \overline{x}\in\mathcal{C}_t \; \exists \; \hat{e}_1,\hat{e}_2,\hat{e}_3$ t.c. le superfici $\perp \hat{e}_j$ in $\overline{x}$ sono soggette solo a sforzi normali dove gli $\hat{e}_j$ sono gli autovettori di $T(\overline{x})$. Le direzioni individuate da $\hat{e}_j$ sono dette \textbf{direzioni principali di sforzo} e i valori corrispondenti degli sforzi normali (autovalori di $T(\overline{x})$) sono gli \textbf{sforzi principali}.
\end{oss}

\begin{oss}
    La simmetria del tensore degli sforzi implica anche che $\sum_{j=1}^3\hat{e}_j\wedge(T\hat{e}_j)=0$.
\end{oss}

\begin{oss}
    Ricapitolando le equazioni del moto per un continuo sono:
    \begin{displaymath}
        \begin{dcases}
            \frac{\partial\rho}{\partial t} +\operatorname{div}(\rho\overline{v})=0\\
            \rho\left(\frac{\partial\overline{v}}{\partial t}+\overline{v}\cdot d\overline{v}\right)=\overline{f}_v+\operatorname{div}T\\
            \sum_{j=1}^3\hat{e}_j\wedge(T\hat{e}_j)=\overline{0} \quad \text{oppure} \quad T=T^T
        \end{dcases}
    \end{displaymath}
    A queste vanno aggiunte le condizioni iniziali:
        \begin{displaymath}
        \begin{cases}
            \rho(\overline{x},0)=\rho_0(\overline{x}) \\
            \overline{v}(\overline{x},0)=\overline{v}_0(\overline{x})
        \end{cases}
        \quad \forall \overline{x}\in\mathcal{C}^*
        \end{displaymath}
    e le condizioni al contorno:
    \begin{displaymath}
    T(\overline{x},t)\hat{n}(\overline{x},t)=\overline{f}_{s,ext}(\overline{x},t), \quad \overline{x}\in\partial\mathcal{C}_t
    \end{displaymath}
\end{oss}

\begin{oss}
    Sono 4 equazioni in 10 incognite!
\end{oss}

\section{Classi costitutive}

Per poter ottenere delle equazioni del moto che ammettano una soluzione determinata è necessario specificare meglio le proprietà del continuo e più precisamente fornire delle relazioni costitutive che caratterizzino la risposta del continuo alle sollecitazioni esterne e quindi nel concreto la dipendenza di $T$ da $\rho$ e $\overline{v}$. 

\begin{oss}
    Ulteriori restrizioni ai tensori degli sforzi possono essere dovute a
    \begin{itemize}
        \item richiesta di invarianza per cambiamento di osservatore;
        \item  simmetrie del continuo;
        \item vincoli interni (ad esempio l'incomprimibilità).
    \end{itemize}
\end{oss}

\begin{defi}
    Un \textbf{fluido} è un continuo comprimibile t.c. in condizioni di equilibrio lo sforzo è sempre interamente normale, cioè $\overline{s}_{\tau}=\overline{0}$.
    
    \noindent Per un fluido in condizioni di equilibrio $\exists$ quindi una funzione $p(\overline{x})$ detto campo di \textbf{pressione idrostatica} t.c. $T(\overline{x})=-p(\overline{x})\mathbbm{1}$. 
    
    \noindent Più in generale Un fluido è detto \textbf{perfetto} se $\exists \;p(\overline{x},t)$ campo di \textbf{pressione} t.c.
    \begin{displaymath}
    \boxed{
        T(\overline{x},t)=-p(\overline{x},t)\mathbbm{1}
        }
    \end{displaymath}
\end{defi}

\begin{oss}
    Per un fluido all'equilibrio il campo $\overline{v}$ deve essere stazionario, cioè $\frac{\partial \overline{v}}{\partial t}=\overline{0}$.
\end{oss}

\begin{prop}
    Un fluido può essere all'equilibrio $\iff$ il campo delle forze è irrotazionale cioè $\nabla \wedge \overline{f}_v=\overline{0}$ e in tal caso le equazioni indefinite si riducono a
    \begin{displaymath}
        \begin{dcases}
            \frac{\partial \rho}{\partial t}+\operatorname{div}(\rho\overline{v})=0 \\
            \nabla p=\overline{f}_v \quad (\textbf{equazione di Stevino})
        \end{dcases}
    \end{displaymath}
\end{prop}

\begin{proof}
    Se $T=-p\mathbbm{1}$ la prima equazione indefinita diventa $\overline{0}\overset{\text{eq.}}{=}\rho\overline{a}=\overline{f}_v-\nabla p$ (eq. di Stevino), ma poiché $\nabla\wedge(\nabla p)=\overline{0}\;\forall$ funzione regolare $p$ allora $\nabla\wedge\overline{f}_v=\overline{0}$.
\end{proof}

\begin{oss}
    Le equazioni (4 scalari) per un fluido non sono ancora sufficienti per determinare i 5 campi $\rho, \overline{v}, p$ ed è necessario aggiungere una relazione costituiva del tipo $\phi(p,\rho)=0$
    \begin{itemize}
        \item $\rho =$ cost.: fluido incomprimibile;
        \item $p=k\rho$: gas perfetto isotermo;
        \item $p=k\rho^{\gamma}$: gas perfetto adiabatico;
        \item $p=p(\rho)$: fluido perfetto in equilibrio ($+\operatorname{div}\overline{v}=0$ fluido perfetto incomprimibile).
    \end{itemize}
\end{oss}

\begin{oss}
    Per un fluido stazionario la condizione al contorno diventano
    \begin{displaymath}
        -p(\overline{x})\hat{n}=\overline{f}_{s,ext}(\overline{x}) \quad \forall \overline{x}\in \partial \mathcal{C}_t
    \end{displaymath}
    Quindi:
    \begin{itemize}
        \item la pressione è in modulo uguale alla densità di forza applicata;
        \item le superficie del fluido è $\perp$ alla direzione della forza applicata.
    \end{itemize}
\end{oss}

\begin{oss}
    Per i fluidi perfetti le equazioni sono
    \begin{displaymath}
        \begin{dcases}
        \frac{\partial \rho}{\partial t} + \operatorname{div}(\rho\overline{v})=0 \\
        \rho\left(\frac{\partial\overline{v}}{\partial t} +\overline{v}\cdot d\overline{v}\right) +\nabla p=\overline{f}_v  \end{dcases}
    \end{displaymath}
\end{oss}

\begin{oss}
    Per i fluidi perfetti, incomprimibili e omogenei valgono le \textbf{equazioni di Eulero}
    \begin{displaymath}
        \begin{dcases}
        \operatorname{div}\overline{v}=0 \\ \frac{\partial\overline{v}}{\partial t} +\overline{v}d\overline{v} +\frac{\nabla p}{\rho}=\frac{\overline{f}_v}{\rho}
        \end{dcases}
    \end{displaymath}
    dove $\overline{f}$ è la forza per unità di massa.
\end{oss}

\begin{teo}[Bernoulli]
\everymath{\displaystyle}
    Dato un fluido perfetto, incomprimibile e omogeneo t.c. 
    \begin{itemize}
        \item il campo di velocità è stazionario cioè $\frac{\partial\overline{v}}{\partial t}=\overline{0}$;
        \item le forze di volume sono conservative cioè $\exists \; U$ t.c. $\overline{f}_v=-\nabla U$;
    \end{itemize}
    allora la quantità $E=\frac{1}{2}\overline{v}^2+\frac{p}{\rho}+\frac{U}{\rho}$ è costante lungo le linee di flusso del sistema.
\end{teo}

\begin{proof}
    Prendendo la seconda equazione di Eulero e riscrivendola usando l'identità del lemma e $\frac{\partial\overline{v}}{\partial t}=\overline{0}$:
    \begin{displaymath}
        \frac{1}{2}\nabla v^2-\overline{v}\wedge(\nabla\wedge\overline{v})+\nabla\left(\frac{p+U}{\rho}\right)=0
    \end{displaymath}
    Poiché $\frac{\partial v}{\partial t}=0$ le linee di corrente e di flusso coincidono: sia $\hat{\tau}=\frac{\overline{v}}{|\overline{v}|}$ il vettore tangente a tali linee, allora
    \begin{displaymath}
        \frac{1}{2}\hat{\tau}\cdot\nabla (v^2)+\frac{1}{\rho}\nabla(p+U)\cdot\hat{\tau}=0 \implies \hat{\tau}\cdot\nabla\left\{\frac{1}{2}v^2+\frac{p+U}{\rho}\right\}=0
    \end{displaymath}
    cioè $E$ non varia lungo le linee di flusso.
\end{proof}

\begin{lemma}
    $\overline{v}\cdot d\overline{v}=\frac{1}{2}\nabla\overline{v}^2 -\overline{v}\wedge(\nabla\wedge\overline{v})$.
\end{lemma}

\begin{defi}
    Un \textbf{fluido viscoso} o \textbf{newtoniano} è definito dalla relazione costitutiva
    \begin{displaymath}
    \boxed{
        T=-p\mathbbm{1} + \lambda div(\overline{v})\mathbbm{1}+2\mu D
        }
    \end{displaymath}
    dove $\lambda, \mu >0$ sono i \textbf{coefficienti di viscosità} e $D$ è il tensore velocità di deformazione cioè $D=\frac{1}{2}(d\overline{v}+d\overline{v}^T)$.
\end{defi}

\begin{prop}
     Le equazioni del moto di un fluido newtoniano sono le \textbf{equazioni di Navier-Stokes}
     \begin{displaymath}
         \begin{dcases}
            \frac{\partial\rho}{\partial t} + \operatorname{div}(\rho\overline{v})=0\\
            \rho\left(\frac{\partial\overline{v}}{\partial t}+\overline{v}\cdot d\overline{v}\right) -(\lambda+\mu)\nabla \operatorname{div}(\overline{v}) -\mu\Delta\overline{v} +\nabla p = \overline{f}_v
        \end{dcases}
     \end{displaymath}
\end{prop}

\begin{proof}
    Dimostriamo che per un flusso newtoniano la divergenza dello sforzo è data da
    \begin{displaymath}
        \operatorname{div}T=-\nabla p+(\lambda+\mu)\nabla(\operatorname{div}\overline{v})+\mu\Delta\overline{v}
    \end{displaymath}
    Infatti
    \[
    \begin{split}
        (\operatorname{div}T)_i&=\sum_k\frac{\partial T_{ki}}{\partial x_k}=\sum_k\frac{\partial}{\partial x_k}\left\{(-p+\lambda\operatorname{div}\overline{v})\delta_{ki}+\cancel{\frac{2}{2}}\mu\left(\frac{\partial v_k}{\partial x_i}+\frac{\partial v_i}{\partial x_k}\right)\right\} \\
        &=\frac{\partial}{\partial x_i}(-p+\lambda\operatorname{div}\overline{v})+\mu\sum_k\left\{\frac{\partial^2v_k}{\partial x_i\partial x_k}+\frac{\partial^2v_i}{\partial x_k^2}\right\} \\
        &=\frac{\partial}{\partial x_i}(-p+\lambda\operatorname{div}\overline{v})+\mu\frac{\partial}{\partial x_i}(\operatorname{div}\overline{v})+\mu\Delta v_i
    \end{split}
    \]
\end{proof}

\begin{oss}
    Le equazioni di Navier-Stokes sono 4 equazioni in 5 incognite!
\end{oss}

\begin{oss}
    Per un fluido incomprimibile si aggiunge l'equazione $\operatorname{div}(\overline{v})=0$.
\end{oss}

\begin{oss}
    Se in aggiunta il fluido è omogeneo $\rho = cost$.
\end{oss}

\begin{oss}
    Le condizioni al contorno tipiche sono quelle \textbf{no slip} (non scorrimento), cioè la velocità relativa fluido/contenitore nulla che nel caso di contenitore fermo diventano $\overline{v}=\overline{0}$ al bordo.
\end{oss}

\end{document}